\documentclass{beamer}

%\usepackage[portuguese]{babel}
%\usepackage[utf8]{inputenc}


\usepackage[brazilian]{babel}
\usepackage[T1]{fontenc} 
\usepackage{ae} 
\usepackage[utf8]{inputenc} 

\usepackage{amsmath}
\usetheme{Copenhagen}
\usecolortheme{beaver}

%\useinnertheme{umbcboxes}
%\setbeamercolor{umbcboxes}{bg=violet!12,fg=black}

\usepackage{rotating} % for defining \schwa
\newcommand{\schwa}{\raisebox{1ex}{\begin{turn}{180}e\end{turn}}}

\newcommand{\arcsinh}{\mathop\mathrm{arcsinh}\nolimits}
\newcommand{\arccosh}{\mathop\mathrm{arccosh}\nolimits}
\newcommand{\codim}{\mathop\mathrm{codim}\nolimits}

\newcommand{\Pu}{P_{\mathrm{amb}}}

\newcommand{\xfase}{\mathbf{x}}
\newcommand{\yfase}{\mathbf{y}}
\newcommand{\qfase}{\mathbf{q}}
\newcommand{\pfase}{\mathbf{p}}
\newcommand{\xifase}{ {\boldsymbol{\xi}} }
\newcommand{\mufase}{ {\boldsymbol{\mu}} }
\newcommand{\Ifase}{\mathbf{I}}
\newcommand{\Pfase}{\mathbf{P}}
\newcommand{\Scat}{\mathbf{S}}
\newcommand{\Jsimp}{\mathbf{J}}
\newcommand{\Dom}{\mathbb{D}}
\newcommand{\Var}{\mathbb{M}}
\newcommand{\bra}[1]{\langle #1|}
\newcommand{\ket}[1]{|#1\rangle}
\newcommand{\braket}[2]{\langle #1|#2\rangle}


\title{Espalhamento caotico em três graus de liberdade}

\author[W.P.K. Zapfe]{W. P. Karel Zapfe Zaldivar}
\institute[CBPF]{
  Centro Brasileiro de Pesquisas Físicas \\
  Rua Dr. Xavier Sigaud, 150 \\
  Urca, Rio de Janeiro,  RJ,  Brasil. \\ 
  CEP: 22290-180
  Rio de Janeiro, RJ. Brasil \\
}


\date{Julho, 2012}



%ppppppppppppppppppppppppppppppppp
\begin{document}

%----------- titlepage ----------------------------------------------%
\begin{frame}

  \titlepage
  
\end{frame}


%----------- slide --------------------------------------------------%
\begin{frame}
  \frametitle{Obrigado}
  
  Leaonardo C.\\
  \begin{center}
    \includegraphics[width=0.8\textwidth]{OrderChaos.jpg}
    \small{Peter Crawley}
  \end{center}   
\end{frame}


\begin{frame}
  \frametitle{Uma particula em movimiento}  
 \begin{center}
 \includegraphics[width=0.8\textwidth]{EspacioFase6Dim01.pdf}
 \end{center} 
 \begin{equation}
   H(\qfase,\pfase)=E
 \end{equation}
\end{frame}

\begin{frame}
  \frametitle{Uma Simetria contínua}  
 \begin{center}
 \includegraphics[width=0.5\textwidth]{PotencialCuadrado02.pdf}
 \begin{equation}
 \begin{split}
   R_\theta(H)&=H, \\
   p_i&=cte, \\
   q_i(t)& =q_i(0)+tp_i 
 \end{split}
 \end{equation}
 \end{center} 
\end{frame}

\begin{frame}
  \frametitle{A seção de Poincaré}
  \begin{center}
    \includegraphics[width=0.5\textwidth]{mapsandflow.png}
    \begin{equation}
      f(\qfase,\pfase)=cte.
    \end{equation}
  \end{center} 
\end{frame}    


\begin{frame}
  \frametitle{A maranha hetereo/homoclínica.}
  Tem essencialmente a dinámica de uma ferradura de Smale.
 \begin{center}
 \includegraphics[width=0.8\textwidth]{1001_dinamica.png}
 \end{center} 
\end{frame}


\begin{frame}
  \frametitle{A desenvolvimento das ferraduras, $E$ fixos.}
  Variando $p_i$.
  \begin{center}
    \begin{tabular}{cc}
      \includegraphics[width=0.4\textwidth]{1000_dinamica.png}&
      \includegraphics[width=0.4\textwidth]{1003_dinamica.png}\\
      \includegraphics[width=0.4\textwidth]{1006_dinamica.png}&
      \includegraphics[width=0.4\textwidth]{1009_dinamica.png}
    \end{tabular}
  \end{center}
\end{frame}


\begin{frame}
  \frametitle{As funções de espalhamento}.
\begin{center}
 \includegraphics[width=0.8\textwidth]{1000_funcion.png}
 \end{center} 
\end{frame}



\begin{frame}
  \frametitle{O momento de espalhamento reflete
    as propriedades das ferraduras}.
  \begin{center}
    \begin{tabular}{cc}
      \includegraphics[width=0.4\textwidth]{1000_funcion.png}&
      \includegraphics[width=0.4\textwidth]{1003_funcion.png}\\
      \includegraphics[width=0.4\textwidth]{1006_funcion.png}&
      \includegraphics[width=0.4\textwidth]{1009_funcion.png}
    \end{tabular}
  \end{center}
\end{frame}



\begin{frame}
  \frametitle{Contudo\ldots}
  Estamos trapasando...\\
  Não é fácil medir a funcão  de espalhamiento no laboratório!
  O que é medido é a seção de choque.
  \begin{center}
  \includegraphics[width=0.8\textwidth]{explicasecaodechoque.pdf}
  \end{center}
\end{frame}



\begin{frame}
  \frametitle{As singularidades arcoiris}
  \begin{center}
  \includegraphics[width=0.8\textwidth]{explicaarcoiris.pdf}
  \end{center}
\end{frame}


\begin{frame}
  \frametitle{Um exemplo}
  \begin{center}
  \includegraphics[width=0.8\textwidth]{SeccionSymetrica.png}
  \end{center}
\end{frame}



\begin{frame}
  \frametitle{Intersecção transversal}
  O parámetro $L$ controlava o desenvolvimiento das ferraduras.
  Consideremos, então, a união de todas as folhas dinamicamente
  independentes.
\begin{center}
  \includegraphics[width=0.8\textwidth]{Union.pdf}
\end{center}
\end{frame}




\begin{frame}
  \frametitle{Uma estrutura conhecida aparece}
  \begin{center}
    \includegraphics[width=1.0\textwidth]{1004_seccion_.png}
  \end{center}
\end{frame}

\begin{frame}
  \frametitle{Ao isolar as singularidades}
  \begin{center}
    \includegraphics[width=1.0\textwidth]{1004_arcoiris-parapresentacion.png}
  \end{center}
\end{frame}


\begin{frame}
  \frametitle{Um único dominio de continuidade}
  \begin{center}
    \includegraphics[width=1.0\textwidth]{1005_singlerainbow.png}
  \end{center}
\end{frame}


\begin{frame}
  \frametitle{Cáusticas do tipo umbílico hiperbólico}
  \begin{center}
    \includegraphics[width=0.7\textwidth, angle=90]{CausticGlassBerry.png}
  \end{center}
\end{frame}  



\begin{frame}[allowframebreaks]{Referências} 

\begin{thebibliography}{9}

\bibitem[Ber76]{Berry76}
M.~V. Berry.
\newblock Waves and Thom's theorem.
\newblock {\em Advances in Physics}, 25(1):1--26, 1976.

\bibitem[dA88]{OdeA88}
A.M.~Ozorio de~Almeida.
\newblock {\em Hamiltonian Systems: Chaos and Quantization}.
\newblock Oxford Univerity Press, 1988.


\bibitem[JORLA08]{JungLuna}
C.~Jung, G.~Orellana-Rivadeynera, and G.~Luna-Acosta.
\newblock Reconstruction of the chaotic set from cross section data.
\newblock {\em J. Phys. A: Math. Gen.}, 38:567, 2008.

\bibitem[Jun90]{Jung90}
C.~Jung.
\newblock Fractal properties in the semiclassical scattering cross section of a
  classically chaotic system.
\newblock {\em J. Phys. A: Math. Gen.}, 23:1217, 1990.

\bibitem[JMSZ10]{yo2010}
C.~Jung, 0.~Merlo, T.H. Seligman, and W.P.K. Zapfe.
\newblock The chaotic set and the cross section for chaotic scattering with
  three degrees of freedom.
\newblock {\em NJP}, 12:103021, 2010.

\bibitem[LitJohn]{LitJohn}
  R.G.~Littlejohn.
 \newblock Semmiclassical structure of trace formulas.
 \newblock {\em J. Math. Phys.} 31:2952, 1990 

\end{thebibliography}
\end{frame}



\begin{frame}
  \frametitle{Assimetria}
  Uma vez quebrada a simetria ($A\neq 0$) \ldots
    \begin{center}
  \includegraphics[width=0.8\textwidth]{1000_funcion_pm3d.png}
  \end{center}
\end{frame}


\begin{frame}
  \frametitle{Para economizar tiempo \ldots}
  \ldots uma transformação simpletica gerada pela função:
  
  \begin{equation}
    G(q,\tilde{p},\theta,\tilde{L})=
    q\tilde{p}+\theta\tilde{L}+(L_{max}-L)(1+A \cos\theta)(-\exp(-q^2)),
  \end{equation}  
    e um vôo livre:
\begin{align}
 \tilde{q} &= q + p \\
 \tilde{\theta} &= \theta + L \\
 \tilde{p} &= p \\
 \tilde{L} &=  L.
\end{align}
  
\end{frame}



\begin{frame}
  \frametitle{A seção de choque clássica}
  \begin{equation}
    g(p_{out})=
   \displaystyle\sum_k \bigg| \frac{d p_{out}}{d q_{in,k}} \bigg|^{-1}
  \end{equation}
Isto é simplesmente a medida induzida na superficie
Lagrangiana especificada pela ação das trajectorias selecionadas.
\begin{equation}
  \sigma(p_{out})=\lambda(q)\bigg|\frac{ \partial q}{\partial p_{out}}\bigg|
\end{equation}
\end{frame}


\begin{frame}
  \frametitle{A seção de choque novamente}

  Com dois \emph{observáveis}:
  \begin{equation}
    \mathbf{I}_{out}(q_{in},\theta_{in})=
      (p_{out}(q_{in},\theta_{in}),\Delta L (q_{in},\theta_{in})).
\end{equation}
 \begin{equation}
    g(\mathbf{I}_{out})=
    \displaystyle\sum_k \bigg| \frac{\partial (p_{out},\Delta L)}
                     {\partial (q_{in,k},theta_{in,k}}) \bigg|^{-1}
  \end{equation}
\end{frame}


\begin{frame}
  \frametitle{Localmente}
  \begin{center}
  \includegraphics[width=0.8\textwidth]{localmente.pdf}
  \end{center}

  \begin{equation}
  \codim W_1 + \codim W_2 = \codim(W_1 \cap W_2)  
  \end{equation}
\end{frame}




\begin{frame}
  \frametitle{Cáusticas do tipo umbílico hiperbólico 2}
  \begin{align}
    p_{out} &=p_0 -l(q_0-q_{in})+b \cos \theta, \\
    \Delta L &=\frac{a \sin\theta}{1-c(q_0-q_{in})}
   \end{align}
  
\begin{center}
    \includegraphics[width=0.7\textwidth]{Gelatina.jpg}
\end{center}
\end{frame}  




\begin{frame}
  \frametitle{A amplitude de espalhamiento semiclássica}
  
  \begin{equation}
    f(\mathbf{I}_{out})=
    \displaystyle\sum_k \bigg| \frac{(\partial p_{out}, \Delta L)}
                    {\partial (q_{in,k}, \theta_{in,k})} \bigg|^{-1/2}
                    \exp(\frac{i}{\hbar}S_k (\mathbf{I}_{out})-\mu_j\pi/2)
  \end{equation}
  \begin{equation}
    g(\mathbf{I}_{out})=|f(\mathbf{I}_{out})|^2
   \end{equation}

Miller:
\begin{equation}
\braket{\mathbf{x}_{in}}{\mathbf{I}_{out}}=\frac{1}{(2i \pi \hbar)^{n/2}}
  \displaystyle\sum_{k}
  \bigg| \frac{\partial^2 S}
                    {\partial\mathbf{I}_{out}\partial\mathbf{x}_{in}} \bigg|^{1/2}
                    \exp(\frac{i}{\hbar}S_k (\mathbf{x}_{in})-\mu_j\pi/2)
\end{equation}


\end{frame}

\begin{frame}
  \frametitle{Ação final}
  \begin{center}
    \includegraphics[width=1.0\textwidth]{1000_acaofinal.png}
  \end{center}
\end{frame}  




\end{document}
