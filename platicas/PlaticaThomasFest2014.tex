\documentclass{beamer}

%\usepackage[portuguese]{babel}
%\usepackage[utf8]{inputenc}


\usepackage[british]{babel}
\usepackage[T1]{fontenc} 
\usepackage{ae} 
\usepackage[utf8]{inputenc} 
%Este paquete al parecer usa mejor los nombres de graphicas
\usepackage{grffile}
\usepackage{amsmath}
\usetheme{Copenhagen}
\usecolortheme{beaver}
\usefonttheme{serif}

%\useinnertheme{umbcboxes}
%\setbeamercolor{umbcboxes}{bg=violet!12,fg=black}

\usepackage{rotating} % for defining \schwa
\newcommand{\schwa}{\raisebox{1ex}{\begin{turn}{180}e\end{turn}}}

\newcommand{\arcsinh}{\mathop\mathrm{arcsinh}\nolimits}
\newcommand{\arccosh}{\mathop\mathrm{arccosh}\nolimits}
\newcommand{\codim}{\mathop\mathrm{codim}\nolimits}

\newcommand{\Pu}{P_{\mathrm{amb}}}

\newcommand{\ihb}{\frac{i}{\hbar}}
\newcommand{\xfase}{\mathbf{x}}
\newcommand{\yfase}{\mathbf{y}}
\newcommand{\qfase}{\mathbf{q}}
\newcommand{\pfase}{\mathbf{p}}
\newcommand{\xifase}{ {\boldsymbol{\xi}} }
\newcommand{\mufase}{ {\boldsymbol{\mu}} }
\newcommand{\Ifase}{\mathbf{I}}
\newcommand{\Pfase}{\mathbf{P}}
\newcommand{\Scat}{\mathbf{S}}
\newcommand{\Jsimp}{\mathbf{J}}
\newcommand{\Dom}{\mathbb{D}}
\newcommand{\Var}{\mathbb{M}}
\newcommand{\bra}[1]{\langle #1|}
\newcommand{\ket}[1]{|#1\rangle}
\newcommand{\braket}[2]{\langle #1|#2\rangle}
\newcommand{\Prom}[2]{\langle #1\rangle_#2}
\newcommand{\rd}{\!\mathrm{d}}


\title{Sub-Planckian Structures, Chaos and Numerics: The Ergodic Conjecture revisited.}
\subtitle{Work in progress}

\author[CBPF]{A. M. Ozorio de Almeida, 
F. Toscano,  E. Zambrano \\ and W.P. Karel Zapfe}


\date{July, 2014}


%ppppppppppppppppppppppppppppppppp
\begin{document}

%----------- titlepage ----------------------------------------------%
\begin{frame}

  \titlepage
  
\end{frame}


%----------- slide --------------------------------------------------%
\begin{frame}
  \frametitle{Order and Chaos}
  \begin{center}
    \includegraphics[width=0.8\textwidth]{OrderChaos.jpg}
    \\
    \small{Peter Crawley}
  \end{center}   
\end{frame}


\begin{frame}
  \frametitle{Intuitive Ergodicity}  
 \begin{tabular}{cc}
 \includegraphics[width=0.45\textwidth]{ErgodicTrayectories01.png} &
 \includegraphics[width=0.45\textwidth]{ErgodicProyection01.png}
 \end{tabular} 
 \begin{equation}
   H(\qfase,\pfase)=E
 \end{equation}
Our work uses the NELSON potential,
 \begin{equation}
   H(\qfase,\pfase)=\|\pfase\|^2+
   q_x^2/20+(q_y-q_x/2)^2
 \end{equation}
\end{frame}

\begin{frame}
  \frametitle{Quantum Stationary States}
  In the Quantum framework, our ``Energy surfaces'' are
  states labeled by Energy. 
  \begin{tabular}{ccc}
    \includegraphics[width=0.30\textwidth]{EigenEstadosNelson-0.1875-1003.png} &
    \includegraphics[width=0.30\textwidth]{EigenEstadosNelson-0.1875-1024.png} &
    \includegraphics[width=0.30\textwidth]{EigenEstadosNelson-0.1875-1045.png} \\
    \includegraphics[width=0.30\textwidth]{EigenEstadosNelson-0.1875-1087.png} &
    \includegraphics[width=0.30\textwidth]{EigenEstadosNelson-0.1875-1126.png} &
    \includegraphics[width=0.30\textwidth]{EigenEstadosNelson-0.1875-1186.png} 
 \end{tabular}

\end{frame}

\begin{frame}
  \frametitle{Scars}
  \begin{center}
    \includegraphics[width=0.85\textwidth]{Baranger1993-02.png} 
  \end{center}
  \emph{Semiclassical Calculations of scars for a Smooth Potential},
  D. Provost and M. Baranger, PRE 71(5), 1993.

\end{frame}    

\begin{frame}
  \frametitle{Scars}
  \begin{center}
    \includegraphics[width=0.85\textwidth]{Baranger1993-02.png} 
  \end{center}
  \emph{Semiclassical Calculations of scars for a Smooth Potential},
  D. Provost and M. Baranger, PRE 71(5), 1993.

\end{frame}    


\begin{frame}
  \frametitle{The Wigner or centre function}
  $\xfase:=(\qfase_x, \pfase_x)$
  \begin{equation}
    W(\xfase)=\frac{1}{(2\pi\hbar)^d}\int \rd \xi \bra{\qfase 
      +\xi/2}\hat{\rho}\ket{\qfase-\xi/2}\exp(-\ihb \pfase \cdot \xi) 
  \end{equation}
  In a more down to Earth  expression, for a pure state:
  \begin{equation}
    W(\xfase)=\frac{1}{(2\pi\hbar)^d}\int \rd \xi \psi^*(\qfase 
      +\xi/2)\psi(\qfase-\xi/2)\exp(-\ihb \pfase \cdot \xi) 
  \end{equation}
  
    \emph{On the quantum correction for thermodynamic equilibrium},\\
    E. Wigner, Phys. Rev. 40, 1932.

  \end{frame}    

\begin{frame}
  \frametitle{The Quantum Ergodic Hypothesis}
  Classical Ergodic Distribution:
  \begin{equation}
  \rho(\xfase)=\frac{1}{C_E} \delta(H(\xfase)-E)
  \end{equation}
  As stated by Voros and Berry:
  \begin{equation}
    W(\xfase)\rightarrow\frac{1}{C_E} \delta(H(\xfase)-E)
  \end{equation}
  \emph{Semiclassical approximations},\\
  A.Voros, Ann. de l'Inst. H.P., 24(1), 1976.\\
  \emph{Regular and Irregular semiclassical wavefunction},\\
  M.V. Berry, J. Phys. A, 10 (12), 1977.
\end{frame}

\begin{frame}
  \frametitle{Some objections}
  \begin{itemize}
  \item Hudsons Theorem: The \emph{only} non negative Wigner function
    there is  corresponds to the Coherent State.
  \item Balasz Proof: A Dirac Delta on a curved Manifold does not represent
    a positive operator.
  \item Correlations of a Wigner Function are invariant with respect to
    Fourier Transforms.
  \end{itemize}
  \emph{When is the Wigner quasi-probability density non-negative},\\
  R.L. Hudson, Rep. Math. Phys, 6 , 1974.\\
  \emph{Weyl's association, Wigner's function and affine geometry},\\
  N. Balazs. Physica A. 102(2), 1980 
    \emph{ S. Chountasis and A. Vourdas, Phys. Rev. A, 1998.},\\
  S. Chountasis and A. Vourdas, Phys. Rev. A, 1998.
\end{frame}


\begin{frame}
  \frametitle{The Need for a better Wigner Ergodic Function}
  \begin{equation}
    \int \rd \xfase W(\xfase) A(\xfase)=\langle \hat{A} \rangle=
    \langle A \rangle_{\text{classic}}.
    \end{equation}
    A bit crude actually.
    \begin{tabular}{cc}
      \includegraphics[width=0.4\textwidth]{2GatosCercanos01.png} &
      \includegraphics[width=0.4\textwidth]{2GatosNotanCercanos01.png} 
    \end{tabular}
  \emph{Scattering Phase of Quantum Dots\ldots},\\
  R. Molina \emph{et all}, Phys. Rev. Lett, 108, 2012.
\end{frame}    


\begin{frame}
  \frametitle{The Weyl or chord function}
  The Simplectic Fourier Transform of the Wigner Function.
  \begin{equation}
    \chi(\xifase)=\frac{1}{(2\pi\hbar)^d}\int \rd \qfase \bra{\qfase 
      +\xi_q/2}\hat{\rho}\ket{\qfase-\xi_q/2}\exp(+ \ihb \qfase \cdot \xi_p) 
  \end{equation}
   \begin{equation}
    \chi(\xifase)=\int \rd \xfase W(\xfase) \exp(\ihb \xifase\wedge \xfase)
  \end{equation}
  \begin{equation}
    \chi(\xifase)= \langle \hat{T}_{-\xifase} \rangle
  \end{equation}
  \emph{The Weyl Representation in Classical and Quantum Mechanics},\\
  A. M. Ozorio de Almeida, Phys. Rep. 295, 1998.
\end{frame}    


\begin{frame}
  \frametitle{The Blind Spots}
  \begin{center}
    Where $\chi(\xifase)=0$. \\
    \includegraphics[width=0.75\textwidth]{EduardoNJPBlindSpots.jpg} 
  \end{center}
  \emph{Blind Spots between Quantum States},\\
  A. M. Ozorio de Almeida and E. Zambrano, NJP 11, 2009.
  \emph{Semiclassical Theory for Small Displacements},\\
  A. M. Ozorio de Almeida and E. Zambrano, J. Phys. A 43, 2010.    
\end{frame}    


\begin{frame}
  \frametitle{The First Blind Spots} 
  \begin{equation} \label{expansionpot}
    \chi(\xifase)=1-\ihb \Prom{\xfase}{E}\wedge\xifase
    -\frac{1}{2 \hbar^2}\Prom{\xfase\wedge\xifase}{E}^2
    +\frac{i}{6 \hbar^3 }\Prom{\xfase\wedge\xifase}{E}^3+\ldots
  \end{equation} 
  We use the cumulants to \emph{fhe first} nodal lines.\\
\emph{Semiclassical Theory for Small Displacements},\\
A. M. Ozorio de Almeida and E. Zambrano, J. Phys. A 43, 2010.    
\end{frame}    


\begin{frame}
  \frametitle{A weaker Ergodic Conjecture}
  \begin{equation}
    \chi_E(\xifase) \rightarrow N_E \int \rd\xfase \; \exp\left\{\frac{i}{\hbar}(\xifase\wedge \xfase)\right\} \; \delta(H(\xfase)-E).
  \end{equation}
  \label{weakhyp}
Let us just throw any crazy distribution near the energy surface:
Proposal 1: Random points. 
  \begin{equation}
    W(\xfase) = \frac{1}{N}\sum_{H(\xfase_j)=E} \delta(x-x_j)
  \end{equation}
\end{frame}    



\begin{frame}
  \frametitle{Trivially ergodic: 1D HA}
  \begin{center}
  \includegraphics[width=0.5\textwidth]{CondiniExemplo01.png}
  \end{center}
  \begin{equation}\label{BerryApproach}
    \chi(\xifase)=J_0(|\xifase|m(E)^{1/2}/\hbar),
  \end{equation}
\end{frame}


\begin{frame}
\begin{tabular}{cc}
\includegraphics[width=0.5\textwidth]{BesselRadial01.png} &
\includegraphics[width=0.5\textwidth]{CuerdaspaceZerosLowData.png} \\
\includegraphics[width=0.5\textwidth]{CuerdaspaceZeroMediumData.png} &
\includegraphics[width=0.5\textwidth]{CuerdaspaceZeroHighData.png} 
\end{tabular}
\\
Exact, 500, 5000, 50'000 points.
\end{frame}


\begin{frame}
  \begin{tabular}{cc}
    \includegraphics[width=0.4\textwidth]{200_ZerosBandN.pdf} &
    \includegraphics[width=0.4\textwidth]{2000_ZerosBandN.pdf} \\
    \includegraphics[width=0.4\textwidth]{20000_ZerosBandN.pdf} &
    \includegraphics[width=0.4\textwidth]{200000_ZerosBandN.pdf} 
  \end{tabular}\\
200, 2000, 20'000, 200'000 points.
\end{frame}


\begin{frame}
\frametitle{An artificial 4D example}
  \begin{equation}\label{BerryApproach}
    \chi(\xifase)=J_1(|\xifase|m(E)^{1/2}/\hbar), \; \hbar=1
  \end{equation}  
  \begin{center}
    \includegraphics[width=0.9\textwidth]{2000_Deltas3Sphere.pdf}\\
    2000 points.   
  \end{center}
\end{frame}


\begin{frame}
\frametitle{An artificial 4D example}
  \begin{equation}\label{BerryApproach}
    \chi(\xifase)=J_1(|\xifase|m(E)^{1/2}/\hbar), \; \hbar=1
  \end{equation}  
  \begin{center}
    \includegraphics[width=0.9\textwidth]{2000_Deltas3Sphere.pdf}\\
    2000 points.   
  \end{center}
\end{frame}


\begin{frame}
 It is still very good.
  \begin{center}
    \includegraphics[width=0.9\textwidth]{20000_Deltas3Sphere6Zeros.pdf}
    \\
    20000 points.   
  \end{center}
\end{frame}


\begin{frame}
  \frametitle{Enough foolin' around}
  A true Hamiltonian function:
   \begin{equation}
   H(\qfase,\pfase)=\|\pfase\|^2+
   q_x^2/20+(q_y-q_x/2)^2
   \end{equation}
   \begin{tabular}{cc}
     \includegraphics[width=0.49\textwidth]{NubeDeltasQProjection01.png} &
     \includegraphics[width=0.49\textwidth]{NubeDeltasYProjection01.png} \\
     $q$-space projection. &
     $\yfase$-projection.     
   \end{tabular}
\end{frame}


\begin{frame}
  We choose a section (the chord that corresponds to the $\yfase$ centre).
    \begin{tabular}{cc}
      \includegraphics[width=0.5\textwidth]{FabricioWeyl02-ZerosContour.pdf} &
      \includegraphics[width=0.5\textwidth]{KarelWeyl-ZerosContour.pdf}
    \end{tabular}
    And we fail misserably.
\end{frame}


\begin{frame}
  \frametitle{The Cumulants, again}
    \begin{equation} \label{expansionpot}
    \chi(\xifase)=1-\ihb \Prom{\xfase}{E}\wedge\xifase
    -\frac{1}{2 \hbar^2}\Prom{\xfase\wedge\xifase}{E}^2
    +\frac{i}{6 \hbar^3 }\Prom{\xfase\wedge\xifase}{E}^3+\ldots
  \end{equation} 
  \begin{tabular}{cc}
     \includegraphics[width=0.5\textwidth]
     {ExactoN_0821_WeylAprox3grado-0-0-ZerosContour.pdf} &
     \includegraphics[width=0.5\textwidth]{FabricioWeyl02-ZerosContour.pdf} 
    \end{tabular}
    Looks \emph{slightly better}
\end{frame}

\begin{frame}
  \frametitle{Axis by axis.}
  \begin{tabular}{cc}
    \includegraphics[width=0.49\textwidth]{ComparaAprroxyFab01.pdf} &
    \includegraphics[width=0.49\textwidth]{ComparaAprroxyFab02.pdf} \\
    \includegraphics[width=0.49\textwidth]{ComparaMuQAxis.pdf} &
    \includegraphics[width=0.49\textwidth]{ComparaMuQAxisZoomout.pdf}
    \end{tabular}
    Looks \emph{slightly better}
\end{frame}



\begin{frame}
  \frametitle{Axis by Axis II}
  \begin{tabular}{cc}
    \includegraphics[width=0.5\textwidth]{Compara3ordenRealMuP.pdf} &   
    \includegraphics[width=0.5\textwidth]{Compara3ordenImagMuP.pdf} \\
    \includegraphics[width=0.50\textwidth]{Compara3ordenRealMuQ.pdf} &
    \includegraphics[width=0.50\textwidth]{Compara3ordenImagMuQ.pdf}
 \end{tabular}    
 
\end{frame}




\end{document}
