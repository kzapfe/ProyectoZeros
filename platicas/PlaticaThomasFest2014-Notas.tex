Intuitive Ergodicity:

In classical mechanics, ergodicity means that
with as little as a single orbit we can approach any given point
in phase space as close as we like. Even more: allmost any
orbit that we choose will behave in such a way. In a conservative system,
this means that except for a negligible subset of orbits,
any orbit winds densely on the energy surface.
 
Here I show you the projection in configuration space
and in a three dimensional arbitrary projection the 
orbits of the so called Nelson potential. This potential was
apparentely motivated by Nuclear Physics, altough how it escapes me.
\cite{Baranger1987}.

Quantum Stationary States:

In the Quantum framework, a state characterized by its Energy
is a Stationary State. These give us more definite ``hot spots''
of probability for encountering a particle in a measurement. 
The patern of these high probability areas does not change with time.
The are, in the context of quantum chaos, called scars. 

Scars:
The Scars are associated with periodic orbits. More complex and denser scars
correspond usually to high period  Orbits. These high period orbits
should be good aproximations to aperiodic orbits, at least for a while. 
These scars are stationary, they do not spread out over the avaible
space. This is not an inherent contradiction per se, even in classical
ergodic systems we may have caustics, proyections either in q or p
space which are locally much denser than the average. But it is still
a curios feature: the most prominent orbits are the ones that are most
far from ergodicity, in fact, classically, they are negligible.

The Wigner or centre function:
This real is a quasi distribution function which faithfully represents
a Quantum State over a space which we identify as the classical phase space.  

The Quantum Ergodic Hypotesis:
In a classical ergodic system, we could obtain the time average
or ``expected value'' for any observable using a Dirac Delta distribution
on the Energy Surface. The Quantum Ergodic hypothesis, acording to Voros
and Berry, goes more or less like this: The Wigner function for a certain
Energy Eigenstate has to tend, semiclassicaly, to the classical Dirac
Delta Distribution.

Some Objections:
It seems that in that extreme form, the QEH cannot be. There are some
theoretical issues which should prevent that limiting form.
Very simply, the only Wigner function which can be larger or equal to zero
everywhere in phase space is the one that corresponds to a Coherent State
for the Q Harm Osc. 
There are other more complicated Objections. The density matrix is 
a positive operator, so, the inverse transform of this limiting for
should be also a positive operator. But according to Balasz, this cannot be,
except in the very specific case of a flat energy surface.
The last one appeals to wave mechanical reasoning: The Correlations of
the Wigner function should be invariant with respect to Fourier Transforms.
That means that large scale structures such as the Energy shell should be
accompanied by small structure oscilations \emph{always}.


The Need.
Why should we try to search for a better Ergodic Wigner Function?
If the only use that we had was to give adecuate expected values,
then the usual Ergodic Conjecture would be enough. But we may be missing
a lot.  Notably in the aforementioned paper by Berry, is the deduction
of "local" statistical correlations which should appear as structures on
the Wigner Function. According to Rafael Molina and collaborators, these
structures shall be directly applicable to the experiments of Quantum Dots. 
Moreover, the experimental refinement is increassing, so we shall be able
to detect very delicate interference phenomena.

The Chord Function:



                