\documentclass[a4paper,12pt]{article}

\usepackage[utf8]{inputenc}
\usepackage{amsmath,amssymb}
\usepackage{graphicx}
%\usepackage{subfigure}
%\usepackage[spanish]{babel}
\usepackage{bm}
\usepackage{caption}
\usepackage{subcaption}

\usepackage[cm]{fullpage}
\usepackage[light]{antpolt}
\usepackage[T1]{fontenc}

\usepackage{float}

%Este paquete le pone una barrera a los floats
% al final de cada seccion
\usepackage[section]{placeins}


\bibliographystyle{alpha}

\newcommand{\ihb}{\frac{i}{\hbar}}
\newcommand{\xfase}{\mathbf{x}}
\newcommand{\yfase}{\mathbf{y}}
\newcommand{\qfase}{\mathbf{q}}
\newcommand{\pfase}{\mathbf{p}}
\newcommand{\xifase}{ {\boldsymbol{\xi}} }
\newcommand{\mufase}{ {\boldsymbol{\mu}} }
\newcommand{\Ifase}{\mathbf{I}}
\newcommand{\Pfase}{\mathbf{P}}
\newcommand{\Scat}{\mathbf{S}}
\newcommand{\Jsimp}{\mathbf{J}}
\newcommand{\Dom}{\mathbb{D}}
\newcommand{\Var}{\mathbb{M}}
\newcommand{\bra}[1]{\langle #1|}
\newcommand{\ket}[1]{|#1\rangle}
\newcommand{\braket}[2]{\langle #1|#2\rangle}
\newcommand{\Prom}[2]{\langle #1\rangle_{#2}}


\DeclareMathOperator*{\cod}{cod}
\DeclareMathOperator*{\traza}{traza}


\title{ Weyl Functions for the Nelson Potential:\\ some puzzling comparisions}
\author{\\CBPF}


\begin{document}

\maketitle

\begin{abstract}
Continuing from ``troubleshooting01'', here I present a
possible reason for the appearence of the dissimil chord nodal
lines between Fabricio's computation and mine. 
\end{abstract}




\section{An example from image processing}

I am presenting all the images with the same scale factor
here. The images in the ``frequency domain'' are presented showing
only their amplitude in logarithmic scale, for simplicity 
(they are, in general, complex numbers).


\begin{figure}[H]
  \centering
  \begin{subfigure}[b]{0.48\textwidth}
    \centering
          \includegraphics[scale=0.40]{Original.png}
                \caption{A tyger}
                \label{tigre01}
  \end{subfigure}%
\begin{subfigure}[b]{0.48\textwidth}
    \centering
          \includegraphics[scale=0.40]{Trans_Original.png}
                \caption{The Fourier Transform of a tyger.}
                \label{ftigre01}
  \end{subfigure}
\\
  \begin{subfigure}[b]{0.48\textwidth}
    \centering
          \includegraphics[scale=0.40]{FFT_Padding_4x4.png}
                \caption{The Fourier transform with black padding.}
                \label{ftigreynegro01}
  \end{subfigure}%
\begin{subfigure}[b]{0.48\textwidth}
    \centering
          \includegraphics[scale=0.40]{Inv_FFT_Padding_4x4.png}
                \caption{The inverse transform of the previous picture.}
                \label{ftigreynegro01}
  \end{subfigure} 
\caption{ A simple trick for rendering on screen or on paper
an enlargement of a digital image. The original photo is Fourier transformed,
then padded with black areas, and transformed back. This technique 
avoids the forming of ``big pixel'' effects, but it appears to have less
quality. In reality, the ``quality'' is spread over a larger area.
 }\label{ZoomTrick}
\end{figure}

In the figure \ref{tigre01}, the trick is explained.
This is one of the techniques that are used for enlarging
an image without making it look to ``pixelated''. 
It works precisely by the inverse relation between large and small
scales in Fourier Transforms. In discrete Fourier Transforms (DFT),
this derives from an important propierty: the Fourier Transform of
a discrete set of numbers has the same number of elements in the image
(is a one-to-one function).
Take the transform as a function of a discrete index, and it can be stated that
both the function and its inverse have the same period.
Interpreting this
periodicity as a toroidal domain, it defines the size of the domain.

Of course, if this trick is well defined it should work both ways.
I present that  on the figure \ref{ZoomTrick02}. We can
expand the tyger image with black padding, and obtain an enlarged image
of the frequencies domain. Again, the information on the
small scale is the tyger, wich spreads over the whole domain, and
the ``blackness'' gets evenly distributed. 

\begin{figure}[H]
  \centering
  \begin{subfigure}[b]{0.48\textwidth}
    \centering
          \includegraphics[scale=0.40]{Zero_Padding_4x4.png}
                \caption{A tyger with padding}
                \label{tigreynegro02}
  \end{subfigure}%
\begin{subfigure}[b]{0.48\textwidth}
    \centering
          \includegraphics[scale=0.40]{Trans_Pad_4x4.png}
                \caption{The Fourier Transform of the image to the left.}
                \label{ftigreynegro02}
  \end{subfigure}
\caption{ The trick works both ways, we can use it also for enlarging
the picture of the frecuency domain.
 }\label{ZoomTrick02}
\end{figure}

Of course, as the trick works independently on each axis, we can
make it only in one direction, shown in the figures \ref{ZoomTrick03}. 
I am very slowly driving the point home here,
so I begg you patience.
I want you to think of the tyger as a Wigner function now, or
as a discrete sampling of it, and its transform as a chord function.
  

\begin{figure}[h]
  \centering
  \begin{subfigure}[b]{0.48\textwidth}
    \centering
          \includegraphics[scale=0.40]{Zero_Padding_3x1.png}
                \caption{A tyger with vertical padding, centered.}
                \label{tigrevert01}
  \end{subfigure}%
\begin{subfigure}[b]{0.48\textwidth}
    \centering
          \includegraphics[scale=0.40]{Trans_Pad_3x1.png}
                \caption{Its Fourier transform.}
                \label{ftigrevert01}
  \end{subfigure}
\\
  \begin{subfigure}[b]{0.48\textwidth}
    \centering
          \includegraphics[scale=0.40]{Zero_Padding_1x4.png}
                \caption{A tyger with horizontal padding.}
                \label{ftigrehor01}
  \end{subfigure}%
\begin{subfigure}[b]{0.48\textwidth}
    \centering
          \includegraphics[scale=0.40]{Trans_Pad_1x4.png}
                \caption{And its Fourier Transform.}
                \label{ftigrehor01}
  \end{subfigure} 
\caption{ The trick works independently on each axis and
disregards of centering or origin. The photo of the tyger
could be any real valued function on the plane, including,
conveniently, a Wigner quasi distribution.
 }\label{ZoomTrick03}
\end{figure}


\section{ There are Fourier Transforms and then, there are Fourier Transforms}

\subsection{A Symplectic, continous, exact, bidimensional Fourier Transform}

The transform that goes from Wigner representation to Weyl representation
of the density matrix is this one:

\begin{equation}
\chi(\xifase, \mufase)=\int d\xfase d \yfase
W(\xfase,\yfase)\exp \big(-\ihb (\xfase\wedge \xifase+\yfase\wedge \mufase) \big).
\end{equation}

We are interested on the section $\xifase=0$, so this reduces to tracing over
the $\xfase$ and performing the symplectic Fourier transform on the 
$\yfase$ degree of freedom. Let us call this reduced function simply $W(y)$:

\begin{align}
W(\yfase) &=\int d\xfase
W(\xfase,\yfase)\exp \big(-\ihb \xfase\wedge \xifase \big) \rvert_{\xifase=0} \\ 
&=\int d\xfase 
W(\xfase,\yfase).
\end{align}

Then we need only a continuous Fourier transform in the plane.
It turns out, we are not really doing that --- 
each one of us is doing a different,
but related, Fourier Transform.

\begin{figure}[h]
\centering
\includegraphics[width=0.8\textwidth]{CentrosTraceOverx01.png}
\caption{The $W(\yfase)$ distribution made with Dirac Deltas.}
\label{karelwignerfun}
\end{figure}

My Wigner function is a large set of Dirac Deltas on the Wigner phase space, which 
once traced over reduce to a set of Dirac Deltas on the $\yfase$ plane distributed
as in the figure \ref{karelwignerfun}. Until now I realize that
if I am not interested on any other section of the
chord function, the $\xfase$ variable is completely irrelevant.
 The expression is:
\begin{equation}
W(\yfase)=\frac{1}{N}\sum_{\yfase_k} \delta(\yfase-\yfase_k)
\end{equation}
The nice thing is that I know the analitical expresion of the Fourier transform of
this summatory of Dirac Deltas, so I have only to calculate this sum,
point for point, in a sensible region of the $\mufase$ chord space:
\begin{equation}
\chi(\xifase=0, \mufase)=\frac{1}{N}\sum_{\yfase_k} \exp (-\ihb (\yfase_k\wedge\mufase))
\end{equation}
The discretization of this Fourier Transform  has to do with the minimum commun
denominator of al the $\yfase_k$ up to machine precission. We can take it as trully
continuous.

\subsection{An image processing Fourier Transform}

Now, Fabricio has worked a problem much more similar to the discrete Fourier Transform, and to
the tyger image example above. 
Let me show this in small steps.
The Discrete Fourier Transform for a finite secuence of real or complex numbers
can be defined as:
\begin{equation}
X_j=\sum_{k=0}^{N-1}x_k \exp(-i 2\pi \frac{j k}{N})
\end{equation}
From the definition it can be seen that, in order for the
relation to be invertible, we need at least as many $X_j$ as
$x_k$ and that the $X_j$ repeat themselves after a period of $N$. That means,
both sequences have the same length. 
If we make the same transform with a bidimensional array of numbers, the same
reasoning applies:
\begin{equation}
X_{\alpha, \beta}=\sum_{j,k=0}^{N-1}x_{j,k} \exp(-i 2\pi \frac{j \alpha + k\beta}{N})
\end{equation}
Notice that in this very simple scheme, the array has no ``length'' in the continuous
dimensional sense, only a length defined by number of elements. 

Of course they could be differently sized in each dimension, but for simplicity let
us keep them squared. Now, remember our tyger. Every member of the grid is a pixel,
and its grey hue is the value of the function. The $\alpha, \beta$ domain
is a frecuency domain, the ``whiteness'' of the original photograph
being the value of $x_{\alpha, \beta}$. 
This kind of black and white photos of animals have
similar looking Fourier Transforms, and most of the large structure
which makes one recognize the animal is 
\emph{close to the origin in the frequency domain}. I have put the
origin of the transformed space in the center of the picture by a phase
shift, as we customarly
use. 

If we augment the array by a group of zeros away from the origin, as done the figures
\ref{ZoomTrick}, we change the $N$ number by a larger one without changing
the information in the picture. When we perform the inverse transform, this will appear
in the divisor of the argument of the trigonometric functions, effectively making the
frequencies smaller, and thus, spreading the image over a larger scale. Stated as an
afforism: we have enlarged the period of the image, its large structure encoded
in the origin of the $\alpha, \beta$ space, still untouched. 

\subsection{A simple discretization in the plane}

In the case that we have not an array of deltas, but a numeric representation
of a function whose Fourier Transform we cannot obtain by analytical means, our
only hope is to perform a discretization and sums. We can do this in a lot
of clever ways, but the idea is always essentialy the same. Here I will
translate Fabricio's email and expand it.

We take a rectangle in the phase space which contains the
effective support of the function. Outside this, the functions has allmost zero
values. Sometimes a good sense of what is not relevant is needed to define
the right rectangle. The rectangle has to be larger than this effective support.
We should also make an educated guess about the effective support of the
transform, and define a rectangle there. Then we make a grid $\mathbb{G}$, 
let us say,
$N$ by $N$, and perform the Riemann sum 
\begin{equation}
\chi(\mu)=\sum_{y_k \in \mathbb{G}} W(y_k) \exp(-\ihb  (y_k \mu)) \Delta y
\end{equation}
For clarity I use one dimensional notation.
In a na\"ive sampling, we have
\begin{equation}
y_k=y_{min}+\frac{(y_{max} -y_{min})k}{N} \text{ and } 
\Delta y = \frac{(y_{max} -y_{min})}{N}.
\end{equation}
We make the pertinent substitution in the the expression for the 
Riemann sum,
\begin{equation}
\chi(\mu)=\sum_{0\leq k \le N} W(y_k) 
\exp \Big(-\ihb \mu
\big(\frac{y_{max}-y_{min} k }{N} +y_{min}\big) \Big) 
\Delta y
\end{equation}

That last expression begins to look clearly similar to the discrete Fourier Transform,
with some annoying factors and an extra phase trown in. How differently
would it behave, specially regarding data density?

This is exactly what Fabricio addresed in his response. I copy verbatim:
\begin{verbatim}
Agora defino dois retangulos: dominio e imagem. 
O retangulo dominio deve ser maior que o 
"suporte efetivo de f(x,y)"(ou seja para valores (x,y) 
perto da borda ou fora do retangulo dominio 
f(x,y) é praticamente zero!). O retangulo 
imagem deve incluir o suporte efetivo de F(X,Y).
Agora o negocio é que para fazer o calculo 
numerico você precisa definir uma grade 
bidimensional tanto no retangulo dominio como no retangulo imagem!
O segredo da questão é que a granulosidade da grade no espaço 
imagem esta controlada pelo tamanho 
do retangulo dominio isso quer dizer que se aumento 
o retangulo dominio diminuo a granulosidade no retangulo imagem. 
Para a transformada inversa é ou contrario!
Conclusao você: se você brinca com o 
tamanho dos retangulos (existem limites para isto!)
em essencia o que muda é a nitidez com que você 
vai ver a imagem das respeitivas transformadas 
nao o tamanho!
\end{verbatim}



\end{document}
