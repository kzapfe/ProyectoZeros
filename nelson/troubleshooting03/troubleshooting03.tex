\documentclass[a4paper,12pt]{article}

\usepackage[utf8]{inputenc}
\usepackage{amsmath,amssymb}
\usepackage{graphicx}
%\usepackage{subfigure}
%\usepackage[spanish]{babel}
\usepackage{bm}
\usepackage{caption}
\usepackage{subcaption}

\usepackage[cm]{fullpage}
\usepackage[light]{antpolt}
\usepackage[T1]{fontenc}

\usepackage{float}

%Este paquete le pone una barrera a los floats
% al final de cada seccion
\usepackage[section]{placeins}


\bibliographystyle{alpha}

\newcommand{\ihb}{\frac{i}{\hbar}}
\newcommand{\xfase}{\mathbf{x}}
\newcommand{\yfase}{\mathbf{y}}
\newcommand{\qfase}{\mathbf{q}}
\newcommand{\pfase}{\mathbf{p}}
\newcommand{\xifase}{ {\boldsymbol{\xi}} }
\newcommand{\mufase}{ {\boldsymbol{\mu}} }
\newcommand{\Ifase}{\mathbf{I}}
\newcommand{\Pfase}{\mathbf{P}}
\newcommand{\Scat}{\mathbf{S}}
\newcommand{\Jsimp}{\mathbf{J}}
\newcommand{\Dom}{\mathbb{D}}
\newcommand{\Var}{\mathbb{M}}
\newcommand{\bra}[1]{\langle #1|}
\newcommand{\ket}[1]{|#1\rangle}
\newcommand{\braket}[2]{\langle #1|#2\rangle}
\newcommand{\Prom}[2]{\langle #1\rangle_{#2}}


\DeclareMathOperator*{\cod}{cod}
\DeclareMathOperator*{\traza}{traza}


\title{ Troubleshooting: Third Part.}
\author{Toscano et Zapfe}


\begin{document}

\maketitle

\begin{abstract}
Continuing from ``troubleshooting01'', and after some days working along
with Fabricio (and skipping sending you the other document) here we continue
in the struggle to make the zeros of the Chord Function match.
\end{abstract}


\section{Restating the Fourier convention}


The symplectic product follows the Ozorio de Almeida order:

\begin{equation}
\xfase\wedge\xifase=p_x\xi_q - q_x \xi_p.
\end{equation}

This is also the convention that F. Toscano uses, so we agree on this
matter. 
The chord function is defined as the expected value of the
translation operator \emph{for the additive inverse} of
the chord in question. We settle that affair using the correct relation
between the chord and centre functions:

\begin{align}\label{conventionsign}
\chi(\xifase) &:=\langle \hat{T}_{-\xifase} \rangle \\
&=\int d\xfase
W(\xfase)\exp \big(-\ihb (\xfase\wedge \xifase) \big) \\
&=\int d\xfase
W(\xfase)\exp \big(+\ihb (\xifase\wedge \xfase) \big).
\end{align}

Fabricio, notice please that I usually  write the symplectic product as
($\textbf{centre}\wedge\textbf{chord}) $, so I had the same sign convention
as you. The numeric shows it,  we did not disagree on that.   
You can check this against the definition shown in
Eduardo and Alfredo J.Phys A. from 2010, vol 43, page 205302,
or even better, your own article in Phys. Lett. A 379, 2011, page 21.
I must stress that even if the convention is wrong, the only problem
would be a minus sign on the odd part of the chord
function. As long as none of us comes up with a third sign, it should not
be a problem.

\section{The $\xifase=0$ section}

Our research has tested the \emph{weak ergodic hypothesis} in 
a section of the chord space for the four dimensional problem
posed by the Nelson Potential.

Again, we rewrite the potential to make sure we are all
using the same terminology:


\begin{equation}\label{classicalhamiltonian}
H(\xfase,\yfase)=(p_x^2+p_y^2)/2+\omega_1 q_x^2/2+
\omega_2(q_y-q_x^2/2)^2.
\end{equation}

The classical parameters are as follows:
\begin{itemize}
\item  $\omega_1=0.1$ 
\item $\omega_2=1$,
\item and the value for the energy, $E=0.81384007$.
\end{itemize}

Plank's constant shall have the value  $\hbar=0.05$.

Our chord space consists on the four-dimensional chord
$(\xifase, \mufase)$, the first being the result of
transforming the centre symbols with respect to the
first degree of freedom (denoted, as in the classical
Hamiltonian function, by a $\xfase$), and the second
to the $\yfase$ degree of freedom, see the equation
\ref{classicalhamiltonian}.

To obtain our particular bi-dimensional section of the 
chord function from the centre function one can simply
trace over the $\xfase$ degree of freedom and 
perform the SFT only in the other d.o.f. This can be easily
seen from the definition:

\begin{equation}
\chi(\xifase,\mufase)=\int d\xfase d\yfase
W(\xfase, \yfase)\exp \big(-\ihb (\xfase\wedge \xifase+\yfase\wedge\mufase) \big).
\end{equation}

Notice how in the \emph{weak ergodic hypothesis} we also
identify centres with the classical phase space, as 
in the much stronger \emph{Berry Ergodic hypothesis}. This,
of course, is based on the original intent of the
so called phase space representation of Quantum Mechanics.

Taking into account that we are only interested
in the $\xifase=0$ section:

\begin{align}
\chi(0,\mufase) & = \int d\xfase d\yfase
W(\xfase,\yfase )
\exp \big(-\ihb (\xfase\wedge 0+\yfase\wedge\mufase) \big) \\
 & =  \int d \yfase \Biggl( \int W(\xfase, \yfase) d \xfase \Biggr) 
\exp  \bigl(-\ihb (\yfase\wedge\mufase) \bigr).
\end{align}

This last expression is the one which is effectively implemented in the
code that I wrote, although more simplified due to the nature
of the \emph{``pointillist''} Wigner Function approximation.

\subsection{A section from the section}

If one compares in the same scale the nodal lines due to 
Fabricio and mine calculations in chord space,
the most notable difference is the scale of
separation in the vertical directions, which corresponds
to the $\mu_p$ coordinate, as shown in the next figure,
\ref{comparacionzeros01}. 

\begin{figure}[H]
  \centering
  \begin{subfigure}[b]{0.45\textwidth}
    \centering
          \includegraphics[width=\textwidth]{FabricioWeyl02-ZerosContour.pdf}
                \caption{Fabricio's Nodal Lines.}
                \label{FabZeros}
  \end{subfigure}%
\begin{subfigure}[b]{0.45\textwidth}
    \centering
          \includegraphics[width=\textwidth]{KarelWeyl-ZerosContour.pdf}
                \caption{Karel's Nodal Lines.}
                \label{KarelZeros}
  \end{subfigure}%
\caption{The Nodal Lines for both calculations. }\label{comparacionzeros01}
\end{figure}

Their is other big difference between these two images, and is
the number of blind spots. The quantum version shows only two
blind spots closer to the origin, while the pointillist
shows six. We will attack each of the discrepancies at a time.

First, the question of the scale in the $\mu_p$ axis.
It can be noted that the chord function can be obtained over
a section along this axis by further tracing over
the $p_y$ variable. We notice that in our pointillist
approximation, this is equivalently as to protecting
the cloud of points on the $q_y$ axis. 


\begin{equation}
\chi(\xi_q=0, \xi_p=0, \mu_q=0, \mu_p)  = 
  =  \int   \Biggl( \int W(q_x, p_x, q_y, q_p) dq_x dq_y dp_y \Biggr) 
\exp  \bigl(+\ihb (y_q \mu_p ) \bigr) d y_p.
\end{equation}

The fact that our Wigner function looks like a cloud
of points in the four dimensional centre space is due 
to its functional form:

\begin{equation}
W(\xfase, \yfase)= 
=\frac{1}{N}\sum_{(\xfase^*, \yfase^*)_k} \delta (\xfase - \xfase_k^*)
\delta (\yfase - \yfase_k^*)
\end{equation}

Here it is understood that the stared quantities are
\emph{very close} to the surface of energy in question,
this means

\begin{equation}
H(\xfase^*, \yfase^*)  \approx E_{294} = 0.8384007.
\end{equation}


The inner integrated, in our case turns simply to be a projection:
\begin{align}
\int W(q_x, p_x, q_y, q_p) dq_x dq_y dp_y & =
\int 
\frac{1}{N}\sum_{(\xfase^*, \yfase^*)_k}
\delta (\xfase - \xfase_k^*)\delta (\yfase - \yfase_k^*) 
dq_x dq_y dp_y \\
& = \frac{1}{N}\sum_{q_{y,k}^*} 
 \delta(q_y - q_{y,k}^*) 
\end{align}

And the restricted SFT of this expression would simply be:

\begin{equation}\label{chirestricta}
\chi(\xi_q=0, \xi_p=0, \mu_q=0, \mu_p)  = 
\frac{1}{N}\sum_{q_{y,k}^*} \exp(+\ihb q_{y,k}^* \mu_p)
\end{equation}

So, after Fabricio's suggestion, we will first check what
happens in a very coarse approximation to this expression.
We shall think that the $q_{y,k}^*$ are evenly spaced, that is:

\begin{equation}\label{evenlyspaced}
q_{y,k}=q_{y,min}+(q_{y,max}-q_{y,min})k/N
\end{equation}

which makes the expression \ref{chirestricta} a geometric
truncated series, which yields as an approximation:

\begin{align}\label{aprox01}
\tilde{\chi}(\mu_p)  & = 
\frac{1}{N}\sum_{k=0}^{N-1} \exp(+\ihb \mu_p \cdot (q_{y,min}+(q_{y,max}-q_{y,min})k/N)) \\
& = 
\frac{ \exp(+\ihb \mu_p  q_{y,min})}{N}\sum_{k=0}^{N-1} 
\exp\big(+\ihb \mu_p \cdot (q_{y,max}-q_{y,min})/N)\big)^k   \\
& = \frac{\exp(+\ihb \mu_p  q_{y,min})}{N} 
\frac{ 1 - \exp\big(-\ihb \mu_p (q_{y,max}- q_{y,min})\big)}
{ 1 - \exp\big(+\ihb \mu_p (q_{y,max}- q_{y,min})/N\big)} \\
& =\frac{\exp\big(+\ihb \mu_p q_{y,min}\big) - \exp\big(+\ihb \mu_p q_{y,max}\big)}
{ N - N \exp\big(+\ihb \mu_p (q_{y,max}- q_{y,min})/N\big)} \\
\end{align}

Now, $N$ is really large, so we take the approximation for
the denominator
\begin{equation}
\exp\big(+\ihb \mu_p (q_{y,max}- q_{y,min})/N\big) \approx
1 + \ihb  (q_{y,max}- q_{y,min})/N,
\end{equation}
resulting in 
\begin{equation}
\tilde{\chi}(\mu_p)  = -i\hbar
\frac{\exp\big(+\ihb \mu_p q_{y,max}\big) - \exp\big(+\ihb \mu_p q_{y,min}\big)}
{\mu_p ( (q_{y,max}- q_{y,min})}.
\end{equation}

This is already a palatable expression that we can use to plot and compare with 
the two numerical approach's. 

The real and imaginary part are as follows:
\begin{align}
\Re \tilde{\chi} (\mu_p)  &= \hbar
\frac{\sin(\mu_p q_{y,max} /\hbar)-\sin(\mu_p q_{y,min} /\hbar)}
{\mu_p ( (q_{y,max}- q_{y,min})} \\
\Im \tilde{\chi} (\mu_p)  &= -\hbar
\frac{\cos(\mu_p q_{y,max} /\hbar)-\cos(\mu_p q_{y,min} /\hbar)}
{\mu_p ( (q_{y,max}- q_{y,min})} 
\end{align}

To get some values of the $q_{y,min}$ and $q_{y,max}$ we only look
at the pointillist approximation in the position space
 or in the $\yfase$ d.o.f. projection, shown in figure \ref{poincentros}.
$q_{y,min}=-1$ and $q_{y,max}=8$ seem to be good enough values.

\begin{figure}[h]
 \centering
  \begin{subfigure}[b]{0.45\textwidth}
    \centering
          \includegraphics[width=0.9\textwidth]{NubeDeltasQProjection01.png}
                \caption{$q$-space projection.}
                \label{NubeQ}
  \end{subfigure}%
\begin{subfigure}[b]{0.45\textwidth}
    \centering
          \includegraphics[width=0.9\textwidth]{NubeDeltasYProjection01.png}
                \caption{$\yfase$-projection.}
                \label{NubeY}
  \end{subfigure}%
\caption{The projections for the pointillist Wigner function approximation. }\label{poincentros}\label{nubes}
\end{figure}

The resulting comparison is plotted in the figure \ref{comparafab}. We shown the
$\tilde{ \chi }$ approximation versus Fabricio's Quantum Chord function, both
over this special one-dimensional section. The
crossing points with the zero-axis look terribly wrong for my data,
while the crude approximation seems to be very adequate. Fabricio's data
shows very few points at this magnification, but that doesn't seem to be a
problem. An interesting discrepancy between his data and the crude approximation
is the mean spacing between peaks. Even though the firs zero is more or
less adequately placed, the approximate function has a higher 
``frequency'', probably due to over simplification. Both seem to have at
this scale a Bessel-like behaviour.   


\begin{figure}[h]
  \centering
  \begin{subfigure}[b]{0.45\textwidth}
    \centering  
    \includegraphics[width=0.99\textwidth]{ComparaAprroxyFab01.pdf}
      \caption{Real Part}
      \label{comparafabreal}
  \end{subfigure}%
  \begin{subfigure}[b]{0.45\textwidth}
    \centering  
    \includegraphics[width=0.99\textwidth]{ComparaAprroxyFab02.pdf}
      \caption{Imaginary Part}
      \label{comparafabimag}
  \end{subfigure}%
  \caption{Comparison for the first zeros along the $\mu_p$ axis. 
}\label{comparafab}
\end{figure}     


We can also make the whole procedure along the $\mu_q$ axis, obtaining
very similar expressions:

\begin{align}
\Re \tilde{\chi} (\mu_q)  &= \hbar
\frac{\sin(\mu_1 p_{y,max} /\hbar)-\sin(\mu_q p_{y,min} /\hbar)}
{\mu_p ( (q_{y,max}- q_{y,min})} \\
\Im \tilde{\chi} (\mu_q)  &= -\hbar
\frac{\cos(\mu_q p_{y,min} /\hbar)-\cos(\mu_1 p_{y,max} /\hbar)}
{\mu_p ( (p_{y,max}- p_{y,min})} 
\end{align}

But now , due to the symmetry in the Hamiltonian function, we have
$p_{y,min}=-p_{y,min}$, which results in


\begin{align}\label{CrudeApprox2MuQ}
\Re \tilde{\chi} (\mu_q)  &= \hbar
\frac{\sin(\mu_1 p_{y,max} /\hbar)}
{\mu_p q_{y,max}} \\
\Im \tilde{\chi} (\mu_q)  &= 0
\end{align}

The last expression is comfortably reassuring. We make the plot of the real
part against Fabricio's and Karel's numerical calculations in the figure
\ref{comparatodas02}. The plots have been normalized so that the 
crude $\tilde{\chi}$ approximation has a maximum of $1$. That means
that I have to re-scale Fabricio's data by a factor $1/\pi^2$
(somewhat curios) and my data by $\sqrt(\pi*\hbar)$. This makes all
plots ``normal''.

\begin{figure}[h]
  \centering     
  \begin{subfigure}[b]{0.45\textwidth}
    \centering
    \includegraphics[width=0.80\textwidth]{ComparaMuQAxis.pdf}
   \end{subfigure}%
  \begin{subfigure}[b]{0.45\textwidth}
    \centering
    \includegraphics[width=0.80\textwidth]{ComparaMuQAxisZoomout.pdf}
    \end{subfigure}%
  \caption{Comparison of the real part along the $\mu_q$ axis. Notice that
the three functions plotted behave similarly, but Fabricio's and Karel's cross
the zero axis much closer to each other than to the crude approximation. At right we
see a wider interval, where Fabricio's and Karel's two or three peaks follow pace
while the approximation falls behind faster. }
  \label{comparatodas02}
\end{figure}

The Crude Approximation was obtained with the hypothesis of
a more or less homogeneous distribution for the $q_y$ centres of the
pointillist Wigner function. This is very far from the case, as can
be seen from the histogram presented in the figure \ref{histoqy},
or, for those with an intuitive grasps, from the projections themselves in
the figure \ref{nubes}. The approximate distribution
inferred from the histogram has absolutely nothing in common
with an homogeneous distribution.  We should now compare this with the distribution
obtained from Fabricio's data, and try why his distribution is more similar to
the Crude Approximation.

From Fabricio's Data, which shows a relative large area of
the chord function in the $\xifase=0$ plane, I can perform an inverse
SFT and obtain the traced over Wigner function on the $\yfase$ plane. This is
one of the things that keeps me perplexed: the inverse transform from Fabricio's data
doesn't look well centred, although, being a traced over quasi-distribution and not
a section, Fabricio's expressed little concern for its looks. This result is shown
in figure \ref{WignerFromFab}. I have expressed my astonishment on the displacement
of the peaks from the origin, an apparent translation corresponding
to a very precise value: $q_y \rightarrow q_y+4$. If we trace the $p_y$ coordinate
out of this last data, we obtain a numerical approximation to the probability
distribution function on the $q_y$ axis, using the properties of the
Wigner function.

\begin{equation}
\int W(p,q) d p = \bra{q}\hat{\rho}\ket{q} = \| \psi (q) \|^2
\end{equation}

From this property we obtain the numerical probability density 
seen in the figure \ref{FabProb}. This compared with the figure \ref{histoqy}
is quite annoying. It shows that the distribution is concentrated around this
mysterious $q_y=4$, instead of the origin. We see that both of them have a peak around
one unit right from their centre of distribution, and both decay abruptly after it.
Also, they show almost no significant values in places more left than 
one unit of their centre. This of course is expected from the shape of the
Energy Surface, hinted by the cloud of deltas in figure \ref{nubes}, or
by classically checking the value of the permissible values for this coordinate
directly from the Hamiltonian ($q_{y,min}=\sqrt{E}=0.9021$). 

A comparison which gives trust to the reasoning this far is
the distribution of the $\yfase_p$ values. The histogram obtained
from the pointillists Wigner function appears very homogeneous, so
that the Crude Approximation in eq. \ref{CrudeApprox2MuQ} is adequate
and hits very near in the three calculations.

\begin{figure}[h]
  \centering     
    \includegraphics[width=0.80\textwidth]{DistribucionNormalizada_y_q_ExactNCentros.pdf}
  \caption{An empirical histogram showing the appearance of various $q_y$ values
for the pointillist Wigner function. It can be seen that it roughly corresponds
to the sizes of the $y$-horns of the energy surface.}
  \label{histoqy}
\end{figure}


\begin{figure}[h]
  \centering     
    \includegraphics[width=0.80\textwidth]{WignerDirectFromFabChords01.png}
  \caption{The Traced over Wigner function obtained by performing the inverse
SFT on Fabricio's quantum data for his $\xifase=0$ section of the chord function.
The data is not normalized, and the colour scale is only chosen on contrast grounds.
The black curve is the zero contour level line. }
  \label{WignerFromFab}
\end{figure}


\begin{figure}[h]
  \centering     
    \includegraphics[width=0.80\textwidth]{PsiTrazedOverXSquaredFromFabricio01.pdf}
    \caption{After performing an inverse SFT over the Quantum section for the chord function,
and tracing over the $p_y$ coordinate, we obtain this approximate p.d.f for the $q_y$
value.}
    \label{FabProb}
\end{figure}


We shall investigate what can probably lead to these very different
distributions for the $q_y$ value. Fabricio obtained the cords function
directly from the density matrix for a pure state in the position representation:

\begin{equation}\label{fabchidef}
\chi(\xifase)=\int \bra{q+\xi_q/2}\hat{\rho}\ket{q-\xi_q/2}\exp (+\ihb q \xi_p) d q
\end{equation}

I also obtained the ``Berry'' cumulants for the pointillist centre function, which
must be very close to the classical expected values for the $p^m q^k$  type of
functions. In principle, from the expression \ref{fabchidef}, it would be possible
to obtain the same information deriving the chord function. Numerically, this would
need very high precision around the origin, which we do not have. We can see
that the density of data for the quantum result is to low in figure \ref{comparafab},
although we may try to make a wild guess for the second derivatives
on the real part. The third derivative would be more difficult to obtain.
From the following polynomial expansion we can produce a good conjecture
about the correct placement of the first nodal lines.

\begin{equation} \label{expansionpot}.
\chi(\xifase)=1-\ihb \Prom{\xfase}{E}\wedge\xifase
-\frac{1}{2 \hbar^2}\Prom{\xfase\wedge\xifase}{E}^2
+\frac{i}{6 \hbar^3 }\Prom{\xfase\wedge\xifase}{E}^3+\ldots
\end{equation} 

I want to point out a strange feature on the two plots of figure \ref{comparafab}.
The pointillist approximation shows a not so smooth kink before the $\pm 0.05$ values
of $\mu_p$. It is after this change of curvature that my chord function
begins to spread out, making the zeros so far away from the origin. Let us check again
the zeros obtained directly from the cumulants in the expansion \ref{expansionpot}.
We should notice that the scale in which things seem to fail is the scale of $\hbar$,
although this could be a misleading clue. We have to see how the
3rd order approximation fits the different data then.


\begin{figure}[h]
  \centering     
  \begin{subfigure}[b]{0.45\textwidth}
    \centering
    \includegraphics[width=0.80\textwidth]{Compara3ordenRealMuP.pdf}
    \caption{Real part along $\mu_p$}
   \end{subfigure}%
  \begin{subfigure}[b]{0.45\textwidth}
    \centering
    \includegraphics[width=0.80\textwidth]{Compara3ordenImagMuP.pdf}
    \caption{Imaginary part along $\mu_p$}
  \end{subfigure}\\
   \begin{subfigure}[b]{0.45\textwidth}
    \centering
    \includegraphics[width=0.80\textwidth]{Compara3ordenRealMuQ.pdf}
    \caption{Real part along $\mu_q$}
    \label{real3ordenmuq}
   \end{subfigure}%
  \begin{subfigure}[b]{0.45\textwidth}
    \centering
    \includegraphics[width=0.80\textwidth]{Compara3ordenImagMuQ.pdf}
    \caption{Imaginary part along $\mu_q$}
    \label{imag3ordenmuq}
  \end{subfigure}
  \caption{Comparison of the pointillist, quantum and truncated to third order
chord functions along the axis of the $\xifase=0$ plane. }
  \label{compara3orden}
\end{figure}

In the figure \ref{compara3orden} we can appreciate that close to the origin the
three functions really look similar, but the third order approximations lacks
this change of curvature and therefore, crosses the abscissa much nearer
to the place that Fabricio's Data does. In the sub-figure \ref{real3ordenmuq},
where in principle there is nothing wrong with the scales, we observe that
this parabolic approximation falls to close to the origin, as it does not
change direction of curvature. This makes me think that such an approximation
should be taken with a grain of salt at those scales. Nevertheless, it is very
comforting to see how it fits the curvature and tilt near the origin. 
In the sub-figure \ref{imag3ordenmuq} Fabricio's Data shows a slight deviation
from the theoretical value $0$, due that his data doesn't fall exactly over the
$\mu_p=0$ axis, but is a slightly off ($\mu_p =0.002802$).  

I think both tests point to errors in the $\Prom{q_y}{E}$ value for the state.
The mistake seems to lie here, although I still do not understand why the inverse
transform of Fabricio's data seems to be so far from the classical 
average values for $\Prom{q_y}{E}$, and why this displacement is
a so neat whole number. This displacement seems to \emph{neat}. Observing Fabricio's
first figure in the document \verb|chord-fun-niv294.pdf| produces
the impression that the peak of the distribution for $q_y$ must be just slightly above
the origin (attention to the two almost black peaks). 

The relevant piece of FORTRAN code that I am using for obtaining the inverse transform 
(which produces a traced over $\xfase$ Wigner function!) is as follows. Notice that
I zealously calculate a ``imaginary part'' for the Wigner function, which
gives me a very adequate result of $0.0000$:

\begin{verbatim}
  do k=1,muestreo
     do l=1,muestreo

        centerp=-maxcp+float(l)/float(muestreo)*2.0*maxcp
        centerq=mincq+float(k)/float(muestreo)*(maxcq-mincq)
        wignerreal=0.d0
        wignerimag=0.d0

!! Esta parte corresponde al muestreo en el programa de ida
        do i=1,N1
           do j=1,N2
           !! Tienes que tener cuidado de usar la convencion opuesta!
              !! a la que tienes en el programa NelsonPlano y derivados
              !! Recuerda: x^xhi= x.p*xhi.q-x.q*chi.p
              !! asi las cosas, aqui la transformada es 
              !!Wigner(center)=(Xi(chord)*exp(center^chord/hb)

              argument=(centerp*chordq(i,j)-centerq*chordp(i,j))/hb 
                           
              wignerreal=wignerreal  &
                   +cos(argument)*weylreal(i,j) &
                   -sin(argument)*weylimag(i,j)
              
               wignerimag=wignerimag  &
                   +cos(argument)*weylimag(i,j) &
                   +sin(argument)*weylreal(i,j)

              
           end do
           
1        end do
        

        !! Obvio que en la integral de Riemann va la long del intervalo
        wignerreal=wignerreal*intervaloq*intervalop/pi
        wignerimag=wignerimag*intervaloq*intervalop/pi
 
        write(30, '(4F20.12)') centerq,centerp,wignerreal,wignerimag
        
     end do
     
     print*, "Vamos en el Bloque ", k
     write(30, *) ""

  end do

\end{verbatim}
  
\end{document}

