Deutsch   Nachricht übersetzen
Deaktivieren für: Portugiesisch
Caro Karel,

Continuando minha resposta, minha preferência pelo Mondrian, se deve ao fato
das bordas dos retângulos do Rothko serem mal definidas, o que não é o nosso caso.

1 Restating conventions:

Realmente, vai facilitar termos a mesma notação. Note que há um erro estranho
de latex no (2). O chato é que ainda temos problema com o produto exterior,
ou simplético. Adoto sempre a regra do Arnold, que é o sinal oposto à sua em (7),
isto é, vale a regra da mão esquerda: Se os dedos menores da mão esquerda
formam o 1o vetor e o polegar for o 2o, então o produto será positivo.
O interessante é que no meu Report defino as transformações entre Wigner e corda
com o sinal oposto ao seu, portanto, depois de duas trocas de sinal, temos o mesmo resultado!    

2 STF of a rectangular area

(15) Latex?

(19) agree, but not with Fig 1.

Notice that (19), or more simply (18),  leads to 4 plane waves:

\pm exp[ i {(a1 \pm l1) xi1 +(a2 \pm l2) xi2} ] (Here l is lower case for L)

The real nodal lines are the zeroes of the superposition of the 4 terms:

cos {(a1 \pm l1) xi1 +(a2 \pm l2) xi2} 

and the imaginary nodal lines are the zeroes of the superposition of the terms:

sin {(a1 \pm l1) xi1 +(a2 \pm l2) xi2} 

These are nonlinear functions of the pair if variables xi1 and xi2, so that I hardly
expect anything like Fig 1 (what is J_0 in the caption?)

Notice that the blind spots, ie the intersection of the nodal lines, are insensitive to
the overall phase that depends on the centre a. So one obtains the blind spots directly by putting a=0. However, this simplification is no longer possible for
superpositions of rectangles.

Vamos ver se agora agilizamos nossa interação, para tentar concluir algo.

Um abraço,  Alfredo

