\documentclass[a4paper,12pt]{article}

\usepackage[utf8]{inputenc}
\usepackage{amsmath,amssymb}
\usepackage{graphicx}
%\usepackage{subfigure}
%\usepackage[spanish]{babel}
\usepackage{bm}
\usepackage{caption}
\usepackage{subcaption}

\usepackage[cm]{fullpage}
\usepackage[light]{antpolt}
\usepackage[T1]{fontenc}

\usepackage{float}

%Este paquete le pone una barrera a los floats
% al final de cada seccion
\usepackage[section]{placeins}


\bibliographystyle{alpha}

\newcommand{\ihb}{\frac{i}{\hbar}}
\newcommand{\xfase}{\mathbf{x}}
\newcommand{\yfase}{\mathbf{y}}
\newcommand{\qfase}{\mathbf{q}}
\newcommand{\pfase}{\mathbf{p}}
\newcommand{\xifase}{ {\boldsymbol{\xi}} }
\newcommand{\mufase}{ {\boldsymbol{\mu}} }
\newcommand{\Ifase}{\mathbf{I}}
\newcommand{\Pfase}{\mathbf{P}}
\newcommand{\Scat}{\mathbf{S}}
\newcommand{\Jsimp}{\mathbf{J}}
\newcommand{\Dom}{\mathbb{D}}
\newcommand{\Var}{\mathbb{M}}
\newcommand{\bra}[1]{\langle #1|}
\newcommand{\ket}[1]{|#1\rangle}
\newcommand{\braket}[2]{\langle #1|#2\rangle}
\newcommand{\Prom}[2]{\langle #1\rangle_{#2}}
\newcommand{\dif}{\, \mathrm{d}}
\newcommand{\indicator}[1]{\mathbf{1}_{ \{   #1 \} } } 

\DeclareMathOperator*{\cod}{cod}
\DeclareMathOperator*{\traza}{traza}


\title{ Boring but necesary Apendix B: Formula for the
Caustic in q-space for the Nelson Potential.}
\author{Karel}


\begin{document}

\maketitle

\begin{abstract}
We obtain exact analytical calculation for the Caustic 
Curves in q-space given the Nelson Potential.
\end{abstract}


\section{The Nelson Potential}

The Nelson Potential has in ``natural'' coordinates ($m=1$) the form:

\begin{equation}\label{classicalhamiltonian}
H(\xfase)=(p_{x1}^2+p_{x2}^2)/2+\omega_1 q_{x1}^2/2+
\omega_2(q_{x2}-q_{x1}^2/2)^2.
\end{equation}

The classical parameters are as follows:
\begin{itemize}
\item  $\omega_1=0.1$ 
\item $\omega_2=1$,
\item and the value for the energy, $E$.
\end{itemize}


In the following calculations I shall write $x:=x_1, and y:=x_2$
to avoid unnecesary clutering of indexes. I shall also use
$\omega=\omega_1$ and omite writting explicitly  $\omega_2=1$.
Now, the caustic is when the kinetk energy is zero, that means:

\begin{align}
E=V(q_x, q_y) & =  \frac{\omega q_x^2}{2}+(q_y - \frac{q_x^2}{2})^2 \\
0 &=  \frac{\omega q_x^2}{2}+(q_y - \frac{q_x^2}{2})^2  -E \\
0 &=  \frac{\omega q_x^2}{2}+ q_y^2 - q_yq_x^2 +\frac{q_x^4}{4} -E
\end{align}

This gives us the following expression for the two curves, after
aplication of ``La Chicharronera''.
\begin{equation}
y_{\pm}(x)=\frac{1}{2}\sqrt{4E-2\omega x^2}.
\end{equation}
The domain is the set between the two anhilation points of the 
discriminant, $x\in [-\sqrt{2E/\omega},+\sqrt{2E/\omega} ]$

\end{document}