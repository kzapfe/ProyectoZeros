\documentclass[a4paper,12pt]{article}

\usepackage[utf8]{inputenc}
\usepackage{amsmath,amssymb}
\usepackage{graphicx}
%\usepackage{subfigure}
%\usepackage[spanish]{babel}
\usepackage{bm}
\usepackage{caption}
\usepackage{subcaption}

\usepackage[cm]{fullpage}
\usepackage[light]{antpolt}
\usepackage[T1]{fontenc}

\usepackage{float}

%Este paquete le pone una barrera a los floats
% al final de cada seccion
\usepackage[section]{placeins}


\bibliographystyle{alpha}

\newcommand{\ihb}{\frac{i}{\hbar}}
\newcommand{\xfase}{\mathbf{x}}
\newcommand{\yfase}{\mathbf{y}}
\newcommand{\qfase}{\mathbf{q}}
\newcommand{\pfase}{\mathbf{p}}
\newcommand{\xifase}{ {\boldsymbol{\xi}} }
\newcommand{\mufase}{ {\boldsymbol{\mu}} }
\newcommand{\Ifase}{\mathbf{I}}
\newcommand{\Pfase}{\mathbf{P}}
\newcommand{\Scat}{\mathbf{S}}
\newcommand{\Jsimp}{\mathbf{J}}
\newcommand{\Dom}{\mathbb{D}}
\newcommand{\Var}{\mathbb{M}}
\newcommand{\bra}[1]{\langle #1|}
\newcommand{\ket}[1]{|#1\rangle}
\newcommand{\braket}[2]{\langle #1|#2\rangle}
\newcommand{\Prom}[2]{\langle #1\rangle_{#2}}
\newcommand{\dif}{\, \mathrm{d}}
\newcommand{\indicator}[1]{\mathbf{1}_{ \{   #1 \} } } 

\DeclareMathOperator*{\cod}{cod}
\DeclareMathOperator*{\traza}{traza}


\title{ Boring but necesary Apendix A: detailed obtantion of the
Volume inside the Energy Shell for the Nelson Potential}
\author{Karel}


\begin{document}

\maketitle

\begin{abstract}
We need the detailed calculus for the volume and area of
the Energy Shell in the Nelson Potential. We put everything here.
\end{abstract}


\section{The Nelson Potential}

The Nelson Potential has in ``natural'' coordinates ($m=1$) the form:

\begin{equation}\label{classicalhamiltonian}
H(\xfase)=(p_{x1}^2+p_{x2}^2)/2+\omega_1 q_{x1}^2/2+
\omega_2(q_{x2}-q_{x1}^2/2)^2.
\end{equation}

The classical parameters are as follows:
\begin{itemize}
\item  $\omega_1=0.1$ 
\item $\omega_2=1$,
\item and the value for the energy, $E=0.81384007$.
\end{itemize}

Plank's constant shall have the value  $\hbar=0.05$.

The Volume is the value of the indicator function for
the set where $H(\xfase)<E$. 

In the following calculations I shall write $x:=x_1, and y:=x_2$
to avoid unnecesary clutering of indexes. I shall also use
$\omega=\omega_1$ and omite writting explicitly  $\omega_2=1$.

\section{Analytic Calculation}

\begin{equation}
\int \dif\xfase \indicator{H(\xfase)<E}
\end{equation}

Detailing, we can simply interpret the indicator as limits of integration.
In the momentum part, the maximum avaible momentum is given by:

\begin{equation}
|p_{max}|=\sqrt{2 E- 2 V(q_x, q_y)}
\end{equation}

So the integral over the momentum variables can be done
in polar coordinates, giving us simply:

\begin{equation}
V= 2 \pi \iint \dif q_x \dif q_y E-V(q_x, q_y)
\end{equation}

Now the limits of integration are the turning points of the
trayectories, that is, the caustic of the manifold proyected
in configuration space. It is simpler to put this in
terms of $q_x^2$:

\begin {align }
E=V(q_x, q_y) & = &\frac{\omega q_x^2}{2}+(q_y - \frac{q_x^2}{2})^2 \\
0 &= &\frac{\omega q_x^2}{2}+(q_y - \frac{q_x^2}{2})^2 \\
0 &= &\frac{\omega q_x^2}{2}+ q_y^2 - q_yq_x^2 +`\frac{q_x^4}{4}
\end{align}

\begin{equation}
ads
\end{equation}


\end{document}
