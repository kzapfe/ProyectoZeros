\documentclass[a4paper,12pt]{article}

\usepackage[utf8]{inputenc}
\usepackage{amsmath,amssymb}
\usepackage{graphicx}
%\usepackage{subfigure}
\usepackage[spanish]{babel}
\usepackage{bm}
\usepackage{caption}
\usepackage{subcaption}

\usepackage[cm]{fullpage}
\usepackage[light]{antpolt}
\usepackage[T1]{fontenc}

\usepackage{float}

%Este paquete le pone una barrera a los floats
% al final de cada seccion
\usepackage[section]{placeins}


\bibliographystyle{alpha}

\newcommand{\ihb}{\frac{i}{\hbar}}
\newcommand{\xfase}{\mathbf{x}}
\newcommand{\yfase}{\mathbf{y}}
\newcommand{\qfase}{\mathbf{q}}
\newcommand{\pfase}{\mathbf{p}}
\newcommand{\xifase}{ {\boldsymbol{\xi}} }
\newcommand{\mufase}{ {\boldsymbol{\mu}} }
\newcommand{\Ifase}{\mathbf{I}}
\newcommand{\Pfase}{\mathbf{P}}
\newcommand{\Scat}{\mathbf{S}}
\newcommand{\Jsimp}{\mathbf{J}}
\newcommand{\Dom}{\mathbb{D}}
\newcommand{\Var}{\mathbb{M}}
\newcommand{\bra}[1]{\langle #1|}
\newcommand{\ket}[1]{|#1\rangle}
\newcommand{\braket}[2]{\langle #1|#2\rangle}
\newcommand{\Prom}[2]{\langle #1\rangle_{#2}}


\DeclareMathOperator*{\cod}{cod}
\DeclareMathOperator*{\traza}{traza}


\title{ Weyl Functions for the Nelson Potential:\\ some puzzling comparisions}
\author{\\CBPF}


\begin{document}

\maketitle

\begin{abstract}
Continuing the ``Puzzled01'' document, here I present the various 
related calculations to the Weyl, Wigner, and Husimi function.
An unexpected phase appeared on F. Toscano data.
\end{abstract}




\section{Be sure we all speak the same language}

First things first, is to make sure we are speaking of the same
problem before we compare results. I shall use the notations that
I have been using in previous reports. The name of the coordinates
in ``normal space'' as well as ``center space'' coincide, and 
are given in the following expression for the Hamiltonian function:


\begin{equation}
H(\xfase,\yfase)=(p_x^2+p_y^2)/2+\omega_1 q_x^2/2+
\omega_2(q_y-q_x^2/2)^2.
\end{equation}

The classical parameters are as follows:
\begin{itemize}
\item  $\omega_1=0.1$ 
\item $\omega_2=1$,
\item and the value for the energy, $E=0.81384007$
\end{itemize}

The quantum quantity is the value of Planck's Constant, $\hbar$ is
$0.05$ in Fabricio's and Karel's work (I check that at least once per day).
Our symplectic Fourier transform (SFT) follows this
convention:

\begin{equation}
F(\xifase, \mufase)=\int d\xfase d \yfase
f(\xfase,\yfase)\exp \big(-\ihb (\xfase\wedge \xifase+\yfase\wedge \mufase) \big).
\end{equation}

That is, $\xifase$ is the chord for the center $\xfase$ as
$\mufase$ is for the $\yfase$.
Is good to notice that the convention for the symplectic product is
the one of Alfredo:

\begin{equation}
\xfase\wedge\xifase=p_x\xi_q - q_x \xi_p.
\end{equation}

In this convention, the SFT is a ``forward'' transform
in the $p$ variables leading to chord $q$ variables, and a 
``backward'' transform of the $q$ variables to chord $p$.
This will be important later. 


\section{Inverting Fabricio's Result}


To compare the point-like Wigner function approximation with
Fabricio's exact Wigner function I performed a partial 
inverse transform for his data. This gives me the 
``traced over $\xfase$'' Wigner function, as follows.
Let $f$ denote a function in center space and $F$ its SFT in chord
space. We have the data corresponding to the $\xifase=0$ section.

\begin{multline}
F(0,\mu):=\int d\xfase d\yfase f(\xfase,\yfase) 
\exp \big( -\ihb (\xfase\wedge 0 + \yfase \wedge \mufase) \big) \\
=\int \int f(\xfase,\yfase)d\xfase \exp \big( -\ihb  \yfase \wedge \mufase \big) 
d \yfase
\end{multline}

After performing the partial  inverse transform, we obtain the projection
of the Wigner quasi distribution on the $\yfase$ plane:

\begin{multline}\label{halfinverse}
\tilde{f} (\yfase):=\int d\xfase  f(\xfase,\yfase) \\
=\int d\mufase  F(0,\mufase)  \exp \big(+\ihb  \yfase \wedge \mufase \big) 
d \yfase.
\end{multline}

Here comes the first thing a bit off. In Fabricio's document, ``chord-fun-niv294.pdf'',
the Quantum Wigner function appears with the peak at the origin of coordinates, 
but performing the above transform (eq. \ref{halfinverse} 
on his data I obtain the following picture (fig. \ref{KfromF01-Wigner}):

\begin{figure}[H]
\begin{center}
  \includegraphics[width=0.8\textwidth]
                  {WignerfromWeyl02.png} %
\caption{The traced over Wigner function obtained
after performing a partial backwards SFT over Fabricio's chord function data. }
\label{KfromF01-Wigner}
\end{center}
\end{figure}

As can be seen, the peak of the function is displaced exactly four 
units on the $y_q$ axis. Fabricio's data is ordered in an array indexed
by integers, so, in order to check if I am using the right interpretation
for his indexes, I plot the chord function and see if it coincides with
the images on his document. The clearest way to show this is to reproduce
the plots for the chord function from Fabricio's data and compare them
to his plots. 

\begin{figure}[H]
  \centering
  \begin{subfigure}[b]{0.45\textwidth}
    \centering
          \includegraphics[width=\textwidth]{CompFabrZerosReal.png}
                \caption{Real part}
                \label{realFchord}
  \end{subfigure}%
\begin{subfigure}[b]{0.45\textwidth}
    \centering
          \includegraphics[width=\textwidth]{CompFabrZerosImag.png}
                \caption{Imaginary part}
                \label{imagFchord}
  \end{subfigure}%
\caption{The real and Imaginary parts of Fabricio's data for the chord
function. I present the same scale that he uses, but the color range is
shortened for higher contrast. Notice that his normalization is different.  
The gray line is the nodal set. }\label{Fabchords}
\end{figure}

The plots shown in the above figure, \ref{Fabchords}, coincide very well
with those on the document ``chord-fun-niv294.pdf''. Notice that his
normalization is different to mine. His real part covers the interval
$(-6.9195,9.6144)$ and the imaginary the interval
$(-6.5239,6.5239)$. So I discard the problem of the interpretation of
his data. Then again, the backward transformation of that data
produces a Wigner quasi distribution on the $\yfase$ plane which is
4 units away from the origin. Remembering the property of
the Fourier Transform for translations;

\begin{equation}
FT[f(x-x_c)](k)= FT[f(x)](k) \exp(-ix_ck).
\end{equation}

Then, I suppose that Fabricio's data has a phase that goes as
$-i4\mu_p/\hbar$ which I do not have. I take that into account in the
backwards transformation, and I get the following picture for
the traced-over-$\xfase$ Wigner function: 


henceforth called the Dirac-Wigner semi-classical function, or DWf, to the exact
Wigner function for the 294th level of the Nelson Potential. We have
in our hands the data for its Weyl or chord transform in the section
$\xifase=0$. We remember that we use the symbol $\xifase$ for the
chord transformation of the center $\xfase$ and the symbol $\mufase$
for the one corresponding to $\yfase$. 

The backward transformation is only a partial transform, as we use
the data restricted to the section in question. 


\begin{figure}[H]
\begin{center}
  \includegraphics[width=0.8\textwidth]
                  {WignerfromWeyl04.png} %
\caption{The traced over Wigner function obtained
after performing a partial SFT over Fabricio's chord function data.
An extra phase of $+i4\mu_p/\hbar$ has been added to the kernel of the transform. }
\label{KfromF02-Wigner}
\end{center}
\end{figure}

Lo and behold! The function appears centered on the origin, as it should. 
So we have this difference in phase in our data, an $4\mu_p/\hbar$.

\section{Taking an extra phase}

The next step is to see if that extra phase would account for the
differences in our chord functions. I will put this extra phase
in the calculation of the chord function for the point-Dirac Wigner
function and compare this with Fabricio's chord function. I
already pointed out that there appeared to be a factor of $8$ units
missing in the $\mu_p$ axis. It seems to be related to this.


\begin{figure}[H]
  \centering
  \begin{subfigure}[b]{0.45\textwidth}
    \centering
          \includegraphics[width=\textwidth]{FabricioWeyl02-ZerosContour.pdf}
                \caption{Fabricio}
                \label{WeylFabNodal}
  \end{subfigure}%
\begin{subfigure}[b]{0.45\textwidth}
    \centering
          \includegraphics[width=\textwidth]{KWeylTroubleshot02-ZerosContour.pdf}
                \caption{Karel}
                \label{WeylKarNodal}
  \end{subfigure}%
\caption{The Nodal lines for the chord function according to Fabricio and myself. 
This time, an extra phase of $-i4\mu_p/\hbar$ has been added to my SFT. }
\label{FabKarelchords}
\end{figure}

Examining the previous figures and comparing it to the figures of the 
last report, we can see that at least the scales for the nodal lines fit now. 
Now I shall present other test. To see how different the nodal lines are, I will
compare real and imaginary parts separately. 



\begin{figure}[H]
  \centering
  \begin{subfigure}[b]{0.45\textwidth}
    \centering
          \includegraphics[width=\textwidth]{ComparaZerosReal.pdf}
          \caption{Real part}
                \label{WeylRealNodal}
  \end{subfigure}%
\begin{subfigure}[b]{0.45\textwidth}
    \centering
          \includegraphics[width=\textwidth]{ComparaZerosImag.pdf}
                \caption{Imaginary part}
                \label{WeylImagNodal}
  \end{subfigure}%
\caption{The Nodal lines for the chord function according to Fabricio and myself, separated in
real and imaginary parts. Black lines are Fabricio's, red ones are mine. }
\label{FabKarelNodal}
\end{figure}

As we can see in the figure \ref{FabKarelNodal}, starting from the origin, the first
three nodal lines are not so different. In the fourth nodal lines my nodal set presents 
some weird features, as circular nodal lines break otherwise more or less leveled sets.
Still, the difference is big enough to cause different sets of blind spots. In the figure
\ref{FabKarelchords} can be seen that the first nodal lines produce six blind spots in Fabricio's
figure, but only 2 in mine, only because the imaginary oval encircles the real one. I will recall a 
figure from the previous report, where I show the nodal lines obtained solely by the calculation
of the first three cumulants of the Wigner dotted distribution. 



\begin{figure}[H]
\begin{center}
 \begin{subfigure}[b]{0.45\textwidth}
    \centering
    \includegraphics[width=0.8\textwidth]
                    {ExactoN_0821_WeylAprox3grado-0-0-ZerosContour.pdf} %
                    \caption{Truncated up to third order }
                    \label{CumulantsK}
 \end{subfigure}%
 \begin{subfigure}[b]{0.45\textwidth}
   \centering
   \includegraphics[width=\textwidth]{ExactNCompleto.pdf}
   \caption{Complete SFT of the dotted Dirac Wigner distribution.}
   \label{CompleteK}
 \end{subfigure}%
\caption{Karel's version of the chord nodal lines without the extra phase.}
\label{Karelsinfase}
\end{center}
\end{figure}

Conspicuously enough, that last figure also shows six blind spots, although the
two nodal lines almost overlap. 

With this not very firm observations, I will make my guess about what is wrong now.
 There is something in Fabricio's SFT that produces an extra phase of the
form $ - i 4 \mu_p/hbar$.  
To test my conjecture I will proceed as follows.  First, I will obtain a very high
resolution chord function from Fabricio's Wigner function, in order to locate very
precisely the first nodal lines in a ball of radius $0.2$ around the origin.
That number is a bit less than $\sqrt{\hbar}$, which is the scale that interests
us.  Then I will produce again this plot  but adding the mysterious phase to 
the SFT. I will compare both plots to my own using also the SFT with and without
that extra phase. I will concentrate my attention on a square of side $0.4$
centered on the origin, and I will make a very dense SFT of Fabricio's data
for the Wigner Quantum function for the energy level 294. In the next figure 
I compare what happens when I make the transform with and without the
$-4 \mu_p /\hbar $. 



\begin{figure}[H]
  \centering
  \begin{subfigure}[b]{0.45\textwidth}
    \centering
          \includegraphics[width=\textwidth]{HighResWignerFabsinfase-ZerosContour.pdf}
          \caption{Without the phase.}
                \label{WeylFabsinfase}
  \end{subfigure}%
\begin{subfigure}[b]{0.45\textwidth}
    \centering
          \includegraphics[width=\textwidth]{ZoomHighResWeylFabconfase4-ZerosContour.pdf}
                \caption{With the phase}
                \label{WeylFabconfase}
  \end{subfigure}%
\caption{The Nodal lines for the chord function obtained after a high density SFT of Fabricio's
Wigner function.}
\label{FabNodalfases}
\end{figure}


If we compare the figure \ref{WeylFabconfase} with the figures \ref{WeylFabNodal}
and \ref{WeylKarNodal}, we can perceive a set of convenient similarities.  For
starters, the closest lines to the origin share the same scale, so this extra phase
was present in Fabricio's chord function for sure, confirming the translation
of the backwards obtained Wigner function seen in the figure \ref{KfromF01-Wigner}. 
Second, and most important: it shows that the topological differences
in the first blind spots is due to operating the zero-searching algorithm over
insufficient points. Fabricio's data has  a grid of 1000 by 1000 in a square of
size 4.4 by 5.6. I have used his Wigner function and obtained a 600 by 600 grid
in a square of side 0.4 for the chord function, and the zero-searching algorithm
produces better located nodal lines. A relieving characteristics is that both the figure
\ref{WeylFabconfase} and \ref{KfromF01-Wigner} have only two blind spots for the nodal
lines closest to the origin, and topologically, they are equivalently located. 
Another thing which is still puzzling is that, although figures \ref{WeylFabsinfase}
and \ref{CompleteK} share the same scale, they are topologically very different,
specially in imaginary nodal manifold. Equally intriguing is the fact that 
the figure \ref{CumulantsK}, even when it corresponds to a truncated 
expansion without the extra phase, resembles more
the figures with the extra phase.

I would conclude then that Fabricio's routine for the SFT is producing this
extra phase, and for that I would like to ask him to produce a finer version of
his chord function in the $0.4$ sided square around the origin, a 600 by 600
grid data would be fine. If things as I say, the nodal lines should
be like that in \ref{WeylFabconfase}. 



\section{The huwi and huwe functions}

As an extra test for correctness of the SFT, we will check the huwi functions
of both sets of data and their SFT, which shall be labeled huwe (Husimi-Weyl) 
functions. Due to the convolution property for  Fourier Transforms, the
huwe function is just the chord function multiplied by a coherent Gaussian function.
This has to preserve the position of the nodal surfaces, if everything in the code is
right.

As the Husimi transform of a Wigner function can be seen as a loss-less coarsening
of the Wigner Function, I will use uppercase letters for its domain. The Husimi 
transform of a two degree of freedom Wigner function $W(\xfase,\yfase)$
shall be thus:


\begin{equation}
Q(X,Y)=\int W(\xfase,\yfase) \exp \big( (-\|\xfase-X\|^2-\|\yfase-Y\|^2)/\hbar \big)
d\xfase d\yfase
\end{equation}

A partially traced over Husimi function would be obtained by integrating over the
$\xfase$ variables and then making the Husimi transform over the $\yfase$ symplectic
pair. This is not equal to the Husimi function in the $X=0$ plane. 
 As the data that I have from Fabricio corresponds to the traced over
Wigner function, that shall be the function presented in the next figure, followed by 
that done using my dotted Wigner function.


\begin{figure}[H]
\begin{center}
  \includegraphics[width=0.8\textwidth]
                  {HusimiFabricio01.png} %
\caption{The traced over $\xfase$ Husimi function from Fabricio's data,
subtracting the extra phase. We see that it is positive and
bounded.}
\label{HusimiFab}
\end{center}
\end{figure}


\begin{figure}[H]
\begin{center}
  \includegraphics[width=0.8\textwidth]
                  {HusimiKarel02.png} %
\caption{The traced over $\xfase$ Husimi function from Karel's data,
subtracting the extra phase. We see that it is quite similar to the previous
figure, although lacks some of the valleys. Also the scale is lower, as if 
the function 
had been smeared to fill the lower regions.}
\label{HusimiKar}
\end{center}
\end{figure}


Now we shall investigate its SFT, which we could call (a bit erroneously) a
huwe function:

\begin{equation}
\begin{split}
huwe(\mufase)=\int W(\xfase,\yfase) d\xfase \exp\big(-\|\yfase-Y\|^2/\hbar \big)
d \yfase \exp (-\ihb(Y\wedge \mufase)) d Y \\
=\chi(0,\mufase)\exp(-\|\mufase\|^2/\hbar)
\end{split}
\end{equation}

The next pair of double figures show the huwe functions for Fabricio's data and my own, 
that is, for the exact quantum Wigner function of the 294 level and for the
doted Dirac Wigner semi-classical approximation for an energy shell around the same energy. 
The huwe function has been calculated with and without the extra phase of $-4i\mu_p /\hbar$.



\begin{figure}[H]
  \centering
  \begin{subfigure}[b]{0.45\textwidth}
    \centering
          \includegraphics[width=\textwidth]{ZoomHighReshuweFabsinfase-ZerosContour.pdf}
          \caption{Fabricio}
                \label{huweFabsinfase}
  \end{subfigure}%
\begin{subfigure}[b]{0.45\textwidth}
    \centering
          \includegraphics[width=\textwidth]{ZoomHighReshuweKarelsinfase-ZerosContour.pdf}
                \caption{Karel}
                \label{huweKarelsinfase}
  \end{subfigure}%
\caption{The Nodal lines for the huwe function, discarding the $-i4\mu_p/hbar$ phase that
seems to appear in Fabricio's SFT. Note the usual augment in the scale of the $\mu_p$ direction.  }
\label{huwesinfase}
\end{figure}



\begin{figure}[H]
  \centering
  \begin{subfigure}[b]{0.45\textwidth}
    \centering
          \includegraphics[width=\textwidth]{ZoomHighReshuweFabconfase-ZerosContour.pdf}
          \caption{Fabricio}
                \label{huweFabconfase}
  \end{subfigure}%
\begin{subfigure}[b]{0.45\textwidth}
    \centering
          \includegraphics[width=\textwidth]{ZoomHighReshuweKarelconfase-ZerosContour.pdf}
                \caption{Karel}
                \label{huweKarelconfase}
  \end{subfigure}%
\caption{The Nodal lines for the huwe function, using the $-i4\mu_p/hbar$ phase that
seems to appear in Fabricio's SFT. The similitude with previous figures is clear, \ref{WeylFabconfase} and
\ref{WeylKarNodal}.  The smoothing out due to the Gaussian factor could be responsible for mistakes
in the zero searching algorithm, which produces the closed lines in the second figure. A finer SFT could
solve that.}
\label{huweconfase}
\end{figure}11


\end{document}


