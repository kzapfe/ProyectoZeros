\documentclass[a4paper,12pt]{article}

\usepackage[utf8]{inputenc}
\usepackage{amsmath,amssymb}
\usepackage{graphicx}
\usepackage{subfigure}
%\usepackage[spanish]{babel}
\usepackage{bm}

\usepackage[cm]{fullpage}
\usepackage[light]{antpolt}
\usepackage[T1]{fontenc}

%Este paquete le pone una barrera a los floats
% al final de cada seccion
\usepackage[section]{placeins}


\bibliographystyle{alpha}

\newcommand{\ihb}{\frac{i}{\hbar}}
\newcommand{\xfase}{\mathbf{x}}
\newcommand{\yfase}{\mathbf{y}}
\newcommand{\qfase}{\mathbf{q}}
\newcommand{\pfase}{\mathbf{p}}
\newcommand{\xifase}{ {\boldsymbol{\xi}} }
\newcommand{\mufase}{ {\boldsymbol{\mu}} }
\newcommand{\Ifase}{\mathbf{I}}
\newcommand{\Pfase}{\mathbf{P}}
\newcommand{\Scat}{\mathbf{S}}
\newcommand{\Jsimp}{\mathbf{J}}
\newcommand{\Dom}{\mathbb{D}}
\newcommand{\Var}{\mathbb{M}}
\newcommand{\bra}[1]{\langle #1|}
\newcommand{\ket}[1]{|#1\rangle}
\newcommand{\braket}[2]{\langle #1|#2\rangle}
\newcommand{\Prom}[2]{\langle #1\rangle_{#2}}


\DeclareMathOperator*{\cod}{cod}
\DeclareMathOperator*{\traza}{traza}


\title{ Weyl Functions for the Nelson Potential:\\ some puzzling comparitions}
\author{\\CBPF}


\begin{document}

\maketitle

\begin{abstract}
Here I present various different computations for the Weyl or Chord 
function of the Nelson Potential Problem. The values of Energy
and Planck's constant are equal to  those used on
Toscanos et all \cite{Toscano01}, detailed as follows: 
$\hbar=0.05, E=0.813840071, n=294.$
All of these parameters are in dimensionless reduced units, 
the $n$ parameter is the quantum level of the energy value.
\end{abstract}

\section{Classical Chaos}

The Nelson potential problem is part of a a family of problems which
have applications in nuclear physics. The Hamiltonian function 
for the problem is:

\begin{equation}
H(\xfase,\yfase)=(p_x^2+p_y^2)/2+\omega_1 q_x^2/2+
\omega_2(q_y-q_x^2/2)^2
\end{equation}

We choose $\omega_1=0.1$ and $\omega_2=1$, and explore the volume
between the Energy surfaces corresponding to $E=0.81384007$
and $E+\hbar\omega_1$. 
We will denote this volume by $V_E$. The system
appears to be ``visually''  ergodic, meaning that any non-hyperbolic
subsets are of inappreciable size in relation to the whole
energy hyper-surface in phase space for the energies chosen. 

 
\section{Quantum Ergodic Hypothesis}

The Quantum Ergodic Hypothesis  intuitively states that a representation
of a quantum state for a chaotic system would cover more or less uniformly
the Energy Surface. The most extreme version of this statement is
that the Wigner Function of the state would be a Dirac Delta distribution
 exactly over the Energy Surface \cite{BerryRIR}. However we know
that this could not be true on the general case \cite{Ozorio98}. We know
that a lot of chaotic systems show ``scars'' in the position
or momenta
representation of  eigenstates \cite{haake}.
 



\section{The coarse grained function and the quantum exact result.}

As we have discussed, we will sample points around the energy 
surface in the phase space
of the problem.  Those would be treated as 
the centre of a set of Dirac Deltas, the sum of them being
the Wigner Function of the State. In previous calculations
I was sampling to many points for this, on hopes of obtaining
a better resolution. This was a mistake, as I shall explain.

The Dirac Centres would represent a coarse grained
approximation to a semiclassical Wigner function. The scale which
imposes the finesse of the grain is $\hbar$, the
only quantum parameter that sets  a scale here.
On average, every centre should occupy a volume of 
$\hbar^2$, which would set a sort of resolution
for micro-scars of a highly chaotic state, in which
the Wigner function is not strictly positive everywhere. 
After a little tedious integration, for the parameters
chosen, the volume inside the energy shell is 
around $29.2326$ dimensionless units, in terms of our
value for Planck's Constant this will give space
for at most $N=11 696$ centres.  


In the first figure (fig. \ref{FabQuant}) 
I show the nodal lines in the 
discussed section of the phase space obtained from
Fabricio's exact quantum calculation. I concentrate
my attention in  a vicinity of radius $0.2$ around 
the origin. The curve described by the nodal line
over the origin is an effect of the 
plotting routine, it should appear as a closed
oval shape with a straight line across it. 

\begin{figure}
\begin{center}
  \includegraphics[width=0.8\textwidth]
                  {zoom-quantum-niv294-ZerosContour.pdf} %
\caption{Real and Imaginary nodal lines from
Fabricio's Quantum data. The level of the state
is $n=294$, corresponding to 
$E=0.813840071, \hbar=0.05$. }
\label{FabQuant}
\end{center}
\end{figure}

The second figure (fig. \ref{Cumulant}) has been obtained
from the cumulant expansion for the chord function. 
We expand the chord function up to the third order terms;

\begin{equation}\label{expansion}\chi(\xifase,\mufase)\approx 
1-\ihb (\xifase,\mufase) \wedge \Prom{\hat{\xfase},\hat{\yfase}}{V}
-\frac{1}{\hbar^2}
\Prom{((\xifase,\mufase)\wedge(\hat{\xfase},\hat{\yfase}))^2}{V}
+\frac{i}{\hbar^3}
\Prom{((\xifase,\mufase)\wedge(\hat{\xfase},\hat{\yfase}))^3}{V}+
\cdots
\end{equation}

\begin{figure}
\begin{center}
  \includegraphics[width=0.8\textwidth]
                  {ExactoN_0821_WeylAprox3grado-0-0-ZerosContour.pdf} %
\caption{The nodal lines for the cumulant expansion of the chord
function, up to third order terms.}
\label{Cumulant}
\end{center}
\end{figure}
 
Until the second figure, both methods seem to give
reasonable similar results for the ``blind manifold'', the set of
intersections for zeros of the real and imaginary parts of
the chord function. 

Now I present the puzzling figure. If I make the ``point by point''
Fourier Transform of the Dirac Centres, every member of the
sum would contribute with the trigonometric expression:

\begin{equation}\label{contri}
FT[\delta(\xfase-\xfase_c)](\xifase)=\exp(-\ihb \xfase \wedge \xifase)
/\sqrt{\hbar \pi N}
\end{equation}

One would expect that summing over all $N$ such contributions, one would
obtain an even better approximation to the result shown in the
figure \ref{Cumulant}, but it turns out worse. It appears as if the
$\xifase_p$ axis has been expanded by a factor of eight, as seen
on the last figure,\ref{Complete}. 


\begin{figure}
\begin{center}
  \includegraphics[width=0.8\textwidth]
                  {ExactNCompleto.pdf} %
\caption{Real and Imaginary nodal lines for the sum of all
terms of the form of equation \ref{contri}. It is evident that
something is off. }
\label{Complete}
\end{center}
\end{figure}

It must be stated that the second and third figures where
calculated using the same set of points as centres for
the Wigner function, so the cumulant are those which
come from the same whole set. 

\bibliography{../../ziegos}

\end{document}
