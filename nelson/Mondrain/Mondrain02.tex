\documentclass[a4paper,12pt]{article}

\usepackage[utf8]{inputenc}
\usepackage{amsmath,amssymb}
\usepackage{graphicx}

\usepackage{bm}
\usepackage{caption}
\usepackage{subcaption}

\usepackage[cm]{fullpage}
\usepackage[light]{antpolt}
\usepackage[T1]{fontenc}

\usepackage{float}

%Este paquete le pone una barrera a los floats
% al final de cada seccion
\usepackage[section]{placeins}

\bibliographystyle{alpha}

\newcommand{\ihb}{\frac{i}{\hbar}}
\newcommand{\xfase}{\mathbf{x}}
\newcommand{\yfase}{\mathbf{y}}
\newcommand{\qfase}{\mathbf{q}}
\newcommand{\pfase}{\mathbf{p}}
\newcommand{\xifase}{ {\boldsymbol{\xi}} }
\newcommand{\mufase}{ {\boldsymbol{\mu}} }
\newcommand{\Ifase}{\mathbf{I}}
\newcommand{\Pfase}{\mathbf{P}}
\newcommand{\Scat}{\mathbf{S}}
\newcommand{\Jsimp}{\mathbf{J}}
\newcommand{\Dom}{\mathbb{D}}
\newcommand{\Var}{\mathbb{M}}
\newcommand{\bra}[1]{\langle #1|}
\newcommand{\ket}[1]{|#1\rangle}
\newcommand{\braket}[2]{\langle #1|#2\rangle}
\newcommand{\Prom}[2]{\langle #1\rangle_{#2}}


\DeclareMathOperator*{\cod}{cod}
\DeclareMathOperator*{\traza}{traza}
\DeclareMathOperator*{\sinc}{sinc}


\title{ Troubleshooting 5: The Mondrain Wigner Function.}
\author{Ozorio de Almeida et Zapfe}


\begin{document}

\maketitle

\section{Restating the conventions}

In order to be absolutely sure, we restate the conventions one more
time, using Ozorio de Almeida Convention.

Points in centre space shall be written as $\xfase=(p_x, q_x)$,
the order being always ``(momenta, position)''. In the case
of many degrees of freedom we use an extra sub index for
each degree of freedom, in our case the elements of centre
space have the following convention:

\begin{equation}\label{convencionpuntos}
\xfase=(x_1,x_2)=(p_{x1},p_{x2},q_{x1},q_{x2}).
\end{equation}

The Nelson Hamiltonian has, in this convention, the following
appearance (in appropriate rationalised units):

\begin{equation}
H(x_1,x_2)=\frac{p_{x1}^2+p_{x2}^2}{2}
+\frac{\omega_1 q_{x1}}{2} +\omega_2 (q_{x2}+\frac{q_{x1}^2}{2}).
\end{equation}

We fix the constants as follows:

\begin{align}
E & =H(x_1,x_2)=0.81384007 \\
\omega_1 &=0.1 \\
\omega_2 &=1 \\
\hbar &=0.05.
\end{align}

Also we state the convention for the symplectic product:

\begin{equation}
\xifase\wedge\xfase=\xi_p q_x -  \xi_q p_x.
\end{equation}

From now on, I shall always write it in the order
``chord wedge centre'', so it coincides with Fabricio's and Eduardo's
notation. Notice, as an example, the equation (1) in J. Phys. A.
205302, ``Semiclassical Theory for small displacements''. If we
compare equations (1),(2) and (3), and the expantion
in eq. (14) we see that the translation operator has some
confusing propierties. My confusion seems to arrise from the 
representation of the Translation Operator, which has an apparent
change of sign. Following the previously cited work, or 
in ``Gaussian Representation of Quantum States''  (Phys. Lett. A 2011),
said operator is written as:
\begin{equation}
\hat{T}_\xifase= \exp \big( \ihb (\xifase\wedge\hat{\xfase}) \big)
\end{equation}
which implies that 
\begin{equation}
\hat{T}_{-\xifase}=\exp \big( -\ihb (\xifase\wedge\hat{\xfase}) \big).
\end{equation}
The controversial sign has an apparent confusing declaration
when we define the chord function as the espected value of the
\emph{inverse} translation operator:
\begin{equation}
\chi(\xifase)=\langle \hat{T}_{-\xifase}\rangle
\end{equation}
and then we read how to obtain this value from the centre function:
\begin{equation}
\chi(\xifase)=\int W(\xfase) \exp (\ihb\xifase\wedge\xfase) d\xfase.
\end{equation}
The problem came (to me) as reading this as a direct insert of
the Translation Operator in the Wigner Quasi Distribution integration\ldots
Of course, that is not the case. To obtain an expected value for
an arbitrary operator using the centre representation, we must first
produce an adecuate form for the operator in this language:
\begin{equation}
T_\xifase(p,q)
=\int d y \bra{q-y/2} \hat{T}_\xifase \ket{q +y/2} \exp(\ihb py)
\end{equation}


Notice the lack of normalising constant. We carry it on the inverse
transform:

\begin{equation}
W(\xfase)=\frac{1}{(2 \pi \hbar)^2}
\int \chi(\xifase) \exp (-\ihb\xifase\wedge\xfase) d\xifase
\end{equation}



\section{The SFT of a rectangular area}

To start, we check all details of the Symplectic Fourier Transform
of the characteristic function over a rectangle. In the first step
we simply perform it over a line in a one dimensional $q$ domain.
The result is a Bessel-Zero-like  function with some constants attached
to it. 

\begin{equation}
f(q)=1 \text { if } q\in [-\alpha, \alpha]\text{; }0 \text{ otherwise}
\end{equation}

\begin{align}
\chi_f(\xi) &= \int f(q) \exp (-\ihb q \xi ) \\
&= \int f(q) \exp (-\ihb q \xi ) \\ 
&= \int_{-\alpha}^{\alpha} \exp (-\ihb q \xi ) \\ 
&= \frac{-\hbar}{i\xi}
[\exp(-\ihb \alpha\xi )- \exp(+\ihb\alpha\xi)]  
\end{align}

The imaginary part cancels itself as usual for symmetric functions,
while real part obtained is:

\begin{equation}
\frac{2 \hbar}{\xi}[\sin(\alpha \xi/\hbar)]=
 2\alpha \sinc \Big( \frac{\xi\alpha}{\hbar} \Big),
\end{equation}
using the Engineers notation for $\sin(x)/x=: \sinc (x)$.

The translation property for Fourier transforms gives us 
the general SFT for any interval on this $q$ domain.
If $h(q)=f(q-q_0)$, $f$ defined as above, then,

\begin{equation}
\chi_h (\xi)=2 \alpha \exp (-\ihb q_o \xi)
\sinc\Big( \frac{\xi\alpha}{\hbar} \Big).
\end{equation}.

This function is not purely real.

In a rectangular domain the characteristic function has independent
functional part on each direction, so we can obtain its general
SFT from the separate directions. If both directions are $q$ directions,
say, $q_1, q_2$ and $\xi_1, \xi_2$ on the ``other''  domain,  then
for a rectangle centred in $(a_1, a_2)$ and of side lengths
$l_1$ and $l_2$, its SFT would be:


\begin{equation}\label{qrect}
\chi_h (\xi)=l_1 l_2 
\exp [-\ihb (a_1 \xi_1 + a_2 \xi_2)]
\sinc \Big( \frac{\xi_1 l_1}{2 \hbar} \Big)
\sinc \Big( \frac{\xi_2 l_2  }{ 2 \hbar} \Big).
\end{equation}.

Notice that the Nodal lines for such a function would form an
irregular tridirectional grid. An example is given in the 
figure \ref{example01}.

\begin{figure}
\centering
\includegraphics[width=0.4\textwidth]{ZerosRotko01.png}
\caption{The nodal lines of the expression \ref{qrect}.
The parameters are as follows: $\hbar=0.05, l_1=8,
l_2=4, \alpha_1=4, \alpha_2=2$. We can decompose
the nodal lines in three sets; namely, when each of
the $J_0$ is zero, which coincides for imaginary and real
parts, or when $\exp(-\ihb (a_1 \xi_1 + a_2 \xi_2))$ is zero, which
gives parallel equidistant lines with different phases for real
and imaginary parts. 
The scale  shown is relevant to our
interest.}\label{example01}
\end{figure}

It is easy to generalise the  expression \ref{qrect} 
\emph{tracking with care the sign conventions}
for a cube parallel to
the axis in both $\qfase$ and $\pfase'$ sub spaces. We put it here
for one degree of freedom:

\begin{equation}\label{total}
\chi_h (\xi)=l_q l_p
\exp [\ihb \mathbf{a}\wedge\xifase]
\sinc \Big( \frac{\xi_p l_q}{2\hbar} \Big)
\sinc \Big( \frac{\xi_q l_p}{2\hbar} \Big).
\end{equation}.


\section{A Mark Rothko Wigner Function}

We will try to approximate the rather unknown centre
function for the Nelson Potential at certain energy with rectangular
pieces. Using firstly the sketch provided by A. Ozorio de Almeida 
some weeks ago, we have a superposition of the
SFT of three rectangles of the $\qfase$ space. A first naive approach appears
as in figure \ref{RectangulosGordos}. In that first approach
we compare the \emph{pointillist} centre function with a very crude
approximation made with rectangular characteristic functions
in centre space. I call this a Rothko centre function. 
This first try was made without regard for almost any information
from the projection of the Energy Surface in $\qfase$ space,
only its general shape. We do not dispose of Fabricio's Quantum Chord
function on this section, only along the $\xi_{p2}$ axis,
so we use it to check only the intersection of the nodal lines with itself.


\begin{figure}[H]
  \centering
  \begin{subfigure}[b]{0.45\textwidth}
    \centering
          \includegraphics[width=\textwidth]{RectangulosGordos01.png}
                \caption{The centre function characteristic areas.}
                \label{FabZeros}
  \end{subfigure}%
\begin{subfigure}[b]{0.45\textwidth}
    \centering
          \includegraphics[width=\textwidth]{RothkoGordobien01.pdf}
                \caption{The chord function nodal lines.}
                \label{KarelZeros}
  \end{subfigure}%
\caption{The first crude try of the Rothko centre function. Completely disregard
for area, density or scale has been used here.
The centre of the  $A$ Rectangle is $(0,0)$, $\alpha_1=4, \alpha_2=1$.
The$B,C$ parameters are equal except for the sign in $x_{q1}$ coordinate.
The centre of $B$ is $(4.5, 3.75)$, and its half-width and half-height are
$(0.25,3.5)$}
\label{RectangulosGordos}
\end{figure}

This approximation clearly is to rough to even qualitatively approach
the nodal lines of the Nelson energy eigenstate in the chord representation. 
A better approach would be to take into account the area covered by the projection
of the energy agencies into the $\qfase$ space. This area we have calculated before
for other purposes, and can be put as the integral between the two solutions of
the implicit function of $q_{x2}(q_{x1})$ given fixed $E$.
The maxima and minima for $q_1, q_2$ shall appear everywhere, so we write them
for reference:

\begin{align}
q_{1,max/min} & =\pm \sqrt{2E/\omega_1} \approx \pm 4.0344\\
q_{2,max} & = E/\omega_1 \approx 8.1384\\ 
q_{2,min} & = -\sqrt{E/\omega_2} \approx -0.9021
\end{align}


\begin{align}
E &=\omega_1 q_{x1}^2/2+\omega_2 (q_{x2}- q_{x1}^2/2)^2\\
\Rightarrow q_{x2\pm} & = \frac{\omega_2 q_{x1}^2\pm \sqrt{4 \omega_2 E -2 \omega_1\omega_2 q_{x_1}^2}} 
{2\omega_2 } \\
A_\qfase & =2 \int_{0}^{q_{x1max}} d q_1  q_{x2+}-q_{x2-} \\
& = 2 \int_{0}^{q_{x1max}} d q_1  \frac{\sqrt{4 \omega_2 E -2 \omega_1\omega_2 q_{x_1}^2}} {\omega_2} \\
&=  4 \sqrt{\frac{ E} {\omega_2}} \int_{0}^{q_{x1max}} d q_1 
\sqrt{1-\frac{\omega_1}{2E} q_1^2 } \\
&= 2\pi  E\sqrt{\frac{2}{\omega_1\omega_2} }
\end{align}

Substituting the values of the constants this gives us a value
for the area in arbitrary units of $A_\qfase\approx 22.8683$. This
seems correct, the bananoid appears to cover a bit more than
$1/3$ of the Area of its delimiting rectangle. 

We shall now put some rectangles in this space
which shall respect the Area of the bananoid. The three rectangles
shall have equal area, and the lower rectangle shall spawn the 
total width of the bananoid, as in the sketch sent by Alfredo. This gives us
the next values for Rothko projection. For the $A$ rectangle we have:

\begin{align}
l_{1A} &= 2 q_{1max}= 2 \sqrt{2 E/ \omega_1} \approx 8.03688 \\
l_{2A} &= 2 q_{1max}= \pi/3 \sqrt{1/ \omega_2 E} \approx 1.1608 \\
(a_{1A} ,a_{2A}) & = (0, 0). 
\end{align}

For the $B$ rectangle the characterisation is as follows:

\begin{align}
l_{2B} &= q_{2max}-l_{2A}/2 \approx 7.5580 \\
l_{1B} &=\frac{A_\qfase} {l_{2B}} \approx 1.0085 \\ 
(a_{1B} , a_{2B}) & \approx  (3.4759, 4.3594). 
\end{align}

The $C$ rectangle is just the mirror reflection of this last one. 

It must be noticed that the characteristic function of these
three rectangles is equally to the characteristic function of a big
rectangle encompassing the other three minus a characteristic function
of the rectangle in the middle. The Fourier Transform of this
function shall be equal to the former. Sadly these functions
do not appear like the chord functions
of Fabricio or mine, not even a little, even after adjusting the
constants and everything to match, they look pretty awful.


\begin{figure}[H]
  \centering
  \begin{subfigure}[b]{0.45\textwidth}
    \centering
          \includegraphics[width=\textwidth]{RothkoConAreasExactas01.png}
                \caption{The centre function characteristic areas.}
                \label{FabZeros}
  \end{subfigure}%
\begin{subfigure}[b]{0.45\textwidth}
    \centering
          \includegraphics[width=\textwidth]{RothkoZerosAreasExactas01.pdf}
                \caption{The chord function nodal lines.}
                \label{KarelZeros}
  \end{subfigure}%
\caption{The second crude try of the Rothko centre function. This time
we respected the area of the projection of the energy surface in $\qfase$ space.
This, sadly, doesn't seem to respect the cumulants of the exact distributions,
so we get very different nodal lines than the expected. Our main interest
lies along the $\xi_{p2}$ axis, so we shall only pay attention
to the nodal lines there. 
}
\label{RectangulosAreas}
\end{figure}

\section{What next}

A lot of work and thinking\ldots


\end{document}
