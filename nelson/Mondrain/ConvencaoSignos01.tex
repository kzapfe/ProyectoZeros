\documentclass[a4paper,12pt]{article}

\usepackage[utf8]{inputenc}
\usepackage{amsmath,amssymb}
\usepackage{graphicx}

\usepackage{bm}
\usepackage{caption}
\usepackage{subcaption}

\usepackage[cm]{fullpage}
\usepackage[light]{antpolt}
\usepackage[T1]{fontenc}

\usepackage{float}

%Este paquete le pone una barrera a los floats
% al final de cada seccion
\usepackage[section]{placeins}

\bibliographystyle{alpha}

\newcommand{\ihb}{\frac{i}{\hbar}}
\newcommand{\xfase}{\mathbf{x}}
\newcommand{\yfase}{\mathbf{y}}
\newcommand{\qfase}{\mathbf{q}}
\newcommand{\pfase}{\mathbf{p}}
\newcommand{\xifase}{ {\boldsymbol{\xi}} }
\newcommand{\mufase}{ {\boldsymbol{\mu}} }
\newcommand{\Ifase}{\mathbf{I}}
\newcommand{\Pfase}{\mathbf{P}}
\newcommand{\Scat}{\mathbf{S}}
\newcommand{\Jsimp}{\mathbf{J}}
\newcommand{\Dom}{\mathbb{D}}
\newcommand{\Var}{\mathbb{M}}
\newcommand{\bra}[1]{\langle #1|}
\newcommand{\ket}[1]{|#1\rangle}
\newcommand{\braket}[2]{\langle #1|#2\rangle}
\newcommand{\Prom}[2]{\langle #1\rangle_{#2}}
\newcommand{\rd}{\, \mathrm{d}}


\DeclareMathOperator*{\cod}{cod}
\DeclareMathOperator*{\traza}{traza}
\DeclareMathOperator*{\sinc}{sinc}


\title{Setting the signs straight. }
\author{Zapfe}


\begin{document}

\maketitle

\section{General things}

I will use the article ``The Weyl Representation in Classical and Quantum
Mechanics'' \cite{Ozorio98} as the guideline for the conventions. I shall cite other
articles by the groups as examples about the sign conventions. 
I must say that this matter is not fundamentally important for the
search of the zeros, as the only effect it produces is a change of
sign in the imaginary part of the Chord Function or an inversion
of the odd part of the centre function. 

The wedge product (symplectic product) follows Arnol'd Convention or left
hand rule:
\begin{equation}\label{wedgeleft}
\xifase\wedge\xfase=\xi_p q_x - \xi_q p_x
\end{equation}
In the last equation we also set the convention for the
sub-indexes and the order of the product shall always be 
``chord wedge centre'', regardless of either operand
being taken as a coordinate or as operator (eq. 1.10 in \cite{Ozorio98}).

The quantum translation operator by a chord (vector) $\xifase$ is 
formally
\begin{equation}\label{transdef}
\hat{T}_\xifase=\exp (\ihb \xifase\wedge\xfase),
\end{equation}
according to the eq. 4.22 of the mentioned document, or in
the first equation of ``Semiclassical Theory for small displacements''
\cite{Zambrano10}, also in the eq. 9 of ``Gaussian representation of 
extended Quantum States'' \cite{Ozorio11}.

We \emph{define} the chord or Weyl function as the expected
value for the \emph{negative or backwards} translation:
\begin{equation}\label{chidef}
\chi(\xifase):=\langle \hat{T}_{-\xifase} \rangle
\end{equation}
There are plenty of interesting ways to get an expected
value from an operator. Semiclassicaly, our most popular
strategy is to use the centre (Wigner) function. We \emph{define}
the Wigner function as follows:
\begin{equation}\label{wigdef}
W(\xfase)=\frac{1}{(2\pi\hbar)^2}
\int \rd \xi_q
\bra{q_x+\xi_q/2} \hat{\rho}\ket{q_x-\xi_q/2} \exp(-\ihb p_x \xi_q).
\end{equation} 
This is the definition found in our canonical
reference \cite{Ozorio98},
and also in \cite{Ozorio11} Here $d$ is the number of 
degrees of freedom. This quasi-distribution in a space
which can be thought of classical phase space
can be used for obtaining expected values just as if where a
probability density function:
\begin{equation}
\langle \hat{A} \rangle =
\int \rd \xfase W(\xfase) A(\xfase)
\end{equation}
The tricky part here is that $A(x)$ is a function
on this space, not a quantum operator. The ``of course'' statement
is that this function should be an adequate representation of
that $\hat{A}$ operator. The recipe for obtaining this
symbol is fairly straightforward:
\begin{equation}\label{opcendef}
A(\xfase)=
\int \rd \xi_q
\bra{q_x+\xi_q/2} \hat{A}\ket{q_x-\xi_q/2} \exp(-\ihb p_x \xi_q).
\end{equation} 
As stated in eq. 5.13. Then we see that the Wigner function
is simply the normalised representation of the density operator.

\section{Now for the chord function}

To calculate the $T_{\xifase}(x)$ symbol for the translation 
operator $\hat{T}_{\xifase}$ we shall proceed carefully step by step.
First thing first, we need to see what happens when we use
the operator on a position eigenstate:
\begin{equation}
\hat{T}_{\xifase}\ket{q_x}=
\exp(\ihb \xi_p\xi_q/2 + \xi_p q_x)\ket{q_x+\xi_q}
\end{equation}
Obtained by using 4.22, 4.17, 4.20 and 4.33 of \cite{Ozorio98}.
This should be equivalent to 
\begin{equation}
\hat{T}_{-\xifase}\ket{q_x}=
\exp(\ihb \xi_p\xi_q/2 - \xi_p q_x)\ket{q_x-\xi_q}.
\end{equation}
So that the integrated in \ref{opcendef} has the following aspect:
\begin{align}
T_{-\xifase}(\xfase)&= 
\int \rd \mu \bra{q+\mu/2}\hat{T}_{-\xifase}\ket{q-\mu/2}
\exp(-\ihb p \mu)\\
&=\exp (\ihb [\xifase_p\xifase_q/2-\xifase_p q_x])
\int \rd \mu \braket{q_x +\mu/2}{q_x -\mu/2-\xifase_q}
\exp  (-\ihb \mu p_x)\\
&=\exp (\ihb [\xifase_p\xifase_q/2-\xifase_p q_x])
\int \rd \mu \delta(q_x-\mu-\xifase_q-q_x-\mu/2) 
\exp  (-\ihb \mu p_x)\\
&=\exp\big[\ihb(\xifase_p\xifase_q/2-\xifase_p q_x)\big]
\exp\big[-\ihb(p_x (-\xifase_q))\big]\\
&=\exp\big[\ihb(-\xifase\wedge\xfase+\xifase_p\xifase_q/2))\big]
\end{align}
And here begins my bewilderment: why does it appear on almost all of the
groups papers the expression 
$\int \rd \xfase W(\xfase) \exp (\ihb \xifase\wedge\xfase)$ ?
I know that the phase $\xifase_p\xifase_q/2$ can be safely be
ignored, but the sign convention seems plainly wrong to me. An
exception can be seen on ``Pure State Correlations: chords in phase
space'' \cite{Ozorio05}, in eq. 32 and 33. Also by applying
the definition for the translation operator in the same paper 
(eq. 15) to a $q$ eigenket, we end with the same result. 

There is also another way to achieve that result, which of course
is following the reasoning in \cite{Ozorio98}, and chequing if
the result of combining equation 4.26 and  5.6 gives what appears there.
It is right, and it is the same: 
$T_\xifase(\xfase)=\exp(\ihb\xifase\wedge\xfase)$. In that paper the
ordering of the product is different, so a minus sign appears, but that poses
absolutely no problem.

So, I think my confusion arose because there are some papers which lack a minus
sign in the symplectic Fourier Transform. Also to notice, there are
factors missing from the expansion of the exponential, but that was already
tackled. 

\bibliography{/home/karel/proyectozeros/firsttries/ziegos}



\end{document}