Estimado Professor:

Desculpa a falta de comunicadão da minha parte. Eu fiquei resolvendo problemas 
da vida cotidiana e administrativos muito tempo agora, mais as cosas vão dando
certo devagarinho. A gente comprou um apartamento com a grana que eu consegui 
economizar no Brasil! Mais ainda tenho que pagar os 2/3 restantes para os
 pais da namorada, e as agencias financiadoras mexicanas sofreram um desconcerto
com o cambio de governo a principio de anho, com o bonitão resultante
de que até agora, eu só recebi um pago mensual a partir de Fevereiro...


Mais eu continuo trabalhando com os centros e cordas e devo dizer, acho que entendi não
a sua última sugestão. A função de centros à Max Rothko /Mondrain da uma função
de cordas horrível (Anexa no correio). A função assim produzida respeita não
os primeiros cumulantes dos operadores (p,q), assim que as líneas nodais
assim obtidas não representam o estado de nosso exemplo. A gente pode
(eu tente) de forçar os retângulos característicos para obter os primeiros
cumulantes de forma correta, mais eu acho que tudo isso é excessivamente
artificial- a gente está obtido os cumulantes da função pontilhista
do centros, e, como eu já mostrei antigamente, truncada até o terceira ordem,
da um resultado muito más perto da função quântica do Fabrício (no primeiro0 
troubleshooting01.pdf). Estou procedendo da seguente forma:

Vou a continuar obtido os cumulantes a partir da função pontilhista.
Eu programei anteriormente (ainda no Rio!) o sistema para obter ate 
o terceiro cumulante e o polinômio à Taylor assim derivado. Acho que deve
existir já um pacote de software livre que faga isso para qualquer
<p^m  q^n>. O chato fica na construção do polinômio. 
 Vou a proceder intentar por aquele lado. Eu também comecei já a
fazer um pacote de c++ que, a partir do artigo do Raul Vallejos, consiga
popular qualquer camada de energia como ``herd-of-cats states'', para 
assim obter alguma função de centros que capture as oscilações perto
dos puntos de viragem (as causticas) da camada de energia. Os
cornos do bananoide são tão grandes que devem ter isas estruturas
alguma influença na função de cordas perto do origem.

Também pensei que, para que a função Rothko-Mondrain de alguma coisa
mas o menos aceptável, tem que ser construída com mais cuidado.

Desculpe meu português- levo já tempo sem trenar e não tive tempo ate agora
de ler os livros que eu trouxe do Rio. Eu goste muito dos ``Habitantes
das Alagoas'' do Graciliano Ramos, mais isso foi já um tempo atras.

Abraço
Karel

