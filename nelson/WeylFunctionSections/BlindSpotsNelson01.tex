\documentclass[a4paper,12pt]{article}

\usepackage[utf8]{inputenc}
\usepackage{amsmath,amssymb}
\usepackage{graphicx}
\usepackage{subfigure}
%\usepackage[spanish]{babel}
\usepackage{bm}

\usepackage[cm]{fullpage}
\usepackage[light]{antpolt}
\usepackage[T1]{fontenc}

\usepackage{float}

\bibliographystyle{alpha}

\newcommand{\ihb}{\frac{i}{\hbar}}
\newcommand{\xfase}{\mathbf{x}}
\newcommand{\yfase}{\mathbf{y}}
\newcommand{\qfase}{\mathbf{q}}
\newcommand{\pfase}{\mathbf{p}}
\newcommand{\xifase}{ {\boldsymbol{\xi}} }
\newcommand{\mufase}{ {\boldsymbol{\mu}} }
\newcommand{\Ifase}{\mathbf{I}}
\newcommand{\Pfase}{\mathbf{P}}
\newcommand{\Scat}{\mathbf{S}}
\newcommand{\Jsimp}{\mathbf{J}}
\newcommand{\Dom}{\mathbb{D}}
\newcommand{\Var}{\mathbb{M}}
\newcommand{\bra}[1]{\langle #1|}
\newcommand{\ket}[1]{|#1\rangle}
\newcommand{\braket}[2]{\langle #1|#2\rangle}
\newcommand{\Prom}[2]{\langle #1\rangle_{#2}}


\DeclareMathOperator*{\cod}{cod}
\DeclareMathOperator*{\traza}{traza}


\title{Blind Spots in the Nelson Potential}
\author{\\CBPF}


\begin{document}

\maketitle

\begin{abstract}

\end{abstract}

\section{General Idea}

The so called Nelson potential represents a possible collective
degree of freedom in nuclear physics. It has been studied and shows
chaotic behavior under classical mechanics \cite{Barang87}. 
Although the Hamiltonian system is not proved to be absolutely
chaotic (ergodic, mixing, hyperbolic) 
the integrable regions may be too
small to be detected in a wavelike approach. In the paper 
\cite{Zambrano09} it was shown that  under very 
general assumptions a large quantum state is orthogonal to some
small displacement of itself. In a semiclassical phase space
representation this appears as an isolated zero in the
Weyl Function. Here we present some possible locations for  this zeroes
in the case of the Nelson Potential. 

The technique is based on the realization that the
quantum chaotic state is not homogeneously spread over
the energy surfaces, but it presents structure as
scars, and it pours outside the classical
allowed region (see the figures in \cite{Barang93}).

We have populated the space between two 
energy shells with random points identified with
Dirac Deltas in a Center (Wigner) representation of
a quantum state. This gives a rough approximation
to a state which has structure but statistically 
fills the energy surface. 

\section{A preliminary estimation}

We are comparing the extent of a state to the relevant
unit of area in phase space, which is $\hbar$. This is 
more or less the size of the structures which should be,
at least in principle, discernible in Quantum Mechanics.

Along this
work our units are chosen so that $\hbar=1$ in the numerical
experiments.
We are taking energy surfaces which should be large
in comparison with this unit. In the following
set of figures we use the values $E=10$ a.u. and
$E=10.05$ a.u. which correspond to a difference of
$\hbar \omega_1$ in the eigenenergy.
An estimate for the volume between the two shells
is obtained and is around $30 \hbar^2$. The volume
inside the lower shell is of the order of $3000 \hbar^2$,
the relation is slightly higher than what we need.

\pagebreak

\section{Some Blind Spots}

\begin{figure}[H]
\begin{center}
\subfigure[Small sampling]{
\includegraphics[width=0.4\textwidth]{80000_10_NelsonW0-0-Countour.pdf}
\label{lowres}
}
\subfigure[$\xifase_1=(0,0)$]{
\includegraphics[width=0.4\textwidth]{500000_10_NelsonW0-0-Countour.pdf}
\label{hires00}
}
\subfigure[$\xifase_1=(0,0.5)$]{
\includegraphics[width=0.4\textwidth]{500000_10_NelsonW05-0-Countour.pdf}
\label{hires50}
}
\subfigure[$\xifase_1=(0.5,0.0)$]{
\includegraphics[width=0.4\textwidth]{500000_10_NelsonW0-05-Countour.pdf}
\label{hires05}
}
\caption{The nodal lines in the chord
function for the Nelson Problem, in various planes. First,
in the figure \ref{lowres} we present the $\xifase_1=(0,0)$ section
with a relative low sampling of points ($N=80000$)
in the center function. In the next three figures a larger sampling was
used ($N=500'000$), and three different planes are shown.
The red line is the nodal line in the imaginary part, and the red in the
real part. Outside the $\xifase_1=(0,0)$ is difficult to locate 
with precision the blind spots.}
\label{BlindSpots}
\end{center}
\end{figure}



\begin{thebibliography}{9}

\bibitem[EZ09]{Zambrano09}
A.~M. Ozorio de~Almeida E.~Zambrano.
\newblock Blind spots between quantum states.
\newblock {\em NJP}, page 113044, 2009.

\bibitem[Bar87]{Barang87}
M. Baranger and D.K.T.  Davies
\newblock Periodic Trajectories for a Two-Dimensional Nonintegrable Hamiltonian
\newblock {\em Ann. Phys.}, page 330, 1987.

\bibitem[Bar93]{Barang93}
M. Baranger and D. Provost
\newblock Semiclassical Calculation of Scars for a Smooth Potential
\newblock {\em PRL}, page 662, 1993.

\end{thebibliography}

\end{document}
