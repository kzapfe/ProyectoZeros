\documentclass[a4paper,12pt]{article}

\usepackage[utf8]{inputenc}
\usepackage{amsmath,amssymb}
\usepackage{graphicx}
\usepackage{subfigure}
%\usepackage[spanish]{babel}
\usepackage{bm}

\usepackage[cm]{fullpage}
\usepackage[light]{antpolt}
\usepackage[T1]{fontenc}

\usepackage{float}

\bibliographystyle{alpha}

\newcommand{\ihb}{\frac{i}{\hbar}}
\newcommand{\xfase}{\mathbf{x}}
\newcommand{\yfase}{\mathbf{y}}
\newcommand{\qfase}{\mathbf{q}}
\newcommand{\pfase}{\mathbf{p}}
\newcommand{\xifase}{ {\boldsymbol{\xi}} }
\newcommand{\mufase}{ {\boldsymbol{\mu}} }
\newcommand{\Ifase}{\mathbf{I}}
\newcommand{\Pfase}{\mathbf{P}}
\newcommand{\Scat}{\mathbf{S}}
\newcommand{\Jsimp}{\mathbf{J}}
\newcommand{\Dom}{\mathbb{D}}
\newcommand{\Var}{\mathbb{M}}
\newcommand{\bra}[1]{\langle #1|}
\newcommand{\ket}[1]{|#1\rangle}
\newcommand{\braket}[2]{\langle #1|#2\rangle}
\newcommand{\Prom}[2]{\langle #1\rangle_{#2}}


\DeclareMathOperator*{\cod}{cod}
\DeclareMathOperator*{\traza}{traza}


\title{ A Collection of Chord Functions}
\author{\\CBPF}


\begin{document}

\maketitle

\begin{abstract}

Here I present a collection of sections of the Chord Function for
the Nelson Potential in various sections. It can be seen that
inexactitude in the estimation of the energy shell causes appreciable
modification in the geometry of the nodal lines of the real and
imaginary part of the Chord Functions.

\end{abstract}

In the figure \ref{LD} we present the Weyl Function for
a sampling of 49000 Dirac Deltas between the energy surfaces corresponding
to $E=300$ and $E=300.05$.
 This presented two problems. This value of the energy is very
high, taking into account that the smallest estimate scale 
for the energy quanta is $\hbar\omega_1=0.05$.  The energy volume
trapped between the two mentioned surfaces thus has two very long and 
thin horns, which are difficult to fill evenly with pseudo random
points. The lack of points in the distal region for these horns produces
a curious effect on the Chord function. The pattern becomes less
round and has clear preference towards the horizontal and vertical
region, near the origin it appears as a chess board. The nodal
line in the imaginary part is still clearly visible along the
$\mu_p=0$ axis.
The section selected for that plot was $\xifase=(0,0)$.


\begin{figure}[H]
\begin{center}
\subfigure[Chord Function, Real part]{
\includegraphics[width=0.4\textwidth]{49000_300_Real.png}
\label{W49r}
}
\subfigure[Chord Function, Imaginary part]{
\includegraphics[width=0.4\textwidth]{49000_300_Imag.png}
\label{W49i}
}
\caption{Weyl or Chord function components, $E=300$, size of
sample $N=49000$. Only the $\xifase_1=(0,0)$ plane is shown.}
\label{LD}
\end{center}
\end{figure}

 
In the following pictures, fig. \ref{HD00}, fig. \ref{HD50}, 
and fig. \ref{HD05}, a 
much lower value of the energy was chosen, namely $E=60$. 
This doesn't result in the exaggeratedly long thin horns
for the equipotential curves, 
but is still a very
large value for the energy considered the estimated
energy quantum, $\hbar\omega_1=0.05$.
The resulting projection is depicted on the
figure \ref{Equipot}. We also have taken a much larger
sampling number for the centers, namely $5\cdot 10^5$.
This value blurs inhomogeneities in the nodal lines,
although a price is paid in time in a standard 
desktop computer.
The first nodal lines seem to run more or less where it is expected
in the $\xifase=(0,0)$ plane. We have a roughly elliptical
area in the real part where the Chord function is positive, and
an outer ring where it is negative.  The Imaginary part shows
the expected change of sign along the abscissa, and the
elliptical shape is oblate in the horizontal direction, instead
of the vertical. A curious feature is that both parts
show pikes protruding along the horizontal axis. My guess is that
this may be caused by inaccuracy in the process of  filling the horns in
the equipotential curve.  


\begin{figure}
\begin{center}
\includegraphics[width=0.45\textwidth]{DiracDeltaDistribution.png}
\caption{The centers fill the area inside the equipotential
curve $V(q_1, q_2)=E+\hbar\omega_1=60.05$ in this
position projection. }\label{Equipot}
\end{center}
\end{figure}

\begin{figure}
\begin{center}
\subfigure[Chord Function, Real part]{
\includegraphics[width=0.45\textwidth]{500000_60_NelsonW0-0-Real.png}
\label{W556r0}
}
\subfigure[Chord Function, Imaginary part]{
\includegraphics[width=0.45\textwidth]{500000_60_NelsonW0-0-Imag.png}
\label{W556i0}
}
\caption{Weyl or Chord function components, $E=60$, size of
sample $N=500000$. The $\xifase_1=(0,0)$ plane is shown.}
\label{HD00}
\end{center}
\end{figure}



\begin{figure}
\begin{center}
\subfigure[Chord Function, Real part]{
\includegraphics[width=0.4\textwidth]{500000_60_NelsonW0-05-Real.png}
\label{W556r1}
}
\subfigure[Chord Function, Imaginary part]{
\includegraphics[width=0.4\textwidth]{500000_60_NelsonW0-05-Imag.png}
\label{W556i1}
}
\caption{Weyl or Chord function components, $E=60$, size of
sample $N=500000$. The $\xifase_1=(0.5,0)$ plane is shown.}
\label{HD50}
\end{center}
\end{figure}






\begin{figure}
\begin{center}
\subfigure[Chord Function, Real part]{
\includegraphics[width=0.4\textwidth]{500000_60_NelsonW05-0-Real.png}
\label{W556r4}
}
\subfigure[Chord Function, Imaginary part]{
\includegraphics[width=0.4\textwidth]{500000_60_NelsonW05-0-Imag.png}
\label{W556i4}
}
\caption{Weyl or Chord function components, $E=60$, size of
sample $N=500000$. The $\xifase_1=(0,0.5)$ plane is shown.}
\label{HD05}
\end{center}
\end{figure}


Finally, we tested the method for an even lower energy value, namely
$E=10$. The function looks very similar in the low resolution sample
in fig. \ref{lower}, but the first nodal ellipsoid is bigger in the
real and imaginary parts. 


\begin{figure}
\begin{center}
\subfigure[Chord Function, Real part]{
\includegraphics[width=0.4\textwidth]{500000_10_NelsonW0-0-Real.png}
\label{Wlow2}
}
\subfigure[Chord Function, Imaginary part]{
\includegraphics[width=0.4\textwidth]{500000_10_NelsonW0-0-Imag.png}
\label{Wlow2}
}
\caption{Weyl or Chord function components, $E=10$, size of
sample $N=500000$. The $\xifase_1=(0,0)$ plane is shown.
Notice the change of scale in the $\mufase$ plane.}
\label{lower}
\end{center}
\end{figure}




\end{document}
