\documentclass[12pt]{iopart}

\def\II{\mathcal{I}}
\font\amsb=msbm10

\newcommand{\parcial}[2]{\ensuremath{\frac{\partial#1}{\partial #2}}}
\newcommand{\parcialn}[3]{\ensuremath{\frac{\partial^{#1}#2}{\partial#3^{#1}}}}
\newcommand{\dparcial}[3]{\ensuremath{\frac{\partial^{2}#1}{\partial#2\partial#3}}}
\newcommand{\braket}[2]{\ensuremath{\left\langle#1\big|#2\right\rangle}}
\def\hslash{\mbox{\amsb\char'176}}
\def\p{{\bm{p}}}
\def\q{{\bm{q}}}
\newcommand{\be}{\begin{equation}}
\newcommand{\ee}{\end{equation}}
\newcommand{\smeq}{\! = \!}
\newcommand{\smneq}{\! \neq \!}
\newcommand{\smpl}{\! + \!}
\newcommand{\smmi}{\! - \!}
\newcommand{\disp}{\displaystyle}
\newcommand{\vecp}{{\mathbf p}}
\newcommand{\vecq}{{\mathbf q}}
\newcommand{\vecI}{{\mathbf I}}

\newcommand{\vxi}{{\mathbf \xi}}
\newcommand{\x}{{\mathbf x}}
\newcommand{\tx}{\tilde{\mathbf x}}
\newcommand{\X}{{\mathbf X}}
\newcommand{\Xb}{\bar{\mathbf X}}
\newcommand{\dX}{\delta{\mathbf X}}
\newcommand{\y}{{\mathbf y}}
\newcommand{\Y}{{\mathbf Y}}
\newcommand{\varu}{{\mathbf u}}
\newcommand{\vl}{{\mathbf l}}	
\newcommand{\vecth}{{\mathbf \theta}}
\newcommand{\J}{{\mathbf J}}
%
\newcommand{\R}{{\mathbf R}}
\newcommand{\C}{{\mathbf C}}
\newcommand{\K}{{\mathbf K}}
\newcommand{\vP}{{\mathbf P}}
\newcommand{\vQ}{{\mathbf Q}}
\newcommand{\ve}{{\mathbf e}}
\newcommand{\W}{{W}}
\newcommand{\A}{{\hat{A}}}
\newcommand{\tA}{{\widetilde{A}}}
\newcommand{\B}{{\hat{B}}}
\newcommand{\tB}{{\widetilde{B}}}
\newcommand{\opA}{{\hat{A}}}
\newcommand{\opB}{{\hat{B}}}
\newcommand{\opC}{{\hat{C}}}
\newcommand{\opR}{{\hat{R}}}
\newcommand{\opT}{{\hat{T}}}
\newcommand{\oprho}{{\hat{\rho}}}
\newcommand{\oprhot}{{\hat{\rho_t}}}
\newcommand{\trho}{{\widetilde{\rho}}}
\newcommand{\mH}{{\mathbf H}}
\newcommand{\opH}{\hat{H}}
\newcommand{\opU}{\hat{U}}
\newcommand{\opI}{\hat{I}}
\newcommand{\tH}{{\widetilde{H}}}
\newcommand{\tS}{{\widetilde{S}}}
\newcommand{\opL}{\hat{L}}
\newcommand{\opLd}{\hat{L}^{\dagger}}
\newcommand{\oq}{\hat{q}}
\newcommand{\op}{\hat{p}}
\newcommand{\oqt}{\hat{q}^{\dagger}}
\newcommand{\opt}{\hat{p}^{\dagger}}
\newcommand{\oa}{\hat{a}}
\newcommand{\oat}{\hat{a}^{\dagger}}
\newcommand{\opx}{\hat{\mathbf x}}
\newcommand{\mA}{{\mathbf A}}
\newcommand{\M}{{\mathbf M}}
\newcommand{\F}{{\mathbf F}}
\newcommand{\vD}{{\mathbf \Delta}}
\newcommand{\zero}{{\mathbf 0}}
\newcommand{\RR}{{\mathbb R}}
\newcommand{\tw}{\widetilde{w}}
\newcommand{\tW}{\widetilde{W}}
\newcommand{\chf}{\chi}
\newcommand{\tV}{\widetilde{V}}
\newcommand{\Id}{{\mathbf I}}
\newcommand{\sh}{\mathop{sh}}
\newcommand{\vleqsim}{ {\scriptstyle {< \atop ^\sim}} }
\newcommand{\geqsim}{ {\scriptstyle {> \atop ^\sim}} }
\newcommand{\der}{\partial}
\newcommand{\scal}{\centerdot}
\newcommand{\GO}{{\mathcal O}}
\newcommand{\vlc}{\left\{}
\newcommand{\rc}{\right\}}

\renewcommand{\Im}{{\rm Im}}
\newcommand{\vct}[1]{\ensuremath\mbox{\boldmath$ #1 $}}
\newcommand{\Vxi}{\vct \xi}
\newcommand{\Veta}{\vct \eta}
\newcommand{\Vl}{\vct l}
\def\ro{\hat{\rho}}
\begin{document}
\title{Weak quantum ergodic conjecture: blind spots}
\author {Karel Zapfe, Fabricio Toscano and Alfredo M. Ozorio de Almeida\footnote{ozorio@cbpf.br}}
\address{Centro Brasileiro de Pesquisas Fisicas,
Rua Xavier Sigaud 150, 22290-180, Rio de Janeiro, R.J., Brazil}
\date{\today}
\date{13/10/2010}
\begin{abstract}

\end{abstract}



\maketitle
\section{Introduction}

The time average of any observable evolved by an ergodic classical Hamiltonian,
$H(\x)= H(p,q)$, is determined by the probability distribution, $N_E \delta(H(\x)-E)$,
where $\delta$ is the Dirac delta-function, E is the energy and $N_E$ 
is the normalization constant. The quantum ergodic conjecture of Voros \cite{Voros76} and Berry \cite{Berry77b}
is that this same distribution can be reinterpreted as the semiclassical limit 
of the Wigner fuction \cite{Wigner}, $W(\x)$, for an eigenstate of the corresponding quantum Hamiltonian 
that has the same eigenenergy E. 

The immediate consequence of this conjecture concerns the expectation value of quantum observables.
The Weyl representation, $A(\x)$, of an operator, $\opA$, is,
in the case of an observable, semiclassically close to the corresponding classical function.
Since the expectation value is given by
\begin{equation}
\langle \opA \rangle = \int d\x \;W(\x)\;A(\x) \ ,
\label{expectation}
\end{equation}
it follows that the conjecture leads to the same expectation value that would be obtained classically.
This is, indeed, confirmed by Shnirelman's theorem \cite{Shnirelman, Verdiere, Zelditch} for a majority of eigenstates, 
albeit with severe restrictions for the allowed classes both of the observables and of the ergodic systems.

It is important to note that, while the Shnirelman theorem concerns individual states 
and expectation values, it is not formulated directly in terms of Wigner functions.
Conversely, Wigner functions that are peaked on the classical energy shell
have been derived semiclassically, but for mixed states within a narrow energy window.
Even so, the appropriate form is given in terms of highly oscillatory Airy functions, 
rather than delta functions.
Further Wigner oscillations arise if one reduces the width of the energy window, 
so that one may presume individual pure Wigner functions to have complex detailed pattern,
full of negative regions and interferences, that is far removed from the original quantum ergodic conjecture.

There are at least three reasons which confirm the need to modify this conjecture.
First, there is the theorem by Hudson \cite{Hudson} that the only non-negative Wigner function for a pure state
is the Gaussian representation of  a coherent state. A second important result is the proof by Balazs \cite{Balazs}
that a delta function on a curved manifold does not represent a positive operator.
\footnote{The construction of a positive operator that would otherwise have a negative expectation
is only given for one degree of freedom, but it can easily be generalized.}
Last but not least, there is the fact that correlations 
of a true Wigner function for an individual pure state are invariant 
with respect to Fourier transformation \cite{Chountasis, OVS}. 
Thus, a large scale structure (the energy shell itself) must be accompanied 
by oscillations with large spatial wave vectors,
that is, at very fine scales. It follows that
expectation values that mimmick those of classical mechanics
can only arise because the smooth functions corresponding to typical observables
average out fine Wigner oscillations in (\ref{expectation}).
The conclusion is that the original quantum ergodic conjecture of Voros and Berry 
is too strong because it relies on a complete description of the quantum state,
which cannot be correct.

One might then question the value of aiming a conjecture at a full representation
of a quantum ergodic state, such as the Wigner function, 
if all it did were to determine good expectation values. 
However, one also obtains predictions that are much more subtle than the gross
correspondence of expectations to their classical values. Most notable, so far, is the use 
of the ergodic conjecture by Berry to obtain local satistical correlations of wave functions.
Eventually, the ever increasing experimental refinement in manipulating quantum states
may allow one to access other delicate properties, such as interference phenomena,
for ergodic eigenstates.

How can one maintain the higher predictive power of the Wigner function mode 
of the ergodic conjecure, without stumbling on its shortcomings? The most promissing alternative
is to focus on the expectation values of the unitary {\it translation operator}
\begin{equation}
\hat{T}_{\Vxi} = \exp{\left[\frac{i}{\hbar}(\Vxi\wedge \hat{\x})\right]}=
\exp{\left[\frac{i}{\hbar}(\Vxi_p\cdot \hat{q}-\Vxi_q\cdot \hat{p})\right]} \ ,
\end{equation}
which acts on the state $|\psi\rangle$ to produce the new state
$|\psi_{\Vxi}\rangle=\hat{T}_{\Vxi}|\psi\rangle$ in strict
correspondence to the classical translation, $\x \mapsto \x + \Vxi$,
by the {\it chord}, $\Vxi$.
%
\footnote{ In the optical context $\hat{T}_{\Vxi}$ is
usually referred to as the {\it displacement operator} and is
expressed in terms of creation and annihilation operators for the
harmonic oscillator.}
The expectation of the translation operator in (\ref{expectation}) 
is just the Fourier transform of the Wigner function:
\begin{equation}
\langle\hat{T}_{-\Vxi}\rangle = 
\int d\x \;W(\x)\; \exp\left\{\frac{i}{\hbar}(\Vxi\wedge \x)\right\} = \chi(\Vxi) .
\label{chift}
\end{equation}
This is known as the {\it chord function} \cite{OzReport}, 
as one of the {\it quantum characteristic functions} of quantum optics \cite{Leonhardt}
(or the {\it Weyl function} as in \cite{Chountasis}); 
it is a full representation of the quantum state.
Evidently, both the real and the imaginary parts of 
$\exp{\left\{\frac{i}{\hbar}(\Vxi\wedge \x)\right\}}$
are smooth functions of $\x$, so that the chord function is insensitive to the high
frequency oscillations of $W(\x)$ if $\Vxi$ is small, 
just as one finds for the expectations of typical observables.

Expectation values in the full chord representation are given by a similar formula to (\ref{expectation}), i.e.,
\begin{equation}
\langle \opA \rangle = \int d\x \;\chi(-\Vxi)\;\tilde{A}(\Vxi) \ ,
\label{chordexpect}
\end{equation}
in which the {\it chord symbol}, $\tilde{A}(\Vxi)$, is the Fourier transform of $A(\x)$. 
So, any power of products of position or momentum operators has its chord symbol given
by derivatives of delta functions at the origin. For example, we obtain explicitly
the moments
\be
\langle\hat\p^n\rangle=(-i\hbar)^{n}\left.\frac{\partial^n\chi}{\partial\xi_q^n}\right|_{\Vxi=0}
\quad\quad\textrm{ and }\quad\quad
\langle\hat\q^n\rangle=(i\hbar)^{n}\left.\frac{\partial^n\chi}{\partial\xi_p^n}\right|_{\Vxi=0}.
\label{deri-moment}
\ee
Hence, full knowledge of the chord function is not required in order to obtain
the expectation value of a smooth observable, only its behaviour near the origin.
It thus turns out that we obtain the same predictions in the case of quantum ergodicity 
from the {\it weak quantum ergodic conjecture}, namely
\be
\chi_E(\Vxi) \rightarrow N_E \int d\x \; \exp\left\{\frac{i}{\hbar}(\Vxi\wedge \x)\right\} \; \delta(H(\x)-E).
\label{weakhyp}
\ee
Here, both 'limits', $\Vxi\rightarrow 0$ and $\hbar\rightarrow 0$, should be interpreted 
in a heuristic sense. This is familiar in the use of the 'semiclassical limit', in which
good approximations may follow from the extrapolation to finite values of Planck's constant
of results derived in the strict asymptotic limit, $\hbar \rightarrow 0$.
Just so, with the short chord limit: It will be shown here how 
the strict limit $\Vxi \rightarrow 0$ can be extended to a finite neighbourhood of the origin,
so as to rederive the previous conjecture for local correlations of quantum ergodic eigenfunctions.

Better still, one finds that this reasonable extension leads to the prediction of new delicate
features concerning the orthogonality of a state with its translation, which, nonetheless, 
depend entirely on the classical backbone of the state. The pattern for these special orthogonal 
translations, refered to as {\it blind spots}, have indeed been found to be determined by the underlying
classical structure in the case of superpositions of coherent or squeezed states and also for the analogous
approximation to (\ref{weakhyp}) for a single degree of freedom, for which ergodicity is trivial.
For integrable systems with higher degrees of freedom, states of similar energy will have different
patterns of blind spots, whereas the weak ergodic conjecture leads to nearly identical patterns
(near the chord origin) for all ergodic eigenstates lying within a narrow energy window. 

.................................................................................................

\cite{Wigner}:
The latter can be redefined, following Royer \cite{OzReport,Royer}, as
\be
W(\x)=\frac{1}{(\pi\hbar)}\langle\psi|\hat{R}_{\x}|\psi\rangle,
\label{Wrefl}
\ee
where $\hat{R}_{\x}$, the Fourier transform of the translation operators,
corresponds classically to the phase space reflection through the {\it reflection centre} $\x$,
i. e. $\x_0 \mapsto 2\x - \x_0$.
 







\section{Small chords and moments of position and momentum}




The definition of the chord function (\ref{chift}) allows us to calculate the distribution moments, or statistical moments, of $\hat\p$ and $\hat\q$ in the form of the derivatives of the chord function, i.e. explicitly
Conversely, if we know all the moments, then we know the chord function, 
because the expansion in a Taylor series of the chord function is
\be
\chi(\Vxi)=
\sum_{n=0}^{\infty}\frac1{n!}\sum_{k=0}^{n}\frac{(-1)^{k}}{(i\hbar)^n}\left(\hspace{-.2cm}\begin{array}{c}n\\n-k\end{array}\hspace{-.2cm}\right)%\\\times
\left\langle\mathcal{M}\left(\hat\q^{n-k}\hat\p^{k}\right)\right\rangle\hspace{.1cm}\xi_q^k\xi_p^{n-k},
\label{X-exp-momentdis}
\ee
where,
\be
\mathcal{M}\left(\hat\q^{n}\hat\p^{k}\right)=
\frac{1}{n+k}\sum_{P_{nk}}\hat\q^{n}\hat\p^{k}
\label{permutations}
\ee
and $P_{nk}$ all possible permutations of products of $q^n$ and $p^k$. 
The equation \eref{permutations} corresponds to the symmetrization of the product $\q^n\p^k$, 
being the important feature of the Weyl symbols, which guarantees the symplectic invariance of the chord function.
\par
Thus, it is clear that knowledge of the chord function in an infinitesimal neighbourhood of the origin
is equivalent to information on the expectation for all observables in the form of polynomials in position and momenta.
Of course, the asumption of knowledge for all statistical moments and of analiticity for the chord function
would then determine it everywhere in the phase space of chords. Our conjecture is that (\ref{weakhyp})
remains a good approximation for large enough chords to provide none-obvious information. 

(deduction of Berry correlations from chord function)








\section{ Global correlations, nodal surfaces and blind spots}

The local correlation function obtained above describes statistical properties of a particular,
though important representation, i.e. the wave function. In contrast, the overlap of a density operator, $\ro$,
with its translation, $\ro_{\Vxi} = \hat T_{\Vxi}\ro\hat T^\dag_{\Vxi}$,
\begin{equation}
C(\Vxi)\equiv\tr [\ro \; \ro_{\Vxi}]=\frac{1}{(2\pi\hbar)^L}\int |\chi(\Veta)|^2 \; e^{-i\Vxi\wedge\Veta/\hbar}d\Veta
= |\chi(\Vxi)|^2,
\label{correlchi}
\end{equation}
is independent of any representation, even if we have here given two expressions in terms of the chord function.
The convenience of this particular representation is that simply:
\be
\chi_{\Vxi}(\Veta)= e^{-i\Vxi\wedge\Veta/\hbar} \; \chi(\Veta).
\label{transchord}
\ee
The last equation in (\ref{correlchi}) shows that the zeros of this global correlation function are just those of the chord function itself.
Some basic facts about correlations and the chord function that were discussed in \cite{ZOA10} 
will be generalized now for more than one degree of freedom, 
because of their relevance to the weak quantum ergodic conjecture.

The chord function \eref{chift} is the Fourier transform of the real Wigner function and is in general complex.
Indeed, the fact that it represents a Hermitian operator, only implies the constraint, 
$\chi(-\Vxi) = \chi(\Vxi)^*$, where the asterix denotes complex conjugation.
On the other hand, the cosine and the sine transforms, $c(\Vxi)$ and $s(\Vxi)$ of the Wigner function are real, 
i.e. the real and imaginary part of $\chi(\Vxi)$, respectively. 
We now construct the hermitian operators 
\begin{equation}
\hat c_{\Vxi} = \frac{\hat{T}_{\Vxi} + \hat{T}_{-\Vxi}}2
\quad\textrm{ and }\quad
\hat s_{\Vxi} = \frac{\hat{T}_{\Vxi} - \hat{T}_{-\Vxi})}{2i},
\end{equation} so that 
$c(\Vxi) = \langle\psi|\hat c_{\Vxi}|\psi\rangle$  and
$s(\Vxi) = \langle\psi|\hat s_{\Vxi}|\psi\rangle$.
\footnote{ An alternative interpretation is to consider $c(\Vxi)$ and $s(\Vxi)$
to be the chord functions for appropriately symmetrized states \cite{Blind}.} 
In the case when the state $|\psi\rangle$ has a centre of symmetry
(i.e. there exists a centre, $\x$, such that $\R_{\x}|\psi\rangle = \pm |\psi\rangle$)
then $s(\Vxi)=0$, so that $\chi(\Vxi)=c(\Vxi)$ \cite{OVS, Blind}. 

The real functions $c(\Vxi)$ and $s(\Vxi)$ will generically have nodal surfaces of codimension-1
in the $2L$-dimensional phase space of chords.
In the general case where there is no reflection symmetry, an intersection of a nodal surface
of $c(\Vxi)$ with a nodal surface of $s(\Vxi)$ defines a codimension-2 surface, 
the locus of translations for which the translated state, $|\psi_{\Vxi}\rangle$, 
becomes orthogonal to $|\psi\rangle$. This is strictly only a {\it blind spot} \cite{Blind,ZOA10}
in the case where $L=1$, but it remains true that these remarkable translations will appear as
isolated points on general 2-dimensional sections of the chord space, where they intersect
the codimension-2 {\it blind manifold}. 
Because $\chi(0)= 1$, the origin lies on a nodal surface of $s(\Vxi)$.
In a neighborhood of the origin, the chord function may be approximated by
\begin{eqnarray}
%\fl
\chi(\Vxi)&=&\langle\hat T_{-\Vxi}\rangle
\simeq
\left\langle1-\frac{i}{\hbar}\Vxi\wedge\hat \x-\frac{1}{\hbar^2}(\Vxi\wedge\hat \x)^2+\cdots\right\rangle
\\&=&
1-\frac{i}\hbar\Vxi\wedge\langle\hat \x\rangle-\frac{1}{\hbar^2}\langle(\Vxi\wedge\hat \x)^2\rangle+\cdots
\label{aproxto2}
\end{eqnarray}
At first sight, it appears that the nodal surface of $s(\Vxi)$ crossing the origin is locally parallel 
to the direction of $\langle\hat\x\rangle$, because the behaviour of the chord function at the origin 
is dominated by the first moment. However, this is an error in \cite{ZOA10}, where it was overlooked
that $\langle\hat\x\rangle$ only determines the overall phase of the chord function, because of (\ref{transchord}).
Thus, it will be the third moments that determine the direction of the nodal surface of $s(\Vxi)$ at the origin.
 
On the other hand, because the origin is a local maximum of the chord function, 
the nodal surfaces of $c(\Vxi)$ for small chords avoid the origin. 
It follows from \eref{aproxto2} that the  closest nodal surface surrounding the origin is given approximately by
\begin{equation}
\langle(\Vxi\wedge\hat \x)^2\rangle =  \Vxi \mathbf K \Vxi = \hbar^2,
\label{ellipse}
\end{equation} 
if we neglect higher order terms. This positive quadratic form is defined in terms of 
the Schr\"odinger covariance matrix 
\cite{Schr}, $\mathbf K$, which establishes the extent of the state in phase space.
The nodal line of $c(\Vxi)$ is thus approximated by the ellipsoid \eref{ellipse} and the closest blind spots 
lie along the intersection of this ellipsoid with the nodal surface of $s(\Vxi)$ that contains the origin. 
It is important to note that the present estimate for the pair of {\it closest blind spots}
depends only on the first moments.

The highly excited Bohr-quantized states for $L=1$ have a covariance matrix that is well described 
by the classical averages discussed in the previous section, i.e. $\det \mathbf K \gg (2\pi\hbar)^2$. 
Thus, the ellipse \eref{ellipse} lies in the deep interior of a neighbourhood of the origin 
with $\sqrt{\hbar}$ linear dimensions. The same behaviour is here predicted for the blind
spots resulting from the intersection of the approximate ellipsoidal nodal surface of $c(\Vxi)$
with the nodal surface of $s(\Vxi)$ that contains the origin, in the case of quantum ergodic states.
This is in line with the discussion in \cite{Blind}: Notwithstanding the delicate quantum nature
of blind spots, they can be found in the `classical' neighbourhood of the origin and they
are precisely determined by classical features. This apparent paradox is resolved by
the reciprocal relation between large and small scales of pure states in phase space
that follows from the universal invariance of the intensity of the chord function for pure states
with respect to Fourier transformation \cite{Chountasis, OVS}, that appears in (\ref{correlchi}).
So far, we have only estimated the closest blind spots to the origin. It is hopeless to pursue the Taylor
expansion \eref{aproxto2} any further to find further orthogonalities. On the other hand, 
the real and imaginary parts of the chord function (\ref{weakhyp})
have many nodal surfaces in the region where it may approximately hold.
  






\section{Characteristic function for classical energy shells}

The Fourier transform of a Dirac delta-function on an energy shell can be obtained analytically
in the case of the harmonic oscillator. A linear canonical transformation (which leaves the chord function
invariant) simplifies the Hamiltonian to
\be
H(\x) = \sum_l {\frac{\omega_l}{2}({p_l}^2 + {q_l}^2)}. 
\ee
Thus, the Fourier transform of a classical probability density that is just a delta function on the energy shell,
inserted in (\ref{weakhyp}), becomes
\be
\chi_E(\Vxi)= \frac{2^L N_E}{\sqrt{\omega_1...\omega_L}}\int dpdq \; 
\exp\left\{\sum\frac{2i}{\hbar\omega_l} \Vxi_l\wedge \x_l\right\}\; \delta(\sum ({p_l}^2 +{q_l}^2)-E) = J(|\Vxi|)
\label{HOchi}
\ee
This is a real function, as it should be, because the energy shell is symmetric around the origin,
and it is isotropic. Also one verifies that the nodal surfaces, concentric spheres, 
pack in closer to the origin with increasing energy.

In the case of a single degree of freedom, we have $\chi = J_0$ with circular nodal lines.
This is the approximation for the harmonic oscillator eigenstates, obtained in \cite{OVS}
for short chords. The eigenstates of higher dimensional harmonic oscillators correspond
to separable tori, so the classical distribution will be a product of delta functions
over the quantized circles in each of the conjugate phase space planes.  
From this, one immediately obtains the short chord approximation for the eigenstates
as products of Bessel functions with nodal lines along the cylinders, $J_0(|\xi|)=0$,
which is far removed from the characteristic function for a full energy shell.

In the absence of a full semiclassical theory for chaotic eigenstates,
we resort to numerical calculations of the classical characteristic function for a particular system,
so as to compare it with the quantum chord function in the following section.
The classical {\it Nelson Hamiltonian} is defined as \cite{BarDav},
\begin{equation}
\label{hamilnelson}
\mbox{H}({\bf x}_1,{\bf x}_2)=
\frac{p_1^2+p_2^2}{2} + \frac{\omega_1^2\,q_1^2}{2}+\omega_2(q_2-\frac{q_1^2}{2})^2
\;,
\end{equation} 
where ${\omega_1}^2= .1$ and $\omega_2= 1$,
and its quantum counterpart results from the replacement: 
${\bf x}_1\equiv(q_1,p_1)\rightarrow (\hat{q}_1,\hat{p}_1)$ and
${\bf x}_2\equiv(q_2,p_2)\rightarrow (\hat{q}_2,\hat{p}_2)$ in (\ref{hamilnelson}).
The restriction of the  classical Hamiltonian (\ref{hamilnelson}) 
to the 2-D $\X'$-planes $q_1=Q^{'}$ and $p_1=P^{'}$
defines harmonic oscillators in the $\x_2$ variables: 
$h_2(\x_2)=\mbox{H}(\X',{\bf x}_2)$. 
The classical trajectories of this harmonic oscillator
coincide with the intersection of the energy shell of the full Hamiltonian 
with the planes $q_1=Q^{'}$ and $p_1=P^{'}$.
The alternative classical sections $\x_2=\X'$ on (\ref{hamilnelson}) defines
anharmonic oscillators, $h_1(\x_1)=\mbox{H}({\bf x}_1,\X')$, 
whose closed trajectories are not ellipses.  

The classical dynamics of this system has been studied in considerable depth.
Instead of merely relying on the Poincar\'e section for a few typical trajectories
(exhibited in \cite{BarDav, RibAg}
(see \cite{PradAg} and references therein)
made extensive studies of a large number of the periodic orbits 
with relatively low period and their elaborate bifurcation trees as a function of energy. 
It was possible to follow these families of periodic orbits as they
become unstable in an energy range above $E=0.3$, though a few recover 
stability at $E\approx 10$. All quantum states here depicted were calculated
in an energy window  $0. 81< E < 0.84$, 
where it can be guaranteed that only orbits of very high period
may be stable. Hence, the knowledge of the general pattern 
for the dynamics from the skeleton of periodic orbits,
precludes any stability islands that are not very thin, so that
the system is nearly ergodic. 
The choice of $\hbar=0.05$ allows us to compute the eigenstates
in the energy range corresponding to (quasi-)chaotic motion for this classical Hamiltonian.

The Nelson Hamiltonian has no centre of symmetry, so that the immaginary part of the chord function
is not identically zero. Hence there will be intersections of the real and the immaginary
nodal surfaces. The Fourier transform of the energy shell was calculated numerically,
by averaging $\exp\left\{\frac{i}{\hbar}(\Vxi\wedge \x)\right\}$ for discrete points 
placed randomly in a narrow neighbourhood of the energy shell. This method was callibrated
by calculating the exact Bessel functions for the harmonic oscillator.




\section{Comparison of ergodic and direct numerical chord functions}

It would be natural to use a harmonic oscillator basis 
for calculating those of the Nelson Hamiltonian,
but it is more efficient to use a basis of {\it distorted oscillators} \cite{TosOa99}.
 

%%%%%%%%%%%%%%%%%%%%%%%%%%%%%%%%%%%%%%%%%%%%%%






\section{Discussion}





The chord function portrays a pure state by exhibiting its overlap with all its possible translations.
There are cases where such a state has a clear classical correspondence: the superposition of well
separated coherent or squeezed states treated in \cite{ZO}, Bohr-quantized states analyzed in \cite{ZOA10}
and their integrable generalization to higher degrees of freedom. In all these cases, it has been found
that the chord function is determined by a purely classical average in a nontrivial neighbourhood of short chords.
Our conjecture here merely extends this scenario to quantum ergodic states.
Contrary to naive considerations, the decay in the square modulus of the overlap is not smooth 
and classical like, within this neighbourhood with roughly a volume of $\hbar^L$. 
We have here verified computationally that blind spots, denoting zero overlap, 
arise deep within this classically small neighbourhood.
This feature depends basically on the Schr\"odinger covariance matrix: The greater its determinant,
the closer to the origin will the blind spots lie.  

To obtain complete information about a quantum state, one must specify the chord function
everywhere. It is certain that sufficiently large translations, 
for there to be no longer a classical overlap with the original energy shell,
lie in an evanescent region of the chord function, just as occurs for one degree of freedom \cite{ZOA10}.
No prediction is here made about the rich structures that lie in between this outer region and
the neighbourhood of the origin for quantum ergodic states. Indeed, the essence of our conjecture
is to specify the chord function only in a small but relevant region. Thus one avoids the
pitfalls of overspecifying the Wigner function in the original form of the conjecture.

One possibility for future investigation is to substitute the delta-function over the
energy shell in the definition of the chord function for the quantum ergodic conjecture
by an Airy function. This is the correct form for the Wigner function for a mixture of
eigenstates in a narrow energy window [Berry, Toscano...]. Thus, even if it gives the same
chord function in the limit of short chords, it may be a better approximation for longer chords.




\ack{We thank Rodolfo Jalabert and Raul Vallejos for interesting discussions. Financial support by...
is gratefully acknowledged.} 



\section*{References}
\begin{thebibliography}{99}
\bibitem{Voros76} Voros A 1976 Ann. Inst. Henri Poincare $\bm{26}$ 31
\bibitem{Berry77b} Berry M. V. 1977 Regular and irregular semiclassical wavefunctions,
J. Phys. A {\bf 10}, 2083-2092.
\bibitem{Wigner} Wigner E P 1932 \emph{Phys Rev} $\bm{40}$ 749-759
\bibitem{Shnirelman} Shnirelman A. 1974 Ergodic properties of eigenfunctions, Uspekhi Mat. Nauk. {\bf 29}, 181-182.
\bibitem{Verdiere}  Colin de Verdi\`ere Y. 1985 Ergodicit� et fonctions propres du laplacien, 
Comm. Math. Phys. {\bf 102}, 497-502.
\bibitem{Zelditch} Zelditch S Math Encyclopedia
\bibitem{Hudson} Hudson R L 1974 Rep. Math Phys. $\bm{6}$ 249
\bibitem{Balazs} Balazs N Physica A
\bibitem{Chountasis} Chountasis S and Vourdas A 1998
         \emph{Phys. Rev. A} $\bm{58}$ 848 - 855 
\bibitem{OVS} Ozorio de Almeida A M, Vallejos R and Saraceno M 2005
        \emph{J. Phys. A: Math. Gen} $\bm{38}$ 1473-1490 
\bibitem{OzReport} Ozorio de Almeida A M 1998
        \emph{Phys. Rep.} $\bm{295}$ 265
\bibitem{ZOA10} Zambrano E and Ozorio de Almeida A M 2010 \emph{J. Phys. A} $\bm{43}$ 000000 
\bibitem{Blind} Zambrano E and Ozorio de Almeida A M 2009 
        to appear in \emph{New J Phys}
\bibitem{BarDav} Baranger M. \& Davies K. T. R. 1987 Periodic trajectories for a two-dimensional nonintegrable
Hamiltonian,  Ann. Phys. (N.Y.) {\bf 177}, 330.
\bibitem{RibAg} Ribeiro A.D., de Aguiar M.A.M. and Baranger M. 
Semiclassical Approximations Based on Complex Trajectories, Phys. Rev. E69 (2004) 66204.
\bibitem{PradAg} Prado S. P. and de Aguiar M. A. M. 1994 Effects of Symmetry Breakdown 
in the Bifurcations of Periodic Orbits of a Nonintegrable Hamiltonian System, 
Ann. Phys. (N.Y.) {\bf 231},  290-310.
\bibitem{TosOa99}  Toscano F. \& Ozorio de Almeida, A. M. 1999
Geometrical approach to the distribution of the zeros for the Husimi function,
J. Phys. A {\bf 32}, 6321-6346.
\bibitem{Berry77} Berry M V 1977
             \emph{Phil. Trans. R. Soc.} $\bm{287}$, 237
\bibitem{VanVleck} Van Vleck J H 1928
        \emph{Proc. Math. Acad. Sci.} U.S.A $\bm{14}$, 178 - 188
\bibitem{Maslov} Maslov V P and Fedoriuk M V 1981
        \emph{Semiclassical Approximation in Quantum Mechanics}(Reidel, Dordrecht)
        (translated from original russian edition, 1965).
\bibitem{Littlejohn} Littlejohn R G 1995 
             in: \emph{Quantum Chaos} (Cambridge Univ. Press: Cambridge)  Casati G, \emph{et al} (Eds)
\bibitem{OzLivro} Ozorio de Almeida A M 1988
            \emph{Hamiltonian systems: Chaos and quantization} (Cambridge: Cambridge Univ. Press)
\bibitem{Royer} Royer A 1977 \emph{Phys Rev A} $\bm{15}$ 449
%\bibitem{Ozorio83} Ozorio de Almeida 1983
%             \emph{Ann Phys} 
\bibitem{ZO} Zambrano E and Ozorio de Almeida A M  2008
        \emph{Nonlinearity} $\bm{21}$ 783-802 
\bibitem{Leonhardt} Leonhardt U 1997
        \emph{Measuring the quantum state of light} (Cambridge: Cambridge Univ. Press)
\bibitem{Berry76} Berry M V
        \emph{Adv. in Phys.} $\bm{25}$ 1, 1-26
\bibitem{abramowitz} Abramowitz M e Stegun I,
        \emph{Handbook of Mathematical Functions} (New York: Dover)
        (1964)
        (1973)
\bibitem{Schr} Schr\"odinger E 1930 
        \emph{Proc Pruss Soc Acad Sci} {\bf 19} 296
\end{thebibliography}
\end{document}
