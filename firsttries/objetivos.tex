\documentclass[a4paper,10pt]{article}

\usepackage[utf8]{inputenc}
\usepackage{amsmath,amssymb}
\usepackage{graphicx}
\usepackage{subfigure}
\usepackage[spanish]{babel}
\usepackage{bm}

\bibliographystyle{alpha}


\newcommand{\xfase}{\mathbf{x}}
\newcommand{\qfase}{\mathbf{q}}
\newcommand{\pfase}{\mathbf{p}}
\newcommand{\xifase}{ {\boldsymbol{\xi}} }
\newcommand{\mufase}{ {\boldsymbol{\mu}} }
\newcommand{\Ifase}{\mathbf{I}}
\newcommand{\Pfase}{\mathbf{P}}
\newcommand{\Scat}{\mathbf{S}}
\newcommand{\Jsimp}{\mathbf{J}}
\newcommand{\Dom}{\mathbb{D}}
\newcommand{\Var}{\mathbb{M}}
\newcommand{\bra}[1]{\langle #1|}
\newcommand{\ket}[1]{|#1\rangle}
\newcommand{\braket}[2]{\langle #1|#2\rangle}




\DeclareMathOperator*{\cod}{cod}
\DeclareMathOperator*{\traza}{traza}


\title{Objetivo del Proyecto de los Puntos Ciegos}
\author{ W. P. Karel Zapfe\\CBPF}


\begin{document}

\maketitle

\begin{abstract}

Después de hablar como por tercera vez sobre los objetivos,
ya comienzas a tener clara una idea de por donde va el trabajo.

\end{abstract}

\section{Preeliminares}

El supuesto ergódico cúantico se formula así. En el límite de alta energía,
o, equivalentemente, de $\hbar$ pequeñísimo, hacemos de cuenta
que el valor esperado de un operador hermitiano descente (es decir,
con una variable clásica bien asociada) corresponde con el promedio
sobre la camada de energía. Al parecer hay resultados serios sobre esa
conjetura \cite{Zeldtich}. Esto se expresa en el lenguaje que nos gusta
como
\begin{equation}
  \bra{n}f_c(\hat{p},\hat{q}) \ket{n}\approx \int d\xfase 
\delta(H(\xfase)-E_n) f_c(p,q),
\end{equation}
para $n$ suficientemente grande. 

Dado que la función de Wigner actúa como una cuasi distribución,
la gente asume que en este caso 

\begin{equation}
W(\xfase)\approx \delta(H(\xfase)-E_n)
\end{equation}

Esto parece ser una aceptable aproximación,, e inclusive otorga 
una seríe de resultados correctos, como en \cite{Berry78}.
Sin embargo no hay manera de que está aproximación sea correcta.
Existen momentos de las distribuciones que resultan 
francamente errados. A pesar de eso, M. Berry obtiene resultados 
aceptables para las funciones de correlación.
Una expresión más descente (una \emph{aproximación uniforme}) sería
la siguiente:

\begin{equation}
\begin{split}
W(\xfase)&=\sum_n \exp((-E_n -E)/d^2) \int d\xfase \delta(H(\xfase)-E_n)\\
&\approx Ai(H(\xfase)-E).
\end{split}
\end{equation}

Ahora bien, el supuesto incorrecto tiene una implicación maluca
en las correlaciones locales, digamos de orden $\epsilon$
\begin{equation}
\psi^*(q+\epsilon)\psi(q-\epsilon)=\int dp W(q,p)e^{-ip\epsilon/\hbar}
\end{equation}
y
\begin{equation}
C_q(\epsilon)=\int_{local} dq \psi^*(q+\epsilon)\psi(q-\epsilon)
\end{equation}

La función de correlación utilizando las deltas no da parte de la
estructura cuántica, por ejemplo, faltan los puntos ciegos. Por
ende, la delta de Dirac es una función de Wigner escencialmente 
incorrecta. 


\section{Objetivos}

Vamos a pensar que en sistemas ergódicos (y probablemente mezclantes)
que la función de cuerdas \emph{se parece} a la transformada de
Fourier de una delta, siempre y cuando las cuerdas sean suficientemente
pequeñas (es decir, correlaciones cercanas):

\begin{equation}
\chi(\xifase)\xrightarrow{\xifase\rightarrow 0}
\int d\xfase \delta(H(\xfase)-E_n) e^{i\xifase\wedge\xfase/\hbar}
\end{equation}

Con está aproximación surfen las siguientes preguntas:
Qué tan largas cuerdas podemos usar? Necesitamos llegar al primer
punto ciego al menos. Estas estructuras subplanckianas no
resultan si usamos la Wigner como delta. 
Es posible extraer las oscilaciones de la función de cuerdas?


Sabemos que la función de cuerdas funciona como una función 
generadora de momentos estadísticos. Vamos a ver que tal funciona eso:

\begin{equation}
\bra{n}\hat{p}^l\ket{n}=\frac{i^l \partial ^l}{\hbar^l \partial q^l}
\chi(\xifase) \lvert_{q=0}
\end{equation}


Usaremos esto para calcular las correlaciones $C_q(\epsilon)$.
La función de cuerdas como representación del estado cuántico
funciona, la que esta errada desde el principio es la de Wigner.
Susutituiremos la suposición ergódica por lo que Alfredo llama
hipótisis ergódiga débil. 
Finalmente calcularemos los puntos ciegos cerca del origen.


\bibliography{ziegos}

\end{document}
