\documentclass[a4paper,12pt]{article}

\usepackage[utf8]{inputenc}
\usepackage{amsmath,amssymb}
\usepackage{graphicx}
\usepackage{subfigure}
%\usepackage[spanish]{babel}
\usepackage{bm}

\usepackage[cm]{fullpage}
\usepackage[light]{antpolt}
\usepackage[T1]{fontenc}


\bibliographystyle{alpha}

\newcommand{\ihb}{\frac{i}{\hbar}}
\newcommand{\xfase}{\mathbf{x}}
\newcommand{\qfase}{\mathbf{q}}
\newcommand{\pfase}{\mathbf{p}}
\newcommand{\xifase}{ {\boldsymbol{\xi}} }
\newcommand{\mufase}{ {\boldsymbol{\mu}} }
\newcommand{\Ifase}{\mathbf{I}}
\newcommand{\Pfase}{\mathbf{P}}
\newcommand{\Scat}{\mathbf{S}}
\newcommand{\Jsimp}{\mathbf{J}}
\newcommand{\Dom}{\mathbb{D}}
\newcommand{\Var}{\mathbb{M}}
\newcommand{\bra}[1]{\langle #1|}
\newcommand{\ket}[1]{|#1\rangle}
\newcommand{\braket}[2]{\langle #1|#2\rangle}
\newcommand{\Prom}[2]{\langle #1\rangle_{#2}}


\DeclareMathOperator*{\cod}{cod}
\DeclareMathOperator*{\traza}{traza}


\title{ Wigner Function Approximations for Chaotic Systems}
\author{\\CBPF}


\begin{document}

\maketitle

\begin{abstract}


The Center or Wigner Function is used as a quasi-distribution in 
phase space analysis
of Quantum Mechanics. Its Fourier Symplectic Transform, 
the Chord or Weyl Function, 
acts as the generator of momenta of the former 
classical quasi-distribution. It is
assumed that a good approximation for the former is a Dirac delta over
the Classical Energy Surface. This approximation is essentially wrong,
even if it catches some adequate results.
In this work we search for a better uniform approximation, which
should respects the old results in the limit $\hbar\rightarrow 0$.
The essential prerequisite for this seems to be the ergodicity of
the classical system underneath the QM.

\end{abstract}

\section{Preliminary Stuff}

\subsection{Notation}

We will adopt Alfredo Ozorio de Almeida notation throughout this work. 
In particular, a point in phase space has the 
following ordering  $\xfase:=(p,q)$, and its units are $[\sqrt{\hbar}]$.
The number $d$ is the number of degrees of freedom. A capital $N$ shall
be used for indicating the Energy Level in which we are interested.
We shall denote random  variables with a superscript asterisk, e.g.
$\xfase^*$.

\section{A ``Chaotic'' Wigner Function}

The usual assumption is that if we have an ergodic, mixing and hyperbolic
classical Hamiltonian system (henceforth a \emph{chaotic system}), the
Center Function will be a Dirac delta distribution localized over the
energy eigensurface, provided that we study  a large 
energy state \cite{Voros76}:

\begin{equation}
W_N(\xfase)=\delta (H(x)-E_N)
\end{equation}

This corresponds to the notion that ergodic systems tend to 
homogenize themselves and that eigenstates of energy are
stationary states.

This assumption is essentially wrong, as it fails to account
for important quantum features of pure states. An important example would
be the so called ``blind spots'' \cite{Zambrano09} 
of the auto-correlation function.  The autocorrelation can be 
expresed either by the Center or Chord functions:

\begin{equation}
C(\xifase)=(2\pi\hbar)^d \int W(\xfase)W(\xfase-\xifase) d\xfase
=\frac{1}{(2\pi\hbar)^d}\int |\chi(\mufase)|^2 e^{i\mufase\wedge\xifase/\hbar}
d\mufase
\end{equation}

If the Center function, $W(\xfase)$ where really a Dirac delta,
then that Correlation would be zero everywhere except for 
the trivial translation,  $\xifase=0$. This is completely
inadequate for delicate interference effects \cite{Zambrano08}.
The only necessary postulation involved in that
last reference  is that the
State is ``highly excited'' (high value of the eigenenergy) 
and is ``large'' (its Center Function spreads over a significant 
area in
the classical phase space).

\section{The Work}

We shall start with a very crude assumption.
 The Wigner Function can be very roughly approximated
by a random distribution of points near the energy surface. 
The spread of the points should not occupy a volume larger than
$\hbar$, and the volume surrounded by the energy surface
should be from the order of $N\hbar$, $N$ being a number
of the order of the hundreds or thousands.  In this cruddiest
approximation the points will be over a spherical shell of
radius of the order $\sqrt{N\hbar}$. In order to have
a chaotic system we need at least two degrees of freedom, so
this assumption can be stated as follows. Let
$\xfase_1=(p_1,q_1), \xfase_2=(p_2,q_2) $ be the 
symplectic coordinates
of the simplest case ($d=2$);  
\begin{equation}
N\hbar< q_1^2+p_1^2+q_2^2+p_2^2<(N+1)\hbar,
\end{equation}
so that these points are between two three dimensional
spherical shells in a volume of the order of $\hbar$. 
The diference in the radius is of the order of
$(\sqrt{N+1}-\sqrt{N})\hbar=\Delta\hbar$.
If we choose spherical coordinates to represent
the point $\xfase^*=(\xfase_1, \xfase_2)$, and an 
asterisk to represent random variables we get:
\begin{equation}
\begin{split}
p_1&=(R+l^*)\sqrt{\hbar}\cos \phi_1^* \\
q_1&=(R+l^*)\sqrt{\hbar}\sin \phi_1^* \cos\phi_2^*\\
p_2&=(R+l^*)\sqrt{\hbar}\cos \phi_1^*\sin\phi_2^*\cos\phi_3^* \\
q_2&=(R+l^*)\sqrt{\hbar}\cos \phi_1^* \sin\phi_2^*\sin\phi_3^* \\
\end{split}
\end{equation}

$R$ must be the radius
The $l^*$ should be distributed in the interval
$(0, \Delta)$, and the angular variables should 
be taken so as to make the distribution equiprobable in
the volume element
\begin{equation}
dV=r^3 \sin^2\phi_1\sin\phi_2 d\phi_1 d\phi_2 d\phi_3.
\end{equation}

The sample size for the
random points  will be of the order of $5 \cdot 10^5=M$.
Lesser values for the distribution gave variations in
the result.
With all these conventions our Center Function appears
as:

\begin{equation}\label{AproxCruda}
W(\xfase)=\frac{1}{M}\sum_{\xfase^*} \delta (\xfase - \xfase^*)
\end{equation}
The Chord function would then be a random superposition
of plane waves. The wave vectors of these components are
the centers of the Dirac Deltas:
\begin{equation}\label{chordfun}
\chi(\xifase)=\frac{1}{M}\sum_{\xfase^*} \exp \left( \ihb \xifase\wedge\xfase^* \right).
\end{equation}

It is clear that for large enough $M$ this expresion should become
isotropical, i.e., not depend on the direction of $\xifase$ but
only in its magnitude. As $N$ becomes larger, in a
way  that the volume
between the spherical shells becomes negligible, we should recover
the results of Berry \cite{BerryRIR}, namely, a 
Bessel Function only dependent on the magnitude of $\xifase$.
A look on the next pictures (figures \ref{XiFunction01},
\ref{XiFunction02},\ref{XiFunction03},\ref{XiFunction04})
reveal that this may be the case. These pictures where
obtained by following simplification.
We assume that a proper Poincaré or Stroboscopyc map
reduces a 2 degrees of freedom system to a chaotic
$1+\frac{1}{2}$ d. o. f. map, which has its own 
Wigner function, as any mechanical system has. 
So we have reduces the plot of the chords function
to a two dimensional $\xifase$ space. The original 
Wigner function looks like the figure
\ref{WigExample}.


\begin{figure}
\begin{center}
  \includegraphics[width=0.6\textwidth]{CondiniExemplo01.png}
\caption{ A realization of the Wigner
function in eq. \ref{AproxCruda}}\label{WigExample}
\end{center}
\end{figure}

\begin{figure}
\begin{center}
\subfigure[Chord Function, Real part]{
\includegraphics[width=0.4\textwidth]{300_Chords_Real.png}
\label{xireal01}
}
\subfigure[Chord Function, Imaginary part]{
\includegraphics[width=0.4\textwidth]{300_Chords_Imag.png}
\label{xiimag01}
}
\subfigure[Chord Function, Amplitude]{
\includegraphics[width=0.4\textwidth]{300_Chords_Ampli.png}
\label{xiampl01}
}
\subfigure[Chord Function, Phase]{
\includegraphics[width=0.4\textwidth]{300_Chords_Fase.png}
\label{xifase01}
}
\caption{The Chord Function for the random distribution of deltas. 
Here the energy is $300$ arbitrary units. Note the difference between
the scales in the real and imaginary part. The latter has a structure much
closer to zero, while the former is highly (and suspiciously) regular.}
\label{XiFunction01}
\end{center}
\end{figure}


\begin{figure}
\begin{center}
\subfigure[Chord Function, Real part]{
\includegraphics[width=0.4\textwidth]{500_Chords_Real.png}
\label{xireal02}
}
\subfigure[Chord Function, Imaginary part]{
\includegraphics[width=0.4\textwidth]{500_Chords_Imag.png}
\label{xiimag02}
}
\subfigure[Chord Function, Amplitude]{
\includegraphics[width=0.4\textwidth]{500_Chords_Ampli.png}
\label{xiampl02}
}
\subfigure[Chord Function, Phase]{
\includegraphics[width=0.4\textwidth]{500_Chords_Fase.png}
\label{xifase02}
}
\caption{The Chord Function for the random distribution of deltas. 
Here the energy is $300$ arbitrary units. Note the difference between
the scales in the real and imaginary part. The latter has a structure much
closer to zero, while the former is highly (and suspiciously) regular.}
\label{XiFunction02}
\end{center}
\end{figure}




\begin{figure}
\begin{center}
\subfigure[Chord Function, Real part]{
\includegraphics[width=0.4\textwidth]{750_Chords_Real.png}
\label{xireal03}
}
\subfigure[Chord Function, Imaginary part]{
\includegraphics[width=0.4\textwidth]{750_Chords_Imag.png}
\label{xiimag03}
}
\subfigure[Chord Function, Amplitude]{
\includegraphics[width=0.4\textwidth]{750_Chords_Ampli.png}
\label{xiampl03}
}
\subfigure[Chord Function, Phase]{
\includegraphics[width=0.4\textwidth]{750_Chords_Fase.png}
\label{xifase03}
}
\caption{The Chord Function for the random distribution of deltas. 
Here the energy is $7500$ arbitrary units. Note the difference between
the scales in the real and imaginary part. The latter has a structure much
closer to zero, while the former is highly (and suspiciously) regular.}
\label{XiFunction03}
\end{center}
\end{figure}




\begin{figure}
\begin{center}
\subfigure[Chord Function, Real part]{
\includegraphics[width=0.4\textwidth]{1000_Chords_Real.png}
\label{xireal04}
}
\subfigure[Chord Function, Imaginary part]{
\includegraphics[width=0.4\textwidth]{1000_Chords_Imag.png}
\label{xiimag04}
}
\subfigure[Chord Function, Amplitude]{
\includegraphics[width=0.4\textwidth]{1000_Chords_Ampli.png}
\label{xiampl04}
}
\subfigure[Chord Function, Phase]{
\includegraphics[width=0.4\textwidth]{1000_Chords_Fase.png}
\label{xifase04}
}
\caption{The Chord Function for the random distribution of deltas. 
Here the energy is $1000$ arbitrary units. Note the difference between
the scales in the real and imaginary part. The latter has a structure much
closer to zero, while the former is highly (and suspiciously) regular.}
\label{XiFunction04}
\end{center}
\end{figure}

\section{Some horrible mathematics and geometry}.

We will try to make a semi-analytical tackle to obtain a good
comprehension of this chord function. 
Then we can easily distribute points uniformly between the 
$N$th and $N=1$th energy shell, using Marsaglia's Algorithm \cite{Mar72}.
We simply choose $2d$ random real numbers, divide the resulting
vector by its euclidian magnitude, multiply it by 
$\sqrt[2d]{(u^*)\hbar^d}\Delta$
and sum $\sqrt{(N)\hbar}$, where $u^* \in [0,1)$. 
For $d=1$ we obtain something
 which looks like figure \ref{WigExample}. 

For every element in the sum of the chord function, 
eq. \ref{chordfun}, we have a random wave whose argument
is the symplectic area delimited by the paralelogram 
with sides $\xfase^*, \xifase$. This is just the sum
of the directed areas of the paralelograms defined
by each degree of freedom. Then we can approximate an
argument of symmetry: given that the $\xfase^*$ is
randomly distributed with no preference for
direction whatsoever, the odd part of the wave
(the imaginary component) goes to zero, because
it is allmost a center-symmetric state \cite{Zambrano09},
and the figures \ref{XiFunction01} to \ref{XiFunction04}
point in that direction. Notice the different scales
between the real and imaginary parts. 

That said, the real and even part of each summand for the
chord function is of the form:

\begin{equation}
\cos(\xifase_1\wedge\xfase_1^*/\hbar)\cos(\xifase_2\wedge\xfase_2^*/\hbar)
\end{equation}

Either one of the functions oscilates very fast, or the other,
or both of them oscilate at more or less the same frequency. 
A good way to deal this is to use Hopf fibration coordinates
\cite{ArnoldDiff, Meyer}. In $d=2$ they would be:

\begin{equation}\label{TorSphere}
\begin{split}
p_1&=r \cos\psi \cos \alpha_1\\
q_1&=r \cos\psi \sin \alpha_1\\
p_2&=r \sin\psi \cos \alpha_2\\
q_2&=r \sin\psi \sin \alpha_2
\end{split}
\end{equation}

In such manner $r\cos\psi= \|\xfase_1\|$ and $r\sin\psi= \|\xfase_2\|$.
The function is symmetric in the indices $1$ and $2$, so we can just
take the contribution from $\psi \in [0, \pi)$ and multiply it by two.

Our first take then, will be to approximate the sum  by an integral,
assuming $M$ to be very large (which it is).

\begin{equation}
\chi(\xifase)=\frac{1}{M}\sum_{\xfase^*} \exp \left( \ihb \xifase\wedge\xfase^* \right)
\rightarrow\int dV(\xfase) \cos \left( \xifase_1\wedge\xfase_1 \right/\hbar)
\cos \left( \xifase_2\wedge\xfase_2 \right/\hbar)
\end{equation}

Here the cumbersome part is to calculate the volume element $dV(\xfase)$. 
The Jacobian of the parametrization in eqs. \ref{TorSphere} is
as follows:

\begin{equation}
\begin{split}
Jac&=
\left|
\begin{array}{cccc}
\cos\psi\cos\alpha_1 & -r\sin\psi\cos\alpha_1 & r\cos\psi\sin\alpha_1 & 0 \\
\cos\psi\sin\alpha_1 & -r\sin\psi\sin\alpha_1 & r\cos\psi\cos\alpha_1 & 0 \\
\sin\psi\cos\alpha_1 & r\cos\psi\cos\alpha_1 & 0 & -r\sin\psi\sin\alpha_2 \\
\sin\psi\sin\alpha_1 & r\cos\psi\sin\alpha_1 & 0 & r\sin\psi\cos\alpha_2 
\end{array}
\right|\\
&=r^3 \sin^3\psi\cos\psi
(\cos^2\alpha_1\sin^2\alpha_1 +\cos^2\alpha_2\sin^2\alpha_2)
\end{split}  
\end{equation}


But there is a problem using this system of coordinates.
They induce a separability in the index $1$ and $2$ set
of coordinates: the energy will be either concentrated
in one or the other degree of freedom, as with two 
weakly or uncoupled harmonic oscillators. In a strong
chaotic system this is a unjustified discrete symmetry. 
Energy can  be split among different coordinates
without regards of the degree of freedom they belong, that
is, can be concentrated evenly among the momenta, without
leaving anything in the positions or viceversa.
In order not to fall in these, we shall pick cleverly 
our integration limits. 



The integral comes out easily in the cartesian system:
\begin{multline}
\int dp_1 dq_1 dp_2 dq_2 \cos \left( \xifase_1\wedge\xfase_1 \right/\hbar)
\cos \left( \xifase_2\wedge\xfase_2 \right/\hbar)\\
=\frac{\hbar^4}{\xi_{1,p}\xi_{1,q}\xi_{2,p}\xi_{2,q}}
\cos \left( \xifase_1\wedge\xfase_1 \right/\hbar)
\cos \left( \xifase_2\wedge\xfase_2 \right/\hbar),
but in order to evaluate we should use either the 
3-spherical or the toroidal-spherical systems.


\end{multline}

\bibliography{ziegos}

\end{document}
