\documentclass[a4paper,12pt]{article}

\usepackage[utf8]{inputenc}
\usepackage{amsmath,amssymb}
\usepackage{graphicx}
\usepackage{subfigure}
%\usepackage[spanish]{babel}
\usepackage{bm}

\usepackage[cm]{fullpage}
\usepackage[light]{antpolt}
\usepackage[T1]{fontenc}


\bibliographystyle{alpha}

\newcommand{\ihb}{\frac{i}{\hbar}}
\newcommand{\xfase}{\mathbf{x}}
\newcommand{\yfase}{\mathbf{y}}
\newcommand{\qfase}{\mathbf{q}}
\newcommand{\pfase}{\mathbf{p}}
\newcommand{\xifase}{ {\boldsymbol{\xi}} }
\newcommand{\mufase}{ {\boldsymbol{\mu}} }
\newcommand{\Ifase}{\mathbf{I}}
\newcommand{\Pfase}{\mathbf{P}}
\newcommand{\Scat}{\mathbf{S}}
\newcommand{\Jsimp}{\mathbf{J}}
\newcommand{\KovM}{\mathbf{K}}
\newcommand{\Dom}{\mathbb{D}}
\newcommand{\Var}{\mathbb{M}}
\newcommand{\bra}[1]{\langle #1|}
\newcommand{\ket}[1]{|#1\rangle}
\newcommand{\braket}[2]{\langle #1|#2\rangle}
\newcommand{\Prom}[2]{\langle #1\rangle_{#2}}


\DeclareMathOperator*{\cod}{cod}
\DeclareMathOperator*{\traza}{traza}


\title{The Cumulants in Semiclassical Representations of a 2 d.o.f. System.}
\author{\\CBPF}


\begin{document}

\maketitle

\begin{abstract}

Here I present the expansion of the chord function for
a 2 degrees of freedom hamiltonian system. The expresion
has a lot of confortable symmetries, which makes them
quite suitable for quick implementation in numerical simulation
that take advantage of linear algebra libraries. 

\end{abstract}

\section{Preliminary Stuff}

\subsection{Notation}

We shall adopt Alfredo Ozorio de Almeida notation throughout this work. 
In particular, a point in 2 d.o.f. phase space has the 
following ordering  $(\xfase,\yfase):=(x_p,x_q,y_p,y_q)$.
The Fourier transformed variables (chords) will be $\xifase, \mufase$,
same ordering. We will identify the Center space with the usual
classical phase space and with the values of the corresponding
quantum operators. 



\section{The 2 d.o.f. Chord Function}


The Chord Function is the expected value of the 
translation operator for the additive inverse of each chord.
In our case this is 

\begin{equation}
\chi(\xifase,\mufase)=\langle T_{-\xifase,-\mufase}\rangle.
\end{equation}

As it turns out, this is simply the Symplectic Fourier Transform of
the Center Function.

\begin{equation}
\chi(\xifase,\mufase)=\int d\xfase d \yfase W(\xfase,\yfase)
\exp\left(\ihb (\xfase\wedge \xifase +\mufase\wedge\yfase)\right).
\end{equation}

Our Center Function (The Wigner quasi-distribution) is a random
cloud of points between two neighbouring energy surfaces, evenly
distributed. The Chord function is also the
a Characteristic Function in the statistical sense, so the
expansion of it contains the information of the geometry of
the Weyl quasi-distribution. In a power expansion around zero,
its coefficients are the cumulants of said
pseudodistribution. The expansion, up to third
order, looks as follows:

\begin{equation}
\chi(\xifase,\mufase)\approx 
\langle 1-\ihb (\xifase\wedge\xfase+\mufase\wedge\yfase)
+\left(\ihb\right)^2(\xifase\wedge\xfase+\mufase\wedge\yfase)^2
-\left(\ihb\right)^3(\xifase\wedge\xfase+\mufase\wedge\yfase)^3
+\cdots\rangle
\end{equation}

The first order term coefficients are simply the average values for
the cloud of points in center space. The second order terms can
identified with the Schr\"odinger Covariancie Matrix, after
taking the expected value. We will abuse the notation a little
and denote by $\KovM$ the matrix of the operators and by
$\langle \KovM \rangle$ the covariancie matrix.

\begin{equation}
\KovM=\left( 
\begin{array}{cccc}
x_p^2   & -x_qx_p & x_qy_q & -x_qy_p \\
-x_px_q & x_q^2 & -x_py_q&  x_py_p \\
x_qy_q & x_py_q & y_p^2    & -y_py_p  \\
-x_qy_q & x_py_q & y_py_q  & y_q^2
 \end{array} 
\right)
\end{equation}

Notice that in the diagonal two by two blocks the individual
operators wich would give
the covarance matrices for $\xfase$ and $\yfase$ are present. We
will denote them $K_x, K_y$. 

The third order terms are not as ugly as they could be. As I will
later identify the expected value of well ordered permutations
of the operators with classical expected values, I will ignore
the different permutations. Still abusing the notation, it 
will look like this:

\begin{equation}
\begin{split}
(\xifase\wedge\xfase+\mufase\wedge\yfase)^3 &=
\xi_p^3x_q^3-3\xi_p^2x_q^2\xi_qx_p +3\xi_q^2x_p^2\xi_px_q-\xi_q^3x_p^3\\
&+\mu_p^3y_q^3-3\mu_p^2y_q^2\mu_qy_p +3\mu_q^2y_p^2\mu_py_q-\mu_q^3y_p^3\\
&+(\xifase K_x\xifase) (\mufase\wedge \yfase)\\
&+(\mufase K_y\mufase) (\xifase\wedge \xfase)
\end{split}
\end{equation}

For $\mu=(0,0)$, only the first line matters. For $\xifase=(0,0)$ only
the second one. 

The only thing which makes this expression cumbersome to implement
are the signs. 

\bibliography{ziegos}

\end{document}
