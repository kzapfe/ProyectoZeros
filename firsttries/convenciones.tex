\documentclass[a4paper,10pt]{article}

\usepackage[utf8]{inputenc}
\usepackage{amsmath,amssymb}
\usepackage{graphicx}
\usepackage{subfigure}
\usepackage[spanish]{babel}
\usepackage{bm}

\bibliographystyle{alpha}


\newcommand{\xfase}{\mathbf{x}}
\newcommand{\yfase}{\mathbf{y}}
\newcommand{\qfase}{\mathbf{q}}
\newcommand{\pfase}{\mathbf{p}}
\newcommand{\xifase}{ {\boldsymbol{\xi}} }
\newcommand{\mufase}{ {\boldsymbol{\mu}} }
\newcommand{\Ifase}{\mathbf{I}}
\newcommand{\Pfase}{\mathbf{P}}
\newcommand{\Scat}{\mathbf{S}}
\newcommand{\Jsimp}{\mathbf{J}}
\newcommand{\Dom}{\mathbb{D}}
\newcommand{\Var}{\mathbb{M}}
\newcommand{\bra}[1]{\langle #1|}
\newcommand{\ket}[1]{|#1\rangle}
\newcommand{\braket}[2]{\langle #1|#2\rangle}




\DeclareMathOperator*{\cod}{cod}
\DeclareMathOperator*{\traza}{traza}


\title{Convenciones y notas varias}
\author{ W. P. Karel Zapfe\\CBPF}


\begin{document}

\maketitle

\begin{abstract}

Pongamos en claro todas las convenciones de notación, signo
y unidades para el trabajo en semiclásica.

\end{abstract}

\section{Unidades}

En este trabajo las posiciones y los momentos cartesianos
tienen todos unidades de $\sqrt{\hbar}$, de forma
que la métrica en un plano fase sea homogenea. Las unidades
de raiz de acción corresponden a las siguientes combinaciónes:

\begin{equation}
\begin{split}
U[\sqrt \hbar] &= U[\sqrt L] \\
&=U[l \sqrt{m/t} ] \\
&=U[\sqrt{E t}]
\end{split}
\end{equation}

Nota que para tus estados coherentes y tus operadores
estás usando unidades tales que $m, \omega =1$. 

\section{Espacio Fase Clásico}

Usaremos la convención de Alfredo O. de A. para el espacio fase,
es decir, $\xfase=(p,q)$.  Siguiendo esa convención la matriz simpléctica
$\Jsimp$ queda dada por:
\begin{equation}
\Jsimp=\left( \begin{array}{cc}
0 & -1 \\
1 & 0
\end{array} \right)
\end{equation}
Entonces el producto antisimétrico o simpléctico quedará dado por:
\begin{equation}
\xifase\wedge\xfase:=\xifase_p \xfase_q -\xifase_q \xfase_p.
\end{equation}


\section{Transformada de Fourier}
La convención para la transformada de Fourier que he utilizado es
no normalizada, y la frecuencia positiva es en contra de las manecillas
del reloj.
\begin{equation}
F(k)=\int dx f(x)e^{-ikx}
\end{equation}
Con esta convención, la relación entre la representación de posición
y momento queda asi:
\begin{equation}
\phi(p)=\frac{1}{\sqrt{2\pi\hbar}^n} \int dq \psi(q)e^{-i\frac{pq}{\hbar}}.
\end{equation}
dado asi que $k=p/\hbar$. La relación inversa es:
\begin{equation}
\psi(q)=\frac{1}{\sqrt{2\pi\hbar}^n} \int dp \phi(q)e^{+i\frac{pq}{\hbar}}.
\end{equation}

Esta es la convención del Messiah \cite{Messiah}.
En esta convención, la transformada de una Gaussiana fuera del origen es así:
\begin{equation}
\begin{split}
 f(x)&=\frac{\exp(-(x-\mu)^2/\sigma^2)}
 {\sigma\sqrt \pi}\\
 \hat{f}(k)&=\exp(-(k\sigma)^2/4)\exp(i\mu k)
\end{split}
\end{equation}


\section{Transformada de Wigner y de Weyl}

Si tenemos un estado en la representación de posición o de momento, 
su representación
como función de Wigner queda definida por
\begin{equation}
\begin{split}
\pi \hbar W(\xfase) &=\int dy \psi^*(q+y)\psi(q-y)e^{2ipy/\hbar}\\
&=\int dy \phi^*(p+y)\phi(p-y)e^{-2iqy/\hbar}
\end{split}
\end{equation}.

Más abstractamente esto es:
\begin{equation}
W(\xfase)= \frac{1}{\pi\hbar}\int d \xifase 
\bra{\xfase+\xifase} \hat{\rho} \ket{\xfase-\xifase}
\exp(2i/\hbar \xfase\wedge\xifase) 
\end{equation}

Aún más abstractamente esto es:
\begin{equation}
W(\xfase)= \frac{1}{\pi\hbar}\bra{\psi}\hat{R_\xfase}\ket{\psi}
=\traza \hat{R_\xfase}\hat{\rho}
\end{equation}
Donde $\hat{R}_\xfase$ es el operador de reflexión
respecto al punto $\xfase$ del espacio fase clásico.

La transformada de Weyl invierte la transformada
de Wigner, mandando una pseudo distribución en el
espacio fase a un operador cuántico en la representación 
de posiciones:
\begin{equation}
\bra{x}\hat{G}\ket{y}=\frac{2 \pi}{\hbar} \int 
g\left(\frac{x+y}{2},p\right) \exp(i(x-y)p/\hbar) dp .
\end{equation}

\section{Algebra de Reflexión y Traslación}

Los operadores cuánticos de reflexión y traslación forman un
álgebra muy divertida. Aquí las propiedades más importantes.

\subsection{Convenciones}
\begin{equation}
 \begin{split}
\hat{T}_\xifase &:= \exp(i/\hbar (\xifase\wedge\hat{\xfase}))\\
\hat{R}_\xfase &:= \int \frac{d\xifase}{(4 \pi \hbar)^2}
\exp((i/\hbar) \xfase \wedge \xifase)\hat{T}_\xifase
 \end{split}
\end{equation}
 El operador de reflexión se define como la transformada
de Fourier Antisimétrica (entre $2^d$) del operador
de traslación.

\subsection{Algebra Elemental}

Las reglas más simples de éste ál-gebar.

\begin{equation}
\begin{split}
\hat{T}_\xifase^{-1}&=\hat{T}_\xifase^\dagger=\hat{T}_{-\xifase}\\
\hat{R}_\xfase^{-1}&=\hat{R}_\xfase^\dagger=\hat{R}_{-\xfase}\\
\hat{T}_\xifase\hat{T}_\mufase &=e^{i/\hbar(\xifase\wedge\mufase)} 
\hat{T}_{\xifase+\mufase}\\
\hat{R}_\xfase \hat{R}_\yfase &=e^{-2i/\hbar(\xfase\wedge\yfase)} 
\hat{T}_{2(\xfase-\yfase)}\\
\hat{T}_\xifase\hat{R}_\xfase &=e^{-i/\hbar(\xfase\wedge\xifase)} 
\hat{R}_{\xfase+\xifase/2}\\
 \hat{R}_\xfase\hat{T}_\xifase &=e^{-i/\hbar(\xfase\wedge\xifase)} 
\hat{R}_{\xfase-\xifase/2}\\
\end{split}
\end{equation}

\subsection{Emparedados}

\begin{equation}
\begin{split}
\hat{R}_\xfase \hat{T}_\xifase \hat{R}_\xfase 
&=e^{-2i/\hbar(\xfase\wedge\xifase)} \hat{T}_{-\xifase}\\
\hat{T}_{-\xifase}\hat{R}_\xfase \hat{T}_\xifase &=
\hat{R}_{\xfase-\xifase}\\
 \hat{R}_{\xfase}\hat{R}_\yfase \hat{R}_\xfase &=
\hat{R}_{2\xfase-\yfase}\\
 \hat{T}_{-\xifase}\hat{T}_\mufase \hat{T}_\xifase &=
e^{2i/\hbar (\mufase\wedge\xifase)}\hat{T}_{\mufase}
\end{split}
\end{equation}

\subsection{Conmutadores}


\begin{equation}
\begin{split}
[\hat{R}_\xfase ,\hat{R}_\yfase ]&=
[\hat{T}_{2\yfase} ,\hat{T}_{-2\xfase} ],\\
[\hat{R}_\xfase ,\hat{T}_\xifase ]&=
e^{-i/\hbar(\xfase\wedge\xifase)}
(\hat{R}_{\xfase-\xifase/2}-\hat{R}_{\xfase+\xifase/2}).
\end{split}
\end{equation}


\section{Estado Coherente y Desplazamientos}

Un estado coherente es un estado de mínima incertidumbre 
que se mantiene coherente en el tiempo. En la aproximación
armónica esto es un estado base desplazado fuera del punto de
equilibrio. Estos son \emph{estados gaussianos}o 
\emph{ coherentes canónicos}. En la literatura convencional
se usa el operador de desplazamiento

\begin{equation}
\hat{D}_z:= \exp(z\hat{a}^\dagger-z^*\hat{a}),
\end{equation}

donde $\hat{a}, \hat{a}^\dagger$ son operadores de bajar y subir,
en unidades de $\sqrt{\hbar}$, 
y $z$ es un número complejo con unidades de $\sqrt\hbar^{-1}$.
El estado coherente queda entonces definido por:

\begin{equation}
\ket{z}=\hat{D}_z \ket{0}
\end{equation}
y es un eigenket del operador de bajada:
\begin{equation}
\hat{a}\ket{z}=z \ket{z}.
\end{equation}

En la convención de Alfredo nos conviene usar mejor el 
operador de traslación:

\begin{equation}
\hat{T}_\xifase := \exp(i/\hbar (\xifase_p\hat{q}-\xifase_q\hat{p}))
=\exp(i/\hbar (\xifase\wedge\hat{\xfase})),
\end{equation}

Aquí $\xifase$ es miembro del espacio de cuerdas y tiene las 
mismas unidades que el espacio fase, asi que el producto
simpléctico tiene unidades de $\hbar$. La relación entre
el operador de desplazamiento y el de traslación está dada si esogemos
$z=1/(2\hbar)(\xifase_q + i \xifase_p)$. En realidad no te queda tan claro
de donde sale el $2$, pero eso es lo que ocurre.
Siguendo esa notación y mandando a volar la dependencia temporal, 
tenemos que: 

\begin{equation}
\bra{q}\hat{T}_\xifase\ket{0}=\frac{1}{(\pi\hbar)^{1/4}}
\exp\left(\frac{-(q-\xifase_q)^2}{\hbar}+
\frac{i\xifase_p (q-\xifase_q/2)}{\hbar}\right).
\end{equation}

Observa la fase (semi irrelevante) $\xifase_p \xifase_q/2\hbar$.
Simplifica mucho los posteriores cálculos.

Su representación en momento es la transformada de Fourier usual.

\begin{equation}
\bra{p}\hat{T}_\xifase\ket{0}=\frac{1}{(4\pi\hbar)^{1/4}}
\exp\left(\frac{-(p-\xifase_p)^2}{4\hbar}-
\frac{i(p-\xifase_p/2)\xifase_q}{\hbar}\right).
\end{equation}

Observese el cambió de signo en la fase. 

Siguiendo esa convención, la función de Wigner para el estado
coherente fuera del origen queda de la siguiente forma:

\begin{equation}
\begin{split}
W_\xifase(\xfase)&=\frac{1}{\sqrt{\pi\hbar}}
  \exp\left(\frac{-1}{\hbar}
\left((q-\xifase_q)^2+(p-\xifase_p)^2\right)\right)\\
&=\frac{1}{\sqrt{\pi\hbar}}
\exp\left(\frac{-1}{\hbar}
\|\xfase-\xifase\|^2\right).
\end{split}
\end{equation}

Ahora la función de cuerdas o de Weyl: ésta esta dada por la 
transformada de Fourier de centros a cuerdas:
\begin{equation}
\chi(\xifase)=\frac{1}{2\pi\hbar}\int d \xfase W(\xfase)
\exp\left(\frac{i}{ \hbar}(\xifase\wedge\xfase)\right).
\end{equation}

La función de cuerdas para un estado coherente se ve así, entonces:
 

\begin{equation}
\begin{split}
\chi(\xifase) & = \frac{1}{2\pi\hbar}\int d \xfase W_\mufase(\xfase)
\exp\left(\frac{i}{\hbar}(\xifase\wedge\xfase)\right)\\
 & = \frac{1}{2\sqrt{\pi\hbar}}
  \exp\left(\frac{1}{\hbar}(i\mufase\wedge\xifase - \|\xifase\|^2/4)\right)
\end{split}
\end{equation}

Esos 2 se están volviendo extraños. Pero no grites hasta no saber.

\section{Superposición de estados coherentes}

La superposición de estados coherentes está dada en la representación
de momentos o posiciones simplemente por la suma compleja de sus
varias partes, con su amplitud correspondiente.


\begin{equation}
\ket{\psi}=\sum_i a_i \ket{\xifase_i}
\end{equation}

Por supuestoq que esto está muy fácil. Para ejemplos consideremos
el caso de dos estados sumados.
La matriz de densidad será en este caso:

\begin{equation}
\hat{\rho}=\sum_{i,j} a_i a_j^* \ket{\xifase_i}\bra{\xifase_j}
\end{equation}

Ahora bien: usando un poco el coco se puede ver que
la función de Wigner es real (como tiene que ser, por el eigenvalor
de $\hat{R}_\xfase$).
La transformada de Wigner es lineal. Entonces se pueden separar
las partes de la suma.  Los elementos diagonales son simplemente
las funciones de Wigner centradas en los $\xifase_i$, por el 
peso correspondiente:

\begin{equation}
  \sqrt{\pi\hbar}W_{ii}(\xfase)=|a_i|^2 W_{\xifase_i}(\xfase).
\end{equation}

Cada elemento no diagonal es como sigue:

\begin{equation}\label{offdiagonal0}
 \sqrt{\pi\hbar} W_{ij}(\xfase)=a_i a_j^* \int dy \psi_{\xifase_j}^*(q+y)
\psi_{\xifase_i}(q-y)e^{2ipy/\hbar}
\end{equation}

Siguiendo a E. Zambrano \cite{tesiseduardo},
la parte no diagonal está dada por (\emph{after a little algebra}):
Se entiende cuando $i$ es un índice y cuando no, verdad?

\begin{equation}\label{offdiagonal1}
 \sqrt{\pi\hbar} W_{ij}(\xfase)=a_i a_j^*\exp\left(
\frac{\|(\xifase_i+\xifase_j)/2-\xfase\|^2}{\hbar}+
\frac{i}{\hbar}(\xifase_i-\xifase_j)\wedge \xfase-
\frac{i}{2\hbar}\xifase_i\wedge \xifase_k
\right)
\end{equation}

La función de cuerdas también la podemos tomar de Zambrano, 
corrigiendo los factores. Igualmente la parte diagonal
no tiene problemas, y la parte no diagonal está dada por:

\begin{equation}\label{chioffdiagonal1}
 2\sqrt{\pi\hbar} \chi_{ij}(\xifase)=a_i a_j^*\exp\left(
\frac{\|(\xifase_i-\xifase_j)-\xfase\|^2}{4\hbar}+
\frac{i}{2\hbar}[(\xifase_i+\xifase_j)\wedge \xifase+
\xifase_i\wedge \xifase_k]
\right)
\end{equation}

Una vez más, atento al cambio de signo al final.

\end{document}


