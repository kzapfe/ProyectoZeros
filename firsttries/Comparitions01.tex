\documentclass[a4paper,12pt]{article}

\usepackage[utf8]{inputenc}
\usepackage{amsmath,amssymb}
\usepackage{graphicx}
\usepackage{subfigure}
%\usepackage[spanish]{babel}
\usepackage{bm}

\usepackage[cm]{fullpage}
\usepackage[light]{antpolt}
\usepackage[T1]{fontenc}

%Este paquete le pone una barrera a los floats
% al final de cada seccion
\usepackage[section]{placeins}


\bibliographystyle{alpha}

\newcommand{\ihb}{\frac{i}{\hbar}}
\newcommand{\xfase}{\mathbf{x}}
\newcommand{\yfase}{\mathbf{y}}
\newcommand{\qfase}{\mathbf{q}}
\newcommand{\pfase}{\mathbf{p}}
\newcommand{\xifase}{ {\boldsymbol{\xi}} }
\newcommand{\mufase}{ {\boldsymbol{\mu}} }
\newcommand{\Ifase}{\mathbf{I}}
\newcommand{\Pfase}{\mathbf{P}}
\newcommand{\Scat}{\mathbf{S}}
\newcommand{\Jsimp}{\mathbf{J}}
\newcommand{\Dom}{\mathbb{D}}
\newcommand{\Var}{\mathbb{M}}
\newcommand{\bra}[1]{\langle #1|}
\newcommand{\ket}[1]{|#1\rangle}
\newcommand{\braket}[2]{\langle #1|#2\rangle}
\newcommand{\Prom}[2]{\langle #1\rangle_{#2}}


\DeclareMathOperator*{\cod}{cod}
\DeclareMathOperator*{\traza}{traza}


\title{ Weyl Functions for the Nelson Potential:\\ some numerical computations.}
\author{\\CBPF}


\begin{document}

\maketitle

\begin{abstract}
Here I present various different computations for the Weyl or Chord 
function of the Nelson Potential Problem. The values of Energy
and Planck's constant are equal to  those used on
Toscanos et all \cite{Toscano01}.
\end{abstract}

\section{Classical Chaos}

The Nelson potential problem is part of a a family of problems which
have certain applications in nuclear physics. The Hamiltonian function 
for the problem is:

\begin{equation}
H(\xfase,\yfase)=(p_x^2+p_y^2)/2+\omega_1 q_x^2+
\omega_2(q_y-q_x^2/2)^2
\end{equation}

We choose $\omega_1=0.05$ and $\omega_2=1$, and explore the volume
between the Energy surfaces corresponding to $E=0.821$ 
and $E=0.826$. We will denote this volume by $V$. The system
appears to be ``visually''  ergodic, meaning that any non-hyperbolic
subsets are of inappreciable size in relation to the whole
energy hypersurface in phase space for the energies chosen. 
A sample of the trajectories in physical space is shown below, in
figure \ref{ClassicalTrayec}. The corresponding Poincaré 
Section is shown in figure \ref{ClassicalPoin}. 



\begin{figure}
\begin{center}
  \includegraphics[width=0.8\textwidth]{SampleTrayectories01.png}
\caption{Classical Trajectories in position space. The trajectories
begin to fill up the space delimited by the equipotential curve 
for $E=0.821$, indicating at least some form of chaos. For this sample
all trajectories have starting point in the symmetry axis $q_x=0$,
which we also use for the Poincare Section below.}
\label{ClassicalTrayec}
\end{center}
\end{figure}




\begin{figure}
\begin{center}
  \includegraphics[width=0.8\textwidth]{PoincareSample01.png}
\caption{The first two iterations of a straight line of initial conditions
in the Poincaré Map of the classical system.  The fold-and-stretch
effect is extreme for these parameters. Black is the original line, red
the first iteration, yellow the second.}
\label{ClassicalPoin}
\end{center}
\end{figure}

 
\section{Quantum Ergodic Hypothesis}

The Quantum Ergodic Hypothesis  states that, intuitively, a representation
of a quantum state for a chaotic system would cover more or less uniformly
the Energy Surface. The most extreme version of this statement is
that the Wigner Function of the state would be a Dirac Delta distribution
over exactly over the Energy Surface \cite{BerryRIR}. However we know
that this could not be true on the general case \cite{Ozorio98}. We know
that a lot of chaotic systems show ``scars'' in the position
representation eigenstates \cite{haake}.
 
As a compromise we try various calculations for a ``simulated'' Wigner
function, and obtain its symplectic Fourier Transform, the
Weyl function. 


\section{Almost Ergodic, brute force approach.}

I shall present two solutions for the ``all-most ergodic'' quantum
hypothesis. This is the first one. It is a compromise between the 
perfectly uniform distribution on the energy surface and the
necessity for correlations between different eigenstates. Some very general
requirements lead to the existence of isolated zeros in the
Weyl function representation of a large quantum state \cite{Zambrano09}.
To represent this ``large'' quantum state we sample a big number 
points in the phase
space between two energy surfaces and suppose the Wigner Function
is absolutely concentrated in those points, that is, it is a sum
of Dirac Deltas over a small energy range. We shall call it
the grained approach. The energy interval is
$E\in [0.821,0.836]$ in agreement with \cite{Toscano01}. The lower
bound corresponds roughly to the 11th Bohr-quantized eigenstate. 
The following set of pictures (figures \ref{WeylFunctioncomplete} and 
\ref{WeylNodalcomplete}) show the Weyl Function obtained after this
distributions.


\begin{figure}
\begin{center}
\subfigure[Real Part]{\includegraphics[width=0.45\textwidth]
{EFabrizio_0821_WeylMedio-Real.png}
\label{RealWeyl}}
\subfigure[Imaginary Part]{\includegraphics[width=0.45\textwidth]
{EFabrizio_0821_WeylMedio-Imag.png}
\label{ImagWeyl}}
\caption{Real and Imaginary parts for the Weyl function, the section
corresponds to $\xifase=(0,0)$. The color range has been reduced in
order to enhance contrast. After $\|\mufase\|\approx 5$ the structure
begins to appear too noisy.}
\label{WeylFunctioncomplete}
\end{center}
\end{figure}



\begin{figure}
\begin{center}
  \includegraphics[width=0.8\textwidth]
                  {EFabrizio_0821_Weylmedio0-0-Countour.png} %
\caption{The nodal lines for the function depicted in the previous figure 
(\ref{WeylFunctioncomplete}). The intersection of both red and black lines is
a true blind spot, a chord such that a state and its displacement are orthogonal.}
\label{WeylNodalcomplete}
\end{center}
\end{figure}




\section{Almost Ergodic, Cumulant approach}

The expansion of the Weyl function representation for  
small chords reveals that the function depends on the cumulants of
the Wigner Function, understood as a quasi-probability. In this
approach it is clear that this is a representation of the geometric
features of the eigenstate in a phase space representation. 
We calculated numerically the first three cumulants for the grained
approach and calculated the Weyl function approximation. Although
it gives a nice qualitative idea of the first nodal lines, it is only
``topologically'' accurate. The calculation is much faster than the 
brute approach, anyway, and it could serve to weakly locate the 
nodal hypersurfaces in a 4 or 6 dimensional problem. The only inconvenience
of this approach is that the higher the moment of the cumulant, the more
cumbersome is to deal with it. In 2 degrees of freedom the third
cumulant is a third degree tensor with a very anti-intuitive 
symplectic symmetry. In the following equation $V$ represents the
volume between the energy eigensurfaces.  
The result is shown in figure \ref{CrudaAprox}


\begin{equation}\label{expansion}
\chi(\xifase,\mufase)\approx 
1-\ihb (\xifase,\mufase) \wedge \Prom{\hat{\xfase},\hat{\yfase}}{V}
-\frac{1}{\hbar^2}
\Prom{((\xifase,\mufase)\wedge(\hat{\xfase},\hat{\yfase}))^2}{V}
+\frac{i}{\hbar^3}
\Prom{((\xifase,\mufase)\wedge(\hat{\xfase},\hat{\yfase}))^3}{V}+
\cdots
\end{equation}



\begin{figure}
\begin{center}
  \includegraphics[width=0.8\textwidth]
{SFabrizio_0821_WeylAprox3grado-0-0-Countour.pdf}
\caption{The nodal lines obtained by taking the expansion in eq. 
\ref{expansion} until the third degree term. Notice the difference
in the vertical scale with previous figures. Only in the crudest
topological approximation can this be considered convenient. At least,
it respects the adequate symmetries and the size is approximately correct,
but the effort of obtaining the tensorial expression for a 2 d.o.f. system
is to big without the help of an algebraic software and very careful 
programming.}
\label{CrudaAprox}
\end{center}
\end{figure}



\section{The Scar approach}


Lastly, I present a very ``scared'' approach. We concentrate all
the Wigner function in the shortest periodic orbit 
(figure \ref{OneScar})
or in the smallest three (figure \ref{ThreeScars}). The
former is exactly the same as the Weyl Function for an harmonic
oscillator with that energy and period, as it is only
dependant on the shape of the Wigner quasi-distribution, even
with our noisy approach. Thus, the imaginary part
disappears for a good enough sample.
 The Three Scars approach seems more
adequate. The periodic orbits chosen as scars are shown in the figure 
\ref{TresDiagram}. Roughly, each scar has an intensity inversely
proportional to its period, so we have taken a sampling in such
fashion for the  figures \ref{ThreeScars} and \ref{ThreeScarsNodal}.
   



\begin{figure}
\begin{center}
  \includegraphics[width=0.8\textwidth]{TresScars_0821_Diagram.png}
\caption{The three shortest periodic orbits (all them unstable) for
the energy value $E=0.821$. The vertical one is simply harmonic with 
period $2 \pi$. The one that starts at $q_y=0.48854$ has period
$11.66$ and the other $6.48$. }
\label{TresDiagram}
\end{center}
\end{figure}



\begin{figure}
\begin{center}
  \includegraphics[width=0.8\textwidth]{EScarNoizy_0821_Real-Weyl.png}
\caption{The One Scar Weyl function, in all regards exactly the
same that one would obtain for an harmonic oscillator. Nodal lines for
the real part are approximately well located in comparison with figure 
\ref{RealWeyl}.}
\label{OneScar}
\end{center}
\end{figure}



\begin{figure}
\begin{center}
\subfigure[Real Part]{\includegraphics[width=0.45\textwidth]
{TresScars_0821_Real_Weyl.png}
\label{tresRealWeyl}}
\subfigure[Imaginary Part]{\includegraphics[width=0.45\textwidth]
{TresScars_0821_Imag_Weyl.png}
\label{tresImagWeyl}}
\caption{Real and Imaginary parts for the Weyl function for
the Three Scars approach. }
\label{ThreeScars}
\end{center}
\end{figure}





\begin{figure}
\begin{center}
  \includegraphics[width=0.8\textwidth]{TresScars_0821_Weyl0-0-Countour.pdf}
\caption{Nodal line intersections for the same function as the 
previous figure.}
\label{ThreeScarsNodal}
\end{center}
\end{figure}



\bibliography{ziegos}

\end{document}
