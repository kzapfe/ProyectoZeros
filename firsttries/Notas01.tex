\documentclass[a4paper,10pt]{article}

\usepackage[utf8]{inputenc}
\usepackage{amsmath,amssymb}
\usepackage{graphicx}
\usepackage{subfigure}
%\usepackage[spanish]{babel}
\usepackage{bm}

\bibliographystyle{alpha}


\newcommand{\xfase}{\mathbf{x}}
\newcommand{\qfase}{\mathbf{q}}
\newcommand{\pfase}{\mathbf{p}}
\newcommand{\xifase}{ {\boldsymbol{\xi}} }
\newcommand{\mufase}{ {\boldsymbol{\mu}} }
\newcommand{\Ifase}{\mathbf{I}}
\newcommand{\Pfase}{\mathbf{P}}
\newcommand{\Scat}{\mathbf{S}}
\newcommand{\Jsimp}{\mathbf{J}}
\newcommand{\Dom}{\mathbb{D}}
\newcommand{\Var}{\mathbb{M}}
\newcommand{\bra}[1]{\langle #1|}
\newcommand{\ket}[1]{|#1\rangle}
\newcommand{\braket}[2]{\langle #1|#2\rangle}
\newcommand{\Prom}[2]{\langle #1\rangle_{#2}}




\DeclareMathOperator*{\cod}{cod}
\DeclareMathOperator*{\traza}{traza}


\title{Better Wigner Function for Chaotic Systems}
\author{ W. P. Karel Zapfe\\CBPF}


\begin{document}

\maketitle

\begin{abstract}

The Wigner Function is used as a cuasi-distribution in phase space analysis
of Quantum Mechanics. Its Fourier Symplectic Transform, the Weyl Function, 
acts as a generator of momenta of a cuasi classical distribution. It is
assumed that a good aproximation for the former is a Dirac delta over
the Classical Energy Surface. This approximation is essentially wrong,
nevertheless it gives some correct results.
In this work we search for a better uniform approximation, which
should respects the old results in the limit $\hbar\rightarrow 0$.
The essential prerequisite for this seems to be the ergodicity of
the classical system underneath the QM.

\end{abstract}

\section{Preeliminary Stuff}

\subsection{Notation}

We will adopt Alfredo Ozorio de Almeida notation throughout this work. 
In particular, a point in phase space has the 
following ordering  $\xfase:=(p,q)$, and its units are $[\sqrt{\hbar}]$.
The number $d$ is the number of degrees of freedom.

\subsection{Wigner and Weyl Representations}

The Wigner or center Function, $W(\xfase)$ is the representation
of a QM  state on classical phase space. One of its
most important features is the fact that it acts allmost as a 
probability function, in the sense that it  can be used to 
obtain the expected values of Hermitian operators. That is expressed
in the following manner: if $g(\xfase)$ is the Wigner representation
of a well ordered operator $\hat{G}$ 
(so called \emph{``Weyl ordering''}) , 
then:


\begin{equation}
\Prom{\hat{G}}{}=\int d\xfase W(\xfase) g(\xfase)
\end{equation}

over the energy surface. As such, the form

\begin{equation}
 d\xfase W(\xfase) = d \mu
\end{equation}

acts as a integration measure over the phase space \cite{Moyal49}.

It is often assumed that a good aproximation for
a Quantum Ergodic System would be:
\begin{equation}
  \bra{n}f_c(\hat{p},\hat{q}) \ket{n}\approx \int d\xfase 
\delta(H(\xfase)-E_n) f_c(p,q).
\end{equation}

This makes sense in the limit of high energy or small $\hbar$ 
\cite{Berry77}. For a classicaly ergodic, mixing, 
hyperbollic subsystem of the phase space this seems to make
sense, after all, ergodicity means that in a dynamicall system

\begin{equation}
 \lim_{T\rightarrow \infty} \frac{1}{T}\int_0^T f(\xfase(t)) dt =
 \int_\Var f(\xfase) d\mu.
\end{equation}

for every well behaved phase space funcion $f$. Mixing would mean
that every measurable subset of classical initial conditions would
be evenly spread over all the phase space ( or the energy surface
for conservative systems), so, any propierty of these subset is
``spread out'' for the whole Lagrangian surface. 
Three observations come to mind: 
\begin{enumerate}
\item In conservative systems energy surfaces do not mix between each other.
Every one is a separate dynamical system. When  we speak of ergodicity,
we mean ergodicity restricted to each folia of the system. 
In very sharp contrast, in QM the different energy eigenstates
can affect each other. This is the reason for the ``Dirac
Delta assumption'' to be wrong. 
\item Ergodicity is only well defined for compact phase spaces. 
Trully chaotic (hyperbollic, ergodic, mixing) Hamiltonian 
systems are rare.
To my mind come only two examples: geodesic flow on a two dimensional
surface of negative curvature \cite{Anosov} and 
Bunimovich-type billiards \cite{Bunim90}. The second
one is a very crazy example: altough it corresponds to a non-generic,
non-continious potential function, is the one who has been more
studied experimentally. The other experimental setups which are used
for dealing with
Quantum Chaos are the phase space of non trivial molecules.
 
\end{enumerate}



para $n$ suficientemente grande. 

Dado que la función de Wigner actúa como una cuasi distribución,
la gente asume que en este caso 

\begin{equation}
W(\xfase)\approx \delta(H(\xfase)-E_n)
\end{equation}

Esto parece ser una aceptable aproximación,, e inclusive otorga 
una seríe de resultados correctos, como en \cite{Berry78}.
Sin embargo no hay manera de que está aproximación sea correcta.
Existen momentos de las distribuciones que resultan 
francamente errados. A pesar de eso, M. Berry obtiene resultados 
aceptables para las funciones de correlación.
Una expresión más descente (una \emph{aproximación uniforme}) sería
la siguiente:

\begin{equation}
\begin{split}
W(\xfase)&=\sum_n \exp((-E_n -E)/d^2) \int d\xfase \delta(H(\xfase)-E_n)\\
&\approx Ai(H(\xfase)-E).
\end{split}
\end{equation}

Ahora bien, el supuesto incorrecto tiene una implicación maluca
en las correlaciones locales, digamos de orden $\epsilon$
\begin{equation}
\psi^*(q+\epsilon)\psi(q-\epsilon)=\int dp W(q,p)e^{-ip\epsilon/\hbar}
\end{equation}
y
\begin{equation}
C_q(\epsilon)=\int_{local} dq \psi^*(q+\epsilon)\psi(q-\epsilon)
\end{equation}

Ahora bien, el hecho de que la transformada de una función kernel
en el espacio fase clásico realmente corresponda a un
operador Hermitiano cuántico depende del orden de sus
elementos, en lo que se conoce como el ordenamiento de Weyl.
Típicamente esto hace referencia a que los elementos de la
función como serie de potencias deben tomar en cuenta el orden, ya 
que el conmutador afecta las cuentitas.


\section{Objetivos}

Vamos a pensar que en sistemas ergódicos (y probablemente mezclantes)
La aproximación de la delta no es tan mala, pero olvida parte
de las relaciones cuánticas importantes. Esto es, vamos a suponer
que la función de cuerdas \emph{se parece} a la transformada de
Fourier de una delta, siempre y cuando las cuerdas sean suficientemente
pequeñas (es decir, correlaciones cercanas):

\begin{equation}
\chi(\xifase)\xrightarrow{\xifase\rightarrow 0}
\int d\xfase \delta(H(\xfase)-E_n) e^{i\xifase\wedge\xfase/\hbar}
\end{equation}

Sabemos que la función de cuerdas funciona como una función 
generadora de momentos estadísticos. Vamos a ver que tal funciona eso:

\begin{equation}
\bra{n}\hat{f}^l\ket{n}=\frac{i^l \partial ^l}{\hbar^l \partial q^l}
\chi(\xifase) \lvert_{q=0}
\end{equation}

Usaremos esto para calcular las correlaciones $C_q(\epsilon)$.
Finalmente calcularemos los puntos ciegos cerca del origen.


\bibliography{ziegos}

\end{document}
