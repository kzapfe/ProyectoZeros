\documentclass[a4paper,12pt]{article}

\usepackage[utf8]{inputenc}
\usepackage{amsmath,amssymb}
\usepackage{graphicx}
\usepackage{subfigure}
%\usepackage[spanish]{babel}
\usepackage{bm}

\usepackage[cm]{fullpage}
\usepackage[light]{antpolt}
\usepackage[T1]{fontenc}


\bibliographystyle{alpha}

\newcommand{\ihb}{\frac{i}{\hbar}}
\newcommand{\xfase}{\mathbf{x}}
\newcommand{\qfase}{\mathbf{q}}
\newcommand{\pfase}{\mathbf{p}}
\newcommand{\xifase}{ {\boldsymbol{\xi}} }
\newcommand{\mufase}{ {\boldsymbol{\mu}} }
\newcommand{\Ifase}{\mathbf{I}}
\newcommand{\Pfase}{\mathbf{P}}
\newcommand{\Scat}{\mathbf{S}}
\newcommand{\Jsimp}{\mathbf{J}}
\newcommand{\Dom}{\mathbb{D}}
\newcommand{\Var}{\mathbb{M}}
\newcommand{\bra}[1]{\langle #1|}
\newcommand{\ket}[1]{|#1\rangle}
\newcommand{\braket}[2]{\langle #1|#2\rangle}
\newcommand{\Prom}[2]{\langle #1\rangle_{#2}}


\DeclareMathOperator*{\cod}{cod}
\DeclareMathOperator*{\traza}{traza}


\title{ Wigner Function Approximations for Chaotic Systems}
\author{\\CBPF}


\begin{document}

\maketitle

\begin{abstract}


The Center or Wigner Function is used as a quasi-distribution in 
the phase space analysis
of Quantum Mechanics. Its Fourier Symplectic Transform, 
the Chord or Weyl Function, 
acts as the generator of momenta of the former 
 quasi-distribution. It is
assumed that a good approximation for the Center Function
 is a Dirac delta over
the Classical Energy Surface. This approximation is essentially wrong,
even if it catches some general results.
In this work we search for a better approximation, which
should respects the old results in the limit $\hbar\rightarrow 0$.
The essential prerequisite for this seems to be the ergodicity of
the classical system underneath the QM.
Our very coarse approximation is based on a 
numeric argument: Dirac Delta sums imitate the discretitation
used internaly by computers, and Montecarlo random 
selection of such discretization works better than the 
homogeneous one. 

\end{abstract}

\section{Preliminary Stuff}

\subsection{Notation}

We shall adopt Alfredo Ozorio de Almeida notation throughout this work. 
In particular, a point in phase space has the 
following ordering  $\xfase:=(p,q)$, and its units are $[\sqrt{\hbar}]$.
The number $d$ is the number of degrees of freedom. A capital $N$ shall
be used for indicating the Energy Level in which we are interested.
We shall denote random  variables with a superscript asterisk, e.g.
$\xfase^*$. When using polar coordinates on a symplectic plane
the convention is $\xfase=(p,q)=(|x|\sin\theta, |x|\cos\theta)$.

\section{A ``Chaotic'' Wigner Function}

The usual assumption is that if we have an ergodic, mixing and hyperbolic
classical Hamiltonian system (henceforth a \emph{chaotic system}), the
Center Function will be a Dirac delta distribution localized over the
energy eigensurface, provided that we study  a large 
energy state \cite{Voros76}:

\begin{equation}
W_N(\xfase)=\delta (H(x)-E_N)
\end{equation}

This corresponds to the notion that ergodic systems tend to 
homogenize themselves and that eigenstates of energy are
stationary states.

This assumption is essentially wrong, as it fails to account
for important quantum features of pure states. An important example would
be the so called ``blind spots'' \cite{Zambrano09} 
of the auto-correlation function.  This autocorrelation can be 
expressed by either the Center or The Chord function:

\begin{equation}
C(\xifase)=(2\pi\hbar)^d \int W(\xfase)W(\xfase-\xifase) d\xfase
=\frac{1}{(2\pi\hbar)^d}\int |\chi(\mufase)|^2 e^{i\mufase\wedge\xifase/\hbar}
d\mufase
\end{equation}

If the Center function, $W(\xfase)$ where really a 
one dimensional Dirac delta whose argument is
the value of the Hamilton Function, 
then that Correlation would be zero everywhere except for 
the trivial translation,  $\xifase=0$. This is completely
inadequate for delicate interference effects 
like the ones seen on \cite{Zambrano08}.
The only necessary postulation involved in those
interesting wave-like effectos  is that the
state is ``highly excited'' (high value of the eigenenergy) 
and is ``large'' (its Center Function spreads over a significant 
area in
the classical phase space).

\section{The Work}

We shall start with a very crude assumption.
 The Wigner Function can be very roughly approximated
by a random distribution of points near the energy surface. 
If we have $M$ such points, then
he spread of the points should not occupy a volume larger than
$\hbar/M$, and the volume surrounded by the energy surface
should be from the order of $N\hbar$, $N$ being 
the energy level, 
a number
of the order of the hundreds or thousands.  In this rawest
approximation the points will be over a spherical shell of
radius of the order $\sqrt{N\hbar}$. In order to have
a chaotic system we need at least two degrees of freedom.
But we can make tests for a single degree of freedom,
corresponding to a Poincaré or Stroboscopic 
symplectic map. 


The points, distributed among energy shells for
a crude spherical approximation, will be between
two energy shells:
\begin{equation}
N\hbar< \|\xfase\|^2<(N+1)\hbar,
\end{equation}
The difference in the radius is then
$(\sqrt{N+1}-\sqrt{N})\hbar=\Delta\hbar$.


\subsection{Crude 1 d.o.f. test}

We shall test the Berry Dirac Delta result \cite{BerryRIR} 
against our 
small volume approach. Our criteria will be the lowest
(nearer to the origin) nodal lines. We will compare
the completely isotropic Bessel Function with the
chord function obtained. The center function would be then:
\begin{equation}\label{AproxCruda}
W(\xfase)=\frac{1}{M}\sum_{\xfase^*} \delta (\xfase - \xfase^*)
\end{equation}
Then the corresponding Chord functions is a random superposition
of plane waves. The wave vectors of these components are
the centers of the Dirac Deltas:
\begin{equation}\label{chordfun}
\chi(\xifase)=\frac{1}{M}\sum_{\xfase^*} \exp \left( \ihb \xifase\wedge\xfase^* \right).
\end{equation}
If this approach would made any sense,
then, in the limit of large $M$ and small $\hbar$
the last expression would converge to:
\begin{equation}\label{BerryApproach}
\chi(\xifase)=J_0(|\xifase|m(E)^{1/2}/\hbar),
\end{equation}
where $J_0(x)$ is the standard Bessel function of the first kind.  
It is clear that for large enough $M$ the expression 
in eq. \ref{chordfun} should become
isotropical, i.e., not depend on the direction of $\xifase$ but
only in its magnitude. 

A look on the next pictures (figures \ref{XiFunction01},
\ref{BesselCompare01})
reveal that this may be the case. 
We reduce  the chords function
to a two dimensional $\xifase$ space. The original 
Wigner function looks like the figure \ref{WigExample}.
For the nodal line comparison, due to
the asymptotic central symmetry, we shall use only the 
real part of $\chi(\xifase)$.


\begin{figure}
\begin{center}
  \includegraphics[width=0.6\textwidth]{CondiniExemplo01.png}
\caption{ A realization of the Wigner
function in eq. \ref{AproxCruda}}\label{WigExample}
\end{center}
\end{figure}

\begin{figure}
\begin{center}
\subfigure[Chord Function, Real part]{
\includegraphics[width=0.4\textwidth]{300_Chords_Real.png}
\label{xireal01}
}
\subfigure[Chord Function, Imaginary part]{
\includegraphics[width=0.4\textwidth]{300_Chords_Imag.png}
\label{xiimag01}
}
\subfigure[Chord Function, Amplitude]{
\includegraphics[width=0.4\textwidth]{300_Chords_Ampli.png}
\label{xiampl01}
}
\subfigure[Chord Function, Phase]{
\includegraphics[width=0.4\textwidth]{300_Chords_Fase.png}
\label{xifase01}
}
\caption{The Chord Function for the random distribution of deltas. 
Here the energy is $300$ arbitrary units. Note the difference between
the scales in the real and imaginary part. The latter has a structure much
closer to zero, while the former is highly (and suspiciously) regular.}
\label{XiFunction01}
\end{center}
\end{figure}


\begin{figure}
\begin{center}
\subfigure[Radial Bessel Function, exact]{
\includegraphics[width=0.4\textwidth]{BesselRadial01.png}
\label{xibessel}
}
\subfigure[Chord Function for 500 random points]{
\includegraphics[width=0.4\textwidth]{CuerdaspaceZerosLowData.png}
\label{xi500}
}
\subfigure[Chord Function for 5000 random points]{
\includegraphics[width=0.4\textwidth]{CuerdaspaceZeroMediumData.png}
\label{xi5000}
}
\subfigure[Chord Function for 50'000 random points]{
\includegraphics[width=0.4\textwidth]{CuerdaspaceZeroHighData.png}
\label{xi50000}
}
\caption{The Bessel function of equation \ref{BerryApproach},
and different numerical realizations of the
Chord Function for the random distribution of Dirac deltas.
Here the energy level is $300$. It is clear that  
after 5000 random points in the expression \ref{AproxCruda},
we have at least the two first nodal lines within the correct
order. Black corresponds to zero values (nodal lines). 
The contrast has been greatly exaggerated in these figures.}
\label{BesselCompare01}
\end{center}
\end{figure}


\begin{figure}
\begin{center}
\subfigure[$M=200$]{
\includegraphics[width=0.4\textwidth]{200_ZerosBandN.pdf}
\label{zero200}
}
\subfigure[$M=2000$]{
\includegraphics[width=0.4\textwidth]{2000_ZerosBandN.pdf}
\label{zero2000}
}
\subfigure[$M=20000$]{
\includegraphics[width=0.4\textwidth]{20000_ZerosBandN.pdf}
\label{zero20000}
}
\subfigure[$M=200000$]{
\includegraphics[width=0.4\textwidth]{200000_ZerosBandN.pdf}
\label{zero200000}
}
\caption{The location of the nodal lines for the Bessel
function in red, fine line, and for diverse 
Dirac Delta realizations in blue thick points. We see a good
agreement between all the ``blind lines'' presented.
The slight offset of the first one could account for the
width of the center function Distribution.}
\label{BesselCompareZeros}
\end{center}
\end{figure}

\subsection{2 d.o.f. Spherical Shell}

Most of the previous reasoning works well for 
the 2 d.o.f. spherical shell. An important 
change is that the Bessel Function in the formula
\ref{BerryApproach} changes to the Bessel-1 instead
of the Bessel-0 function. 
For the numerics we have used the following
set of coordinates for the Volume between two 
3-spheres: 
\begin{equation}\label{4ballcordinates1}
\begin{split}
p_1&=(R+l^*)\sqrt{\hbar}\cos \phi_1^* \\
q_1&=(R+l^*)\sqrt{\hbar}\sin \phi_1^* \cos\phi_2^*\\
p_2&=(R+l^*)\sqrt{\hbar}\cos \phi_1^*\sin\phi_2^*\cos\phi_3^* \\
q_2&=(R+l^*)\sqrt{\hbar}\cos \phi_1^* \sin\phi_2^*\sin\phi_3^* \\,
\end{split}
\end{equation}
in center space, and 
\begin{equation}\label{4ballcordinates2}
\begin{split}
\xifase_{p,1}&= \rho\sqrt{\hbar}\cos \theta_1 \\
\xifase_{q,1}&= \rho\sqrt{\hbar}\sin \theta_1 \cos\theta_2\\
\xifase_{p,2}&= \rho\sqrt{\hbar}\cos \theta_1\sin\theta_2\cos\theta_3 \\
\xifase_{p,2}&= \rho\sqrt{\hbar}\cos \theta_1 \sin\theta_2\sin\theta_3 \\,
\end{split}
\end{equation}
in the chord space.

We present here the zeros found for many sizes of
the Dirac Delta sum. In the figure \ref{ZerosContraPrecision}
we show that the first zero can be trustfully found with an
ensemble of 2000 or more Dirac Deltas in Center space.

\begin{figure}
\begin{center}
  \includegraphics[width=0.9\textwidth]{Zero1Bessel1.pdf}
\caption{The First Zero of the $J_1(\rho)$ Function (black line),
  against the ones obtained for $N=20,200,2000, 20'000$. The
value for the various $\theta_i$ coordinates is fixed. }
\label{ZerosContraPrecision}.
\end{center}
\end{figure}

Then in figures \ref{Zeros2000} and \ref{Zeros20000} 
we proceed to show the irregularities due to the
not completely symmetric distribution, compared with 
the first zero of the Bessel Function.

\begin{figure}
\begin{center}
  \includegraphics[width=0.9\textwidth]{2000_Deltas3Sphere.pdf}
\caption{The First Zero of the $J_1(\rho)$ Function (black line),
  against the ones obtained for $N=2000$, in various 
sections for the $\theta_1$ (indicated by color) and $\theta_2$ 
(projected).}
\label{Zeros2000}
\end{center}
\end{figure}
 

\begin{figure}
\begin{center}
  \includegraphics[width=0.9\textwidth]{20000_Deltas3Sphere.pdf}
\caption{The First Zero of the $J_1(\rho)$ Function (black line),
  against the ones obtained for $N=20000$, in various 
sections for the $\theta_1$ (indicated by color) and $\theta_2$ 
(projected).}
\label{Zeros20000}
\end{center}
\end{figure}
 
Finally, we check the heavier simulation against the first six
zeros of $J_q(x)$, in the figure \ref{SeisZeros}.


\begin{figure}
\begin{center}
  \includegraphics[width=0.9\textwidth]{20000_Deltas3Sphere6Zeros.pdf}
\caption{The first six Zeros of the $J_1(\rho)$ Function (black lines),
  against the ones obtained for $N=20000$, in various 
sections for the $\theta_1$ (indicated by color) and $\theta_2$ 
(projected).}
\label{SeisZeros}
\end{center}
\end{figure}


\bibliography{ziegos}

\end{document}
