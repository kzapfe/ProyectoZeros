\documentclass[a4paper,10pt]{article}

\usepackage[utf8]{inputenc}
\usepackage{amsmath,amssymb}
\usepackage{graphicx}
\usepackage{subfigure}
%\usepackage[spanish]{babel}
\usepackage{bm}

\bibliographystyle{alpha}

\newcommand{\ihb}{\frac{i}{\hbar}}
\newcommand{\xfase}{\mathbf{x}}
\newcommand{\qfase}{\mathbf{q}}
\newcommand{\pfase}{\mathbf{p}}
\newcommand{\xifase}{ {\boldsymbol{\xi}} }
\newcommand{\mufase}{ {\boldsymbol{\mu}} }
\newcommand{\Ifase}{\mathbf{I}}
\newcommand{\Pfase}{\mathbf{P}}
\newcommand{\Scat}{\mathbf{S}}
\newcommand{\Jsimp}{\mathbf{J}}
\newcommand{\Dom}{\mathbb{D}}
\newcommand{\Var}{\mathbb{M}}
\newcommand{\bra}[1]{\langle #1|}
\newcommand{\ket}[1]{|#1\rangle}
\newcommand{\braket}[2]{\langle #1|#2\rangle}
\newcommand{\Prom}[2]{\langle #1\rangle_{#2}}


\DeclareMathOperator*{\cod}{cod}
\DeclareMathOperator*{\traza}{traza}


\title{ Wigner Function Approximations for Chaotic Systems}
\author{\\CBPF}


\begin{document}

\maketitle

\begin{abstract}


The Center or Wigner Function is used as a quasi-distribution in 
phase space analysis
of Quantum Mechanics. Its Fourier Symplectic Transform, 
the Chord or Weyl Function, 
acts as the generator of momenta of the former 
classical quasi-distribution. It is
assumed that a good approximation for the former is a Dirac delta over
the Classical Energy Surface. This approximation is essentially wrong,
even if it catches some adequate results.
In this work we search for a better uniform approximation, which
should respects the old results in the limit $\hbar\rightarrow 0$.
The essential prerequisite for this seems to be the ergodicity of
the classical system underneath the QM.

\end{abstract}

\section{Preliminary Stuff}

\subsection{Notation}

We will adopt Alfredo Ozorio de Almeida notation throughout this work. 
In particular, a point in phase space has the 
following ordering  $\xfase:=(p,q)$, and its units are $[\sqrt{\hbar}]$.
The number $d$ is the number of degrees of freedom.


\section{A ``Chaotic'' Wigner Function}

The usual assumption is that if we have an ergodic, mixing and hyperbolic
classical Hamiltonian system (henceforth a \emph{chaotic system}), the
Center Function will be a Dirac delta distribution localized over the
energy eigensurface, provided that we study  a large energy state:

\begin{equation}
W_0(\xfase)=\delta (H(x)-E)
\end{equation}

This corresponds to the notion that ergodic systems tend to 
homogenize themselves and that eigenstates of energy are
stationary states.

This assumption is essentially wrong, as it fails to account
for important quantum features of pure states. An important example would
be the so called ``blind spots'' \cite{Zambrano09} 
of the auto-correlation function.  This function is given by:

\begin{equation}
C(\xifase)=(2\pi\hbar)^d \int W(\xfase)W(\xfase-\xifase) d\xfase
=\frac{1}{(2\pi\hbar)^d}\int |\chi(\mufase)|^2 e^{i\mufase\wedge\xifase/\hbar}
d\mufase
\end{equation}

If the Center function, $W(\xfase)$ where really a Dirac delta,
then that Correlation would be zero everywhere except for 
the trivial translation,  $\xifase=0$. This is completely
inadequate for delicate interference effects \cite{Zambrano08}.
The only necessary postulation involved here is that the
State is ``highly excited'' (high value of the eigenenergy) 
and is ``large'' (its Wigner Functions spreads over a lot of area in
the classical phase space).

\section{The Work}

We shall start with a very crude assumption.
 The Wigner Function can be very roughly approximated
by a random distribution of points near the energy surface. 
The spread of the points should not occupy a volume larger than
$\hbar^d$, and the volume surrounded by the energy surface
should be from the order of $N\hbar^d$, $N$ being a number
of the order of the hundreds or thousands.  In this cruddiest
approximation the points will be over a spherical shell of
radius of the order $\sqrt{N\hbar}$. 
\begin{equation}\label{AproxCruda}
W(\xfase)=\sum \delta (\xfase - \xfase^*)
\end{equation}
The Chord function would then be a random superposition
of plane waves.
\begin{equation}
\chi(\xifase)=\sum \exp \left( \ihb \xifase\wedge\xfase^* \right)
\end{equation}

For our purposes we used $5\times 10^5$ random $\xfase^*$
points over the energy surface and $\hbar=1$. 
Lesser values for the distribution gave variations in
the result.



\begin{figure}
\begin{center}
\subfigure[Chord Function, Real part]{
\includegraphics[width=0.49\textwidth]{300_Chords_Real.png}
\label{xireal01}
}
\subfigure[Chord Function, Imaginary part]{
\includegraphics[width=0.49\textwidth]{300_Chords_Imag.png}
\label{xiimag01}
}
\subfigure[Chord Function, Amplitude]{
\includegraphics[width=0.49\textwidth]{300_Chords_Ampli.png}
\label{xiampl01}
}
\subfigure[Chord Function, Imaginary part]{
\includegraphics[width=0.49\textwidth]{300_Chords_Fase.png}
\label{xifase01}
}
\caption{The Chord Function for the random distribution of deltas. 
Here the energy is $300$ arbitrary units. Note the difference between
the scales in the real and imaginary part. The latter has a structure much
closer to zero, while the former is highly (and suspiciously) regular.}
\label{XiFunction01}
\end{center}
\end{figure}


\begin{figure}
\begin{center}
\subfigure[Chord Function, Real part]{
\includegraphics[width=0.49\textwidth]{500_Chords_Real.png}
\label{xireal02}
}
\subfigure[Chord Function, Imaginary part]{
\includegraphics[width=0.49\textwidth]{500_Chords_Imag.png}
\label{xiimag02}
}
\subfigure[Chord Function, Amplitude]{
\includegraphics[width=0.49\textwidth]{500_Chords_Ampli.png}
\label{xiampl02}
}
\subfigure[Chord Function, Imaginary part]{
\includegraphics[width=0.49\textwidth]{500_Chords_Fase.png}
\label{xifase02}
}
\caption{The Chord Function for the random distribution of deltas. 
Here the energy is $300$ arbitrary units. Note the difference between
the scales in the real and imaginary part. The latter has a structure much
closer to zero, while the former is highly (and suspiciously) regular.}
\label{XiFunction02}
\end{center}
\end{figure}




\begin{figure}
\begin{center}
\subfigure[Chord Function, Real part]{
\includegraphics[width=0.49\textwidth]{750_Chords_Real.png}
\label{xireal03}
}
\subfigure[Chord Function, Imaginary part]{
\includegraphics[width=0.49\textwidth]{750_Chords_Imag.png}
\label{xiimag03}
}
\subfigure[Chord Function, Amplitude]{
\includegraphics[width=0.49\textwidth]{750_Chords_Ampli.png}
\label{xiampl03}
}
\subfigure[Chord Function, Imaginary part]{
\includegraphics[width=0.49\textwidth]{750_Chords_Fase.png}
\label{xifase03}
}
\caption{The Chord Function for the random distribution of deltas. 
Here the energy is $7500$ arbitrary units. Note the difference between
the scales in the real and imaginary part. The latter has a structure much
closer to zero, while the former is highly (and suspiciously) regular.}
\label{XiFunction03}
\end{center}
\end{figure}




\begin{figure}
\begin{center}
\subfigure[Chord Function, Real part]{
\includegraphics[width=0.49\textwidth]{1000_Chords_Real.png}
\label{xireal04}
}
\subfigure[Chord Function, Imaginary part]{
\includegraphics[width=0.49\textwidth]{1000_Chords_Imag.png}
\label{xiimag04}
}
\subfigure[Chord Function, Amplitude]{
\includegraphics[width=0.49\textwidth]{1000_Chords_Ampli.png}
\label{xiampl04}
}
\subfigure[Chord Function, Imaginary part]{
\includegraphics[width=0.49\textwidth]{1000_Chords_Fase.png}
\label{xifase04}
}
\caption{The Chord Function for the random distribution of deltas. 
Here the energy is $1000$ arbitrary units. Note the difference between
the scales in the real and imaginary part. The latter has a structure much
closer to zero, while the former is highly (and suspiciously) regular.}
\label{XiFunction04}
\end{center}
\end{figure}




\begin{figure}
\begin{center}
\subfigure[Chord Function, Real part]{
\includegraphics[width=0.49\textwidth]{1500_Chords_Real.png}
\label{xireal05}
}
\subfigure[Chord Function, Imaginary part]{
\includegraphics[width=0.49\textwidth]{1500_Chords_Imag.png}
\label{xiimag05}
}
\subfigure[Chord Function, Amplitude]{
\includegraphics[width=0.49\textwidth]{1500_Chords_Ampli.png}
\label{xiampl05}
}
\subfigure[Chord Function, Imaginary part]{
\includegraphics[width=0.49\textwidth]{1500_Chords_Fase.png}
\label{xifase05}
}
\caption{The Chord Function for the random distribution of deltas. 
Here the energy is $1500$ arbitrary units. Note the difference between
the scales in the real and imaginary part. The latter has a structure much
closer to zero, while the former is highly (and suspiciously) regular.}
\label{XiFunction05}
\end{center}
\end{figure}



\begin{figure}
\begin{center}
\subfigure[Chord Function, Real part]{
\includegraphics[width=0.49\textwidth]{3000_Chords_Real.png}
\label{xireal06}
}
\subfigure[Chord Function, Imaginary part]{
\includegraphics[width=0.49\textwidth]{3000_Chords_Imag.png}
\label{xiimag06}
}
\subfigure[Chord Function, Amplitude]{
\includegraphics[width=0.49\textwidth]{3000_Chords_Ampli.png}
\label{xiampl06}
}
\subfigure[Chord Function, Imaginary part]{
\includegraphics[width=0.49\textwidth]{3000_Chords_Fase.png}
\label{xifase06}
}
\caption{The Chord Function for the random distribution of deltas. 
Here the energy is $3000$ arbitrary units. Note the difference between
the scales in the real and imaginary part. The latter has a structure much
closer to zero, while the former is highly (and suspiciously) regular.}
\label{XiFunction06}
\end{center}
\end{figure}



\section{A possible explanation}

The nodal lines of the real part seem to be concentric circles. As a matter
of fact the whole real part seems to have rotational symmetry, which
is broken for lower energies. The imaginary part tends to very low values
because we are close to a point-symmetric state. The approximation of
the high energy eigensurface as a spherical shell provides this symmetry.


If we make the approximation \ref{AproxCruda}, the points $\xfase^*$
would be randomly  distributed in between two energy shells. Using the
spherical model for the shell, that means that each $\xfase^*$ looks
like this:

\begin{equation}
\xfase*=(R+l^*)\sqrt{\hbar}(\sin \phi^*, \cos\phi^*)
\end{equation}

where $l^*$ is randomly distributed over the unit interval,
and $\phi^*$ over the circle.
This looks more or less like the figure \ref{WigExample}.
The superposition of the waves thus obtained will be

\begin{equation}
\chi(\xifase)=\sum_{l^*,\phi^*}\frac{1}{N}
 \exp \left( \ihb(R+l^*) 
(\xifase_p \cos\phi^*- \xifase_q \sin\phi^* ) \right)
\end{equation}

This can be approximated by an integral over the domain
of the random variables. We decompose $\xifase$
in polar coordinates as $\xifase=a(\sin\theta,\cos\theta)$,
where $a$ is element from the unit interval,
and $\theta$ winds around the circle.


\begin{figure}
\begin{center}
  \includegraphics[width=0.6\textwidth]{CondiniExemplo01.png}
\caption{ A realization of the Wigner
function in eq. \ref{AproxCruda}}\label{WigExample}
\end{center}
\end{figure}

\begin{equation}
\chi(\xifase)\approx \int_{-\pi}^\pi \int_0^1 dl d\phi
 \exp \left( a\ihb(R+l) 
(\sin\theta \cos\phi- \cos\theta \sin\phi ) 
\right)
\end{equation}

Let us deal first with the Real part:

\begin{equation}
\chi(\xifase)_{real}\approx \int_{-\pi}^\pi \int_0^1 dl d\phi
 \cos \left( a(R+l)/\hbar 
(\sin\theta \cos\phi- \cos\theta \sin\phi ) 
\right)
\end{equation}


The integral for the $l$ variable is easy:

\begin{equation}\label{realchord}
\chi(\xifase)_{real}\approx \int_{-\pi}^\pi  d\phi
\frac{ \sin \left( a(R+l)/\hbar \sin (\theta-\phi) 
\right)}
{a(R+l)/\hbar \sin (\theta-\phi)}
\end{equation}

The other becomes slightly annoying, as it has nested
trigonometric functions. We are interested in the zeros,
so a visual clue could help.

The integrand is a function of the difference of phases,
so we plot it for a value of $a(R+1)/\hbar=30$ to see
its behavior in figure \ref{integrand2}.
This figure let us appreciate that although the integrand
oscillates a lot, is quite regular and the main 
contributions to the integral are few. 
We shall continue in that direction.



\begin{figure}
\begin{center}
\includegraphics[width=0.49\textwidth]{IntegrandoCorte.png}
\caption{ The integrand in eq. \ref{realchord}, for 
the value of $a(R+1)/\hbar=30$. This corresponds to the
$900$th eigenstate.}\label{integrand02}
\end{center}
\end{figure}



\section{Note}

As the function 
\begin{equation}
 \exp \left( \ihb
(\xifase\wedge\xfase ) \right)
\end{equation}
is clearly separable in the case of multidimensional $\xifase$ and
$\xfase$, we may generalize the results afterward with (hopefully) 
little effort. 


\bibliography{ziegos}



\end{document}
